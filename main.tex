\documentclass[a4paper, 12pt,twoside]{report}  % El tamaño por default del texto es de 12 pt., se puede cambiar de aca el GLOBAL

\usepackage[spanish]{babel} % Permite usar caracteres de idiomas derivados del lat\'in
\usepackage[utf8]{inputenc} %paquete para poder escribir acentos en forma normal.

\usepackage[left=1.9cm,top=2.2cm,head=1.8cm,right=1.9cm,bottom=4cm]{geometry} % Permite cambiar a dedo los margenes de la hoja

\usepackage[table]{xcolor}

\linespread{1} % Line spacing

\usepackage{fancyhdr} %este paquete permite poner un encabezado y pie de página.

\usepackage{amssymb} % amssymb y amsmath se usan juntos para ampliar
\usepackage{amsmath} % la cantidad de simbolos matemáticos disponibles

\usepackage{url} % paquete para que la escritura de las URL salga bien.

\usepackage{graphicx} % Permite usar imágenes fuera del modo dvi (como pdf, png, etc...)

\usepackage{caption} %permite poner epígrafes en distintos elementos, la opción [small] es para que sea en letras pequeñas.
\usepackage{subcaption} %permite poner sub-epígrafes para arrays de figuras.

\numberwithin{equation}{subsection} %esto es para que numere las ecuaciones empezando con el número de subsección a la que pertenecen.

\usepackage{lineno}  %este paquete, si lo habilitan, sirve para que numere las líneas. esto sirve a veces para cuando se manda el borrador del informe, para facilitar las correcciones.

\usepackage{multirow} %Sirve para convinar filas en las tablas

\usepackage{tabularx}
%Formato de tablas

\usepackage{ulem}
% Underlining package

\usepackage{hyperref}
%Referencias

\usepackage{pgfplots}
%Creación de gráficos

\usepackage{tikz}  % Carga el paquete TikZ, necesario para dibujar gráficos.
\usetikzlibrary{shapes.geometric, arrows, positioning}  % Carga las bibliotecas necesarias de TikZ: para formas geométricas, flechas y posicionamiento de nodos.

%sagetex: Permite ejecutar código SageMath dentro de un documento LaTeX.
\usepackage{sagetex}

%tikz: Herramientas para crear gráficos, diagramas y figuras en LaTeX.
%tkz-graph: Extensión de TikZ para crear y manipular grafos fácilmente.
%tkz-berge: Conjunto de macros para trabajar con teoría de grafos basada en los algoritmos de Berge.
\usepackage{tikz,tkz-graph,tkz-berge}

% Encabezado y pie de pagina
\pagestyle{fancy}     %Dejando esto ponemos detalles en pie de página  pero no la numeración de cada una
%\pagestyle{plain}    %Dejando esto enumeramos las hojas abajo desde página 2

%Elimina el número de capítulo en el header
%\renewcommand{\chaptermark}[1]{\markboth{#1}{#1}}
%\fancyhead[R]{}
%\fancyhead[L]{\chaptername\ \thechapter\ --\ \leftmark}

\fancyhead[LE,RO]{\thepage}
\fancyhead[RE]{\slshape\nouppercase{\leftmark}}
\fancyhead[LO]{\slshape\nouppercase{\rightmark}}
%Acá damos la parte izquierda del encabezado
%\fancyhead[C]{UTN Santa Fe} %Acá damos la parte central del encabezado
%\fancyfoot[L]{Proyecto Final - UTN FRSF} %Acá damos la parte izquierda del pie de página.
\fancyfoot[RO,LE]{UTN Santa Fe}
\fancyfoot[RE,LO]{E. Sonzogni, L. Sonzogni} %Acá damos la parte  derecha del pie de página..
\fancyfoot[C]{} %Acá damos la parte central del pie de página.

\usepackage{float}

\usepackage{enumitem}

%Copiar texto en otra parte
\usepackage{clipboard}

%Referenciar título de sección
\usepackage{nameref}

%-----------------------------------------------------------------------------------------------------

%\title{Título del trabajo.} %Acá va el título del informe.

% El autor
%\author{Sonzogni, Luciano Daniel
%\\[2mm] 
%\textit{Materia del trabajo - UTN Santa Fe}
%}
%\date{23 de Junio de 1994(Fecha del trabajo)} % La fecha que tambi\'en puede ser escrita como \date

%-----------------------------------------------------------------------------------------------------



\hypersetup{
    colorlinks=true,
    linkcolor=blue,
    citecolor=blue,
    urlcolor=cyan,
    pdfpagemode=FullScreen,
}

\begin{document} %con esto empezamos a escribir el documento, en el formato seleccionado en \documentclass. 
\renewcommand{\abstractname}{}    % Este comando es para borrar el título del resumen, que por defecto se llama ``Abstract'', normalmente no se utiliza título para esto.
\renewcommand\spanishtablename{Tabla} %aca cambiamos el nombre de ``Cuadro'' a ``Tabla'' para que quede más lindo.
%\linenumbers  % con esto activamos la numeración de líneas.

%\maketitle  % Genera el encabezado del articulo





%----------------------------------------------------------------------------------------
%	CARÁTULA
%----------------------------------------------------------------------------------------

\begin{titlepage} % Como el entorno indica, genera la página para la carátula

\newcommand{\HRule}{\rule{\linewidth}{0.5mm}} % Defines a new command for the horizontal lines, change thickness here

 \begin{center} % Centra todo

\textsc{\LARGE \\%[2cm]
UTN Santa Fe}\\[1cm] % Name of your university/college
\textsc{\large Proyecto Final}\\[0.3cm] % Major heading such as course name
%\textsc{\large }\\[0.5cm] % Minor heading such as course title

\HRule \\[1cm]   %Esto pone una línea que no me gusta
\newcommand{\grad}{$^{\circ}$}	% Comando usado para definir ° (sino LateX no lo acepta)

%\vspace*{\fill}

{ \huge \bfseries Proyecto final de carrera\\
Ingeniería en Sistemas de Información}\\[1cm] % Title of your document
%\HRule \\[1.5cm]   %Esto termina la línea dejando el texto entre tales líneas
\end{center}
\\[0cm]

\large

\noindent\emph{Nombre del proyecto:}\\
\indent Diseño e implementación de sistema multiplataforma de control de inventario y de la producción para Imprenta Lux S.A.\\

\\[1cm]

\noindent\emph{Alumnos:}
%Luciano Daniel \textsc{Sonzogni} % Your name

\noindent Emir Rafael Sonzogni\\
\indent Mail: emirsonzogni279@gmail.com\\
\indent LU: 23.216\\
Luciano Daniel Sonzogni\\
\indent Mail: lsonzogni458@gmail.com\\
\indent LU: 24.182\\

\\[1cm]

\noindent\emph{Director de Proyecto:}\\
\indent Esp. I.S.I. Juan Carlos Ramos\\

\\[1cm]

\noindent\emph{Nombre del Cliente:}\\
\indent Imprenta Lux S.A.\\

\\[1cm]

\center{Santa Fe, Argentina - 2025}

\\[1cm]

\includegraphics[scale = 0.4]{Logo.png}

\vfill % Fill the rest of the page with whitespace

\end{titlepage}
\pagebreak

%----------------------------------------------------------------------------------------
%	FIN DE CARÁTULA
%----------------------------------------------------------------------------------------

\definecolor{marca_US_realizada_anterior}{HTML}{8ead98}
\definecolor{marca_US_realizada}{HTML}{cccccc}
\definecolor{marca_US_emir}{HTML}{bebbf0}
\definecolor{marca_US_luciano}{HTML}{bbe1f0}
\definecolor{diferencia_estimacion_negativa}{HTML}{ea9999}
\definecolor{diferencia_estimacion_positiva}{HTML}{8ead98}
\definecolor{riesgo_bajo}{HTML}{ffe599}
\definecolor{riesgo_medio}{HTML}{f9cb9c}
\definecolor{riesgo_alto}{HTML}{ea9999}
\definecolor {processblue}{cmyk}{0.96,0,0,0}
%Definición de colores

%\counterwithout{section}{chapter}
%No tener en cuenta número de capítulo

\pagenumbering{roman} % para la primera parte

% Índice general
\clearpage
\phantomsection
\addcontentsline{toc}{chapter}{Índice general}
\tableofcontents

% Índice de símbolos
\clearpage
\phantomsection
\addcontentsline{toc}{chapter}{Índice de símbolos}
\chapter*{Índice de símbolos}
\markboth{Índice de símbolos}{Índice de símbolos}

\begin{itemize}
\item \hypertarget{INVEST}{INVEST}: bajo el criterio \hyperlink{INVEST}{INVEST}, buenas historias de usuario (\hyperlink{US}{US}) tienen las características: Independiente, Negociable, Valuable, Estimable, Pequeña (Small), Testeable.
\item \hypertarget{MRO}{MRO}: Mantenimiento, reparación y operaciones (Maintenance, repair, and operations).
\item \hypertarget{Pp}{Pp}: Punto de pedido, nivel de inventario en el cual una nueva orden de reposición de un insumo es emitida (también conocido como \textit{reorder point}, ROP).
\item \hypertarget{SP}{SP}: Story Point (generalmente llamado por su nombre en inglés, en ocasiones es traducido como \textit{punto de historia}), medida de esfuerzo utilizada para estimar el trabajo en un proyecto ágil.
\item \hypertarget{US}{US}: Historia de usuario (User Story), forma en que son representados los requerimientos en las metodologías ágiles.
\item \hypertarget{WIP}{WIP}: Trabajo en proceso (Work In Process).
\end{itemize}

% Índice de figuras
\clearpage
\phantomsection
\addcontentsline{toc}{chapter}{Índice de figuras}
\renewcommand{\listfigurename}{Índice de figuras}
\listoffigures

% Índice de tablas
\clearpage
\phantomsection
\addcontentsline{toc}{chapter}{Índice de tablas}
\renewcommand{\listtablename}{Índice de tablas}
\listoftables

\clearpage
\pagenumbering{arabic}
%numbering style for the following chapters

\raggedbottom
%No estira el contenido. Deja espacio en blanco al final si no hay suficiente contenido. Resulta en páginas de altura variable.

\chapter{Introducción}
\indent Imprenta Lux S.A. es una empresa del sector gráfico, especializada en la producción de diversos materiales impresos, tales como libros, revistas, almanaques, volantes y afiches. Tiene su sede principal en Hipólito Yrigoyen 2463 (Santa Fe, Argentina), aunque también cuenta con una sede secundaria dedicada exclusivamente a tareas de producción. Sin embargo, en las dos instalaciones se llevan a cabo procesos productivos, y, a su vez, en ambas existen sectores destinados al depósito de insumos.\\
\indent Para la empresa, la dispersión geográfica suele representar una dificultad significativa en cuanto a la correcta planificación de los proyectos por parte del gerente general, ya que es frecuente que no se cuente con datos actualizados provenientes de relevamientos de disponibilidad de insumos en los depósitos. Esta situación suele generar demoras en las respuestas que deben darse conforme a las necesidades del cliente.\\
\indent El documento que se presenta a continuación describe el proceso llevado a cabo para diseñar e implementar un sistema de información destinado al control de inventario y de la producción con el que se espera resolver la problemática recientemente adelantada. Se dará principal atención al seguimiento de la ejecución del proyecto con el fin de comparar las especificaciones teóricas brindadas en la etapa de análisis inicial, con aquellos resultados que se obtengan de manera empírica. De este modo se espera obtener conclusiones acerca de los distintos parámetros que se crean relevantes en el desarrollo del proyecto, a saber: metodología empleada, riesgos, tiempo, esfuerzo, calidad.\\
\indent La redacción del presente informe abarca el periodo durante el cual se desarrolló el proyecto, finalizando con la evaluación del cumplimiento de los objetivos establecidos para este, los cuales serán oportunamente abordados. Por lo tanto, el intervalo temporal relevante se extiende desde el 01/07/2024 hasta el 25/04/2025.

\chapter{Acerca de la empresa/organización}
\section{Descripción general de la empresa}

La empresa interesada en la realización de este proyecto es Imprenta Lux S.A. Esta pertenece a la industria gráfica, por lo que en sus labores principales se encuentran la impresión de libros, folletos, catálogos, etc. Cuenta con alrededor de 30
empleados y se ubica en Hipólito Yrigoyen 2463 (Santa Fe, Argentina).\\
\indent La empresa es una sociedad anónima, la cual consta de un directorio conformado por sus
accionistas y cuyo presidente es quien posee más acciones de la
misma. El presidente del directorio es quien toma las decisiones organizativas de la empresa, pero para cuestiones de mayor peso las decisiones se disponen a votación de los miembros del directorio.

\section{Estructura jerárquica de la empresa}

Imprenta Lux S.A. está dividida en 6 sectores de trabajo:
\begin{itemize}
\item \underline{Pre\smash{p}rensa:} se encarga de todos los preparativos y especificaciones de los pedidos que luego pasarán a producción, así como de la interacción con el cliente.
\item \underline{Im\smash{p}resión:} se encarga de imprimir los pedidos según las especificaciones de preprensa.
\item \underline{Encuadernación:} da el formato final a los libros (laminar/barnizar tapas, coser hojas, prensar libros y embalarlos).
\item \underline{Mantenimiento:} se encarga de la reparación de máquinas, del control de insumos y de la resolución de otras fallas de carácter técnico en la empresa.
\item \underline{Ex\smash{p}edición:} entrega de pedidos y transporte de insumos.
\item \underline{Contabilidad:} está compuesta por un único contador, que es el tesorero del Directorio y es elegido mediante votación en el mismo. Lleva cuenta de las transacciones de la empresa, pagos a empleados y compra de suministros.
\end{itemize}
\indent La empresa también cuenta con encargados de preprensa, producción y mantenimiento que se
encargan de monitorear los respectivos sectores. A su vez, estos actúan bajo la supervisión del
Gerente general de la empresa, quien actúa como nexo entre los distintos sectores y controla las
distintas etapas de los procesos de producción.\\
\indent En la figura \ref{organigrama} se muestra el organigrama de la empresa que resume la estructura jerárquica descrita anteriormente.

\begin{figure}[h!]
\centering
{%
\setlength{\fboxsep}{0pt}%
\setlength{\fboxrule}{0.5pt}%
\fbox{\includegraphics[scale = 0.18]{Organigrama Imprenta Lux.jpg}}%
}%
\caption{Organigrama de Imprenta Lux S.A.}
\label{organigrama}
\end{figure}

\section{Descripción física de la empresa}
\label{Descripción física de la empresa}
\subsection{Divisiones espaciales y emplazamiento}

La empresa cuenta con dos establecimientos diferenciados y separados geográficamente. Ambos cuentan con un depósito cada uno donde se almacenan materiales a cuya gestión será destinado el sistema a desarrollar en el proyecto.
\begin{itemize}
\item El primer establecimiento, al cual se denominará <<principal>>, tiene fines de gestión así como es donde se realizan algunas tareas sencillas como diagramación y composición\renewcommand*{\thefootnote}{\fnsymbol{footnote}}\footnote[1]{La diagramación y composición se refiere a la adaptación de los archivos digitales para que las hojas del libro queden en el orden correcto una vez finalizado, y que sean compatibles con el software de las respectivas máquinas de copiado de planchas y de impresión.}\renewcommand*{\thefootnote}{\arabic{footnote}}, copia de planchas (preprensa), corte y terminación (su depósito correspondiente será denominado <<depósito principal>>). Es a su vez donde se encuentra el sector de administración y los talleres de mantenimiento.\\
El depósito en este caso contiene tintas, reveladores y otros insumos.\\
En este lugar hay máquinas de impresión, guillotinas y una alzadora de revistas.
\item El establecimiento restante, al cual denominaremos como <<secundario>>, es el relativo a la impresión de los trabajos y donde se realiza la mayor parte del proceso de producción (bajo el mismo lineamiento, su depósito correspondiente será denominado <<depósito secundario>>).\\
El depósito se compone de estanterías dedicadas puramente al papel que se usará en los proyectos próximos.\\
En este lugar hay máquinas de impresión, dobladoras, una alzadora de libros, guillotinas, una entapadora, prensas, cosedoras, una cortadora trilateral, una embolsadora (termosellado) y equipamiento para expedición (los libros son colocados en cajas, y una parte se deposita en el lugar temporalmente).
\end{itemize}

 \subsection{Descripción de los depósitos y composición de inventarios}

 \begin{itemize}
 \item Depósito principal: el material en él a su vez se encuentra diferenciado en dos subsectores separados físicamente en cuartos separados. En el primer cuarto se encuentra el material a ser tenido en cuenta para la gestión de inventario, mientras que el que se encuentra en el segundo cuarto solo tiene fines de picking, y sus ingresos y egresos no serán registrados en el sistema, por lo que este segundo subsector puede ser obviado a fines prácticos. El material que se encuentra en ambos cuartos tiene relación con cuestiones de mantenimiento o con impresión, como tintas, planchas y reveladores.
\item Depósito secundario: como ya se dijo, en este espacio se almacena papel de distintos formatos y gramajes, el cual tiene relación con las actividades directas de producción de la empresa.
 \end{itemize}

 \section{Descripción de los procesos de producción}
 \label{Descripción de los procesos de producción}
 Las tareas que componen los procesos productivos realizados en la empresa difieren en función de los bienes finales, en particular, en base a la producción de cuatro bienes genéricos cuyas secuencias de tareas están mayormente estandarizadas, a saber: libro, revista, almanaque, volante (y afiche).\\
 \indent El proceso de producción inicia cuando los archivos del cliente son recibidos por preprensa donde se realiza la diagramación, de donde se copian las planchas y se envían a las máquinas de impresión con sus respectivas órdenes (véase fig. \ref{orden taller}), así como a la parte de guillotinado para que el cortador sepa qué material debe cortar, en qué cantidad, tamaño y para qué máquina de impresión. En la medida en que se corta el material, el encargado de impresión carga el papel cortado para realizar su trabajo. Una vez culminada la etapa de impresión, dependiendo del tipo de producto, el material puede ser doblado (por ejemplo en la producción del libro). Luego se intercala, se abrocha o cose si es que correspondiera. En caso de producirse una revista, la tapa se intercala y abrocha junto con los pliegos interiores; posteriormente se hace el refilado (guillotinado final para alcanzar el tamaño del producto terminado). De ser un libro, debe prensarse y entaparse para luego pasar por una guillotina trilateral donde se le da el tamaño final. En último término se empaqueta el producto y se deposita en bolsas o cajas según las especificaciones del cliente.

\begin{figure}[h!]
\centering
{%
\setlength{\fboxsep}{0pt}%
\setlength{\fboxrule}{0.5pt}%
\fbox{\includegraphics[scale = 0.75]{Orden de taller.jpg}}%
}%
\caption{Ejemplo de orden de taller.}
\label{orden taller}
\end{figure}

\indent En las figuras \ref{diagrama actividad almanaques}, \ref{diagrama actividad libros}, \ref{diagrama actividad revistas} y \ref{diagrama actividad volantes} se muestran los diagramas de actividad de los procesos de producción de almanaques, libros, revistas y volantes/afiches respectivamente.

\begin{figure}[h!]
\centering
{%
\setlength{\fboxsep}{0pt}%
\setlength{\fboxrule}{0.5pt}%
\fbox{\includegraphics[scale = 0.19]{Diagrama de Actividad Produccion Almanaques.png}}%
}%
\caption{Diagrama de Actividad para Producción de Almanaques.}
\label{diagrama actividad almanaques}
\end{figure}

\begin{figure}[h!]
\centering
{%
\setlength{\fboxsep}{0pt}%
\setlength{\fboxrule}{0.5pt}%
\fbox{\includegraphics[scale = 0.13]{Diagrama de Actividad Produccion Libros.png}}%
}%
\caption{Diagrama de Actividad para Producción de Libros.}
\label{diagrama actividad libros}
\end{figure}

\begin{figure}[h!]
\centering
{%
\setlength{\fboxsep}{0pt}%
\setlength{\fboxrule}{0.5pt}%
\fbox{\includegraphics[scale = 0.15]{Diagrama de Actividad Produccion Revistas.png}}%
}%
\caption{Diagrama de Actividad para Producción de Revistas.}
\label{diagrama actividad revistas}
\end{figure}

\begin{figure}[h!]
\centering
{%
\setlength{\fboxsep}{0pt}%
\setlength{\fboxrule}{0.5pt}%
\fbox{\includegraphics[scale = 0.185]{Diagrama de Actividad Produccion Volantes.png}}%
}%
\caption{Diagrama de Actividad para Producción de Volantes/Afiches.}
\label{diagrama actividad volantes}
\end{figure}

\chapter{Descripción del problema} \label{Descripción del problema}

Podemos dividir el problema general que mediante el proyecto se plantea resolver en dos partes fundamentales diferenciadas, las cuales son: la gestión de los inventarios de la empresa y el control de la producción.

\section{Control de inventarios}

Actualmente al control de insumos\renewcommand*{\thefootnote}{\fnsymbol{footnote}}\footnote[1]{Solo teniendo en cuenta materias primas y materiales de mantenimiento (\hyperlink{MRO}{MRO}: Maintenance, repair, and operations), ya que tanto el inventario de trabajo en proceso (\hyperlink{WIP}{WIP}: Work In Process) como el de bienes finales no forman parte del problema.}\renewcommand*{\thefootnote}{\arabic{footnote}} y relevamiento de depósitos los realiza el encargado de mantenimiento, sin embargo, no se cuenta con un registro centralizado de los mismos, por lo que es común que se experimenten retrasos en la producción, así como en la toma de decisiones ante el requerimiento de presupuestos y tiempos por parte del cliente, lo cual termina por afectar el trabajo del gerente general. Este último en varias ocasiones no puede estar seguro de las estimaciones de tiempos para los trabajos requeridos ya que para hacerlo debería tener un registro actualizado de los insumos con el que se cuenta en los depósitos, y aún de tenerlo, debería ser posible establecer si dichos insumos, aún disponibles físicamente, no se encuentran ya reservados para ser usados en otros trabajos y se encuentran a la espera de ser utilizados en producción.\\
\indent Además de esto, de no haber una revisión periódica de los inventarios (lo cual parece ser habitual), suelen presentarse incluso faltantes de elementos de producción, o no ser posible emitir una orden de compra a tiempo cuando se llegue al \hyperlink{Pp}{Pp} (punto de pedido) de un insumo, lo cual pueden llegar a traducirse en retrasos en los plazos pactados con el cliente para la entrega de determinados trabajos.\\
\indent El control periódico por parte del encargado de mantenimiento, parecería ser la única medida para recabar la información de disponibilidad actualmente debido a que el ingreso de insumos está a cargo del encargado de reposición, quien, como se mencionó, no es quien se encarga del relevamiento de los depósitos.
\\\\\\\\\\

\section{Control de la producción}

Debido a la separación geográfica de los talleres pertenecientes a la empresa, el encargado de producción debe estar constantemente desplazándose o en contacto permanente con los empleados de producción con el fin de diagramar los trabajos planeados. Del mismo modo, la falta de centralización de la información provoca que, de haber un inconveniente con una máquina de modo que deba pasar a control del equipo de mantenimiento, esto o los retrasos no sean informados sino hasta que fueron resueltos, lo cual termina por dificultar del mismo modo la diagramación de trabajos por parte de los empleados de jerarquías superiores (el mismo encargado de producción o el gerente general). Claramente, estas dificultades se incrementan a medida que aumenta el número de trabajos realizados de manera simultánea.

\chapter{Enfoque de resolución del problema}

Habiendo descrito el problema (véase capítulo \ref{Descripción del problema}), es posible tratar el enfoque de resolución del mismo.\\
\indent Se describirán las características del proyecto así como la metodología de trabajo propuesta para llevarlo a cabo.

\section{Características generales del proyecto}
\label{Caracteristicas_generales_del_proyecto}
Se tiene una serie de características en las que es posible basarse para la elección de un modelo de procesos adecuado para el software a desarrollar. Entre ellas, se pueden mencionar:
\begin{itemize}
	\item Baja cantidad de usuarios y stakeholders en general, debido a que se trata de una empresa de un tamaño relativamente pequeño (alrededor de 30 empleados).
	\item El alcance es reducido en relación a otros sistemas empresariales (integra solamente los procesos de control de la producción e inventarios de la empresa).
	\item Requisitos no absolutamente definidos desde el inicio, ya que la empresa no ha trabajado con un sistema de alcance similar al pretendido para el presente sistema.
	\item Equipo de desarrollo pequeño (dos integrantes).
	\item El cliente muestra predisposición y, al menos inicialmente, buena disponibilidad para la comunicación.
\end{itemize}

\section{Modelo de desarrollo del software utilizado}
\label{Modelo de desarrollo del software utilizado}
En base a las características del proyecto mencionadas (principalmente la probabilidad alta de cambios y la posibilidad de comunicación constante con el cliente), se decidió implementar un modelo de desarrollo de software basado en principios generales de las metodologías ágiles conocidas con el fin de obtener los beneficios que de la utilización de estas se desprenden. En relación a lo anterior, se considera que en este caso no es posible la adopción plena de una metodología ágil predefinida a causa del bajo número con el que cuenta el equipo de desarrollo; esto afecta directamente características con las que cuentan metodologías como Scrum, la cual define distintos roles, o por ejemplo la práctica de \textit{Pair Programming} originaria de XP, la cual de aplicarse aumentaría considerablemente el tiempo necesario de trabajo para finalizar las tareas previstas.\\
\indent Las premisas, por lo tanto, en las que se decidió basarse (determinadas directamente como base de las metodologías ágiles o que vienen como consecuencias de estas) para la realización del presente proyecto son las mencionadas a continuación\footnote{Estos principios se basan en las descripciones presentadas en:\\ \indent <<\textit{Manifesto for Agile Software Development}>> de Agile Manifesto, \url{http://agilemanifesto.org/}. Consultado el 11/04/2024;\\ \indent <<\textit{The 12 Principles behind the Agile Manifesto}>> de Agile Alliance, \url{https://www.agilealliance.org/agile101/12-principles-behind-the-agile-manifesto/}. Consultado el 11/04/2024;\\
	\indent Kumar, Gaurav - Bhatia, Pradeep K. \textit{<<Impact of Agile Methodology on Software Development Process>>}. International Journal of Computer Technology and Electronics Engineering (IJCTEE)
	Volume 2, Issue 4, Agosto 2012. Págs. 46-47.}:
\begin{itemize}
	\item \underline{Comunicación constante con el cliente:} para una mejor comprensión del problema, así como en pos de no incurrir en asunciones erróneas que puedan traducirse en futuros cambios que produzcan pérdidas de tiempo, el que podría utilizarse en aumentar la calidad final del producto, así como en adiciones involuntarias de fallos técnicos (ya sea de funciones particulares o, peor aún, estructurales del sistema) por modificaciones del código.
	\item \underline{Im\smash{p}lementación de \smash{p}rácticas de forma iterativa:} la adaptabilidad del software a modificaciones imprevistas por parte del cliente permite que estas últimas puedan ser realizadas en un plazo corto (de entre algunas semanas a algunos meses), sin extender las consecuencias de una implementación que no se condiga con lo esperado. Esto es relativo tanto a requerimientos adicionales como a modificaciones de los requerimientos del software producido o a producir. Al mismo tiempo, debido a que no se espera hasta el final de la implementación para validar el producto, de acuerdo a las modificaciones propuestas en cada etapa/iteración, se adquiere una mejor perspectiva en cuanto a las expectativas del proyecto, así como una medición constante del progreso y aceptación de los desarrollos parciales.
	\item \underline{Reducida documentación con ma\smash{y}or atención a la calidad del software:} la reducida documentación motiva a los miembros del equipo de trabajo a enfocarse en la calidad del producto, reduciendo al mismo tiempo las cargas producidas por excesivas actividades de documentación. De cualquier modo, es importante destacar que, a pesar de esto, no se está sugiriendo una ausencia total de documentación, sino de evitar realizar excesiva documentación para dar prioridad a la calidad del software implementado.
\end{itemize}

\section{Metodología de desarrollo y coordinación del equipo}
\label{Metodología de desarrollo y coordinación del equipo}

Se considera que, a pesar de haberse optado por la adopción de un método de desarrollo basado en los principios ágiles, no existe un modelo predefinido al cual sea posible adecuarse completamente  (véase capítulo \ref{Modelo de desarrollo del software utilizado}). Por esto, se describe a continuación la metodología de trabajo particular que se implementa en este caso:\\
\indent Inicialmente se opta por hacer una primera aproximación de la solución mediante reuniones con el cliente donde se definen a grandes rasgos los requerimientos del sistema. Tales requerimientos son plasmados de manera resumida en forma de historias de usuario (\hyperlink{US}{US}, las cuales, para este caso, se muestran en el \hyperlink{apendice_a}{Apéndice A} junto a sus respectivos refinamientos).\\
\indent Teniéndose las \hyperlink{US}{US} definidas (pueden no ser la totalidad de las mismas a lo largo del proyecto), los miembros del equipo de desarrollo las refinan de modo de detallarlas en forma de tareas o \textit{<<tasks>>}. A su vez, en la medida en que surgen dudas respecto a aspectos no tratados en las reuniones con el cliente, se genera una lista de preguntas para ser planteadas a este último.\\
\indent La estimación de esfuerzo (tiempo de trabajo) para las \hyperlink{US}{US} definidas se hace mediante la asignación de \textit{<<points>>} a las mismas, las cuales son relativas entre unas y otras, de modo que mayor cantidad de points serán asignados a la \hyperlink{US}{US} que se crea pueda tener más dificultad de desarrollo. Las \hyperlink{US}{US} más sencillas se pueden tomar de referencia, y de acuerdo a sus estimaciones y de forma proporcional a las de mayor complejidad, ser utilizadas para estimar el esfuerzo de estas últimas.\\
\indent A partir de las asignaciones relativas mencionadas, se define la \textit{<<velocidad>>} de trabajo medida en \textit{tiempo/point}. De este modo, habiendo asignado los points a las \hyperlink{US}{US} definidas y establecida la velocidad, es posible estimar el tiempo real de trabajo para cada \hyperlink{US}{US} de manera individual.\\
\indent Se debe seleccionar un tamaño de iteración (\textit{Iteration Size}), que no es más que el tiempo que durará cada iteración o \textit{<<Sprint>>}.\\
\indent Las \hyperlink{US}{US} se identifican en base a la importancia de negocio así como a la prioridad de desarrollo con el fin de poder establecer un orden de implementación acorde (aunque finalmente, como se menciona en la subsección \ref{requerimientos_del_sistema_e_historias_de_usuario}, ya que el proyecto consta de un solo release, se optó por una prioridad basada únicamente en la interdependencia de desarrollo de las \hyperlink{US}{US}). En relación a esto se seleccionan las \hyperlink{US}{US} a implementar por iteración, teniéndose siempre presente que no será posible seleccionar más \hyperlink{US}{US} que las que puedan implementarse de acuerdo a la velocidad estimada.\\
\indent Ya que lo que se intenta es que las estimaciones de los points asignados a las \hyperlink{US}{US} guarden una correcta proporción en cuanto a sus complejidades relativas, así como no introducir más errores humanos al añadir otra posible fuente de error debido a otra estimación basada en la subjetividad del equipo de desarrollo, la velocidad de desarrollo del equipo (cantidad de points finalizados/iteración) será tomada en una iteración dada en función de lo evidenciado en la precedente. Es decir, la velocidad del equipo no puede superar a la obtenida en la práctica en la iteración anterior. Esto, claro está, no puede ser aplicado a la iteración I, para lo que no queda alternativa a la de asignar una velocidad inicial basada en la experiencia del equipo de desarrollo.\\
\indent Luego de planificada la iteración, los miembros del equipo de desarrollo se dividen las \hyperlink{US}{US} (o incluso las tasks) que cada uno desea implementar.\\
\indent El proceso de estimación de esfuerzo, cálculo de la nueva velocidad de trabajo y consiguiente cálculo de la estimación de tiempo de trabajo, así como la planificación de la iteración y ejecución de lo planificado, deberá repetirse hasta haberse completado la totalidad de \hyperlink{US}{US} que se crean necesarias para cumplir con los objetivos del proyecto\renewcommand*{\thefootnote}{\fnsymbol{footnote}}\footnote[1]{Si bien la estimación dada de manera inicial (relativa a lo expuesto en el \hyperlink{apendice_e}{Apéndice E}) había sido de cuatro iteraciones, se verá en el capítulo \ref{Ejecución del proyecto} que finalmente fueron necesarias cinco.}\renewcommand*{\thefootnote}{\arabic{footnote}} (véase \hyperlink{apendice_c}{Apéndice C}).

\section{Descripción general de la solución planteada}
De acuerdo al problema planteado en el capítulo \ref{Descripción del problema}, se pretende diseñar e implementar un sistema de información destinado al control del inventario así como de la producción de la empresa. Debido a la necesidad de portabilidad requerida para los usuarios de los distintos puestos de trabajo (donde no se cuenta en la mayoría de los casos con PCs), el sistema debe poderse acceder tanto mediante dispositivos móviles como mediante equipos de escritorio.\\
\indent Imprenta Lux S.A. cuenta actualmente con dos depósitos físicos (tal como se ha descrito en la sección \ref{Descripción física de la empresa}), tal que se deberán registrar los insumos que en estos se encuentran, así como los ingresos y extracciones hacia y desde los mismos, incluso teniendo en cuenta la transferencia de materiales entre uno y otro, de modo de poder tener esta información disponible para la planificación de nuevos proyectos. De cualquier modo,  ya que la empresa no está limitada a posibles futuras ampliaciones físicas o nuevas disposiciones de gestión, el sistema no deberá estar restringido a los dos depósitos actuales, sino que se deberán poder agregar nuevos e incluso eliminar los existentes.\\
\indent  Con el fin de reducir la frecuencia de verificaciones relativas al estado de los inventarios, deberá ser posible establecer valores de alerta para los distintos insumos, por ejemplo, cuando se llegue al valor de su \hyperlink{Pp}{Pp}, de modo tal que los usuarios interesados sean notificados de haberse alcanzado la cantidad mínima tolerable.\\
\indent En el caso del control de la producción, el sistema debe contar con diagramaciones de proyectos relacionados con bienes genéricos producidos por la empresa (tema tratado en la sección \ref{Descripción de los procesos de producción}), es decir, proyectos cuyas tareas componentes sean genéricas y abarquen en su totalidad las actividades de la empresa. Por ejemplo, para un proyecto de producción de un libro se debe disponer de una diagramación del flujo del proceso cuyas tareas componentes sean genéricas para la producción de este bien y, al mismo tiempo esta debe ser suficientemente abarcativa de modo que dicha planificación sea aplicable a todos los proyectos de producción de libros. Del mismo modo en cada caso para los restantes productos fabricados por la empresa (como ha sido mencionado, además de libros: revistas, almanaques, volantes y afiches). En este aspecto, de ser necesarias planificaciones que no se correspondan con las diagramaciones existentes en el sistema, deberá ser posible generar nuevas.\\
\indent Es importante mencionar que, así como se debe llevar el recuento de los insumos en los depósitos, también deben <<bloquearse>> en el momento en que estén siendo usados para proyectos de producción y registrarse la devolución del excedente hacia los depósitos, así como debe darse de baja el material que ya fue usado en los proyectos terminados. Asimismo, será necesario diferenciar claramente los insumos propiedad de la empresa de aquellos pertenecientes a clientes, con el fin de evitar su uso indebido en otros proyectos.\\
\indent Con esto se espera que el acceso a la información concurrente disminuya la posibilidad de errores humanos relacionados a la gestión basada en prácticas de manejo de información habituales, como la toma de apuntes o la posesión de documentos digitalizados no compartidos por todos los interesados (caso inicial). A la vez que se espera resolver los problemas tanto del control de inventarios como de la producción de manera individual (referentes a cuestiones de disponibilidad de insumos y de seguimiento de las tareas de producción), se busca proporcionar una solución que integre de manera conjunta ambos procesos, de modo que sirva como soporte para la gestión de los procesos productivos de la empresa en general. Con esto se pretende reducir las demoras debidas a la utilización de información incorrecta, poco precisa o desactualizada (por ejemplo, por falta de insumos y la espera de la consiguiente reposición del stock para un proyecto particular), lo cual en general puede traducirse en retrasos en las entregas de pedidos. Esto, a su vez, debería permitir una mejor toma de decisiones en cuanto a la estimación de los tiempos de los proyectos a diagramarse, así como reducir la necesidad de constante comunicación directa entre los diversos sectores, al contarse con una base de información centralizada, de la que puedan disponer los distintos usuarios, en mayor o menor medida, en base a su incidencia en las distintas tareas.

\chapter{Ejecución del proyecto}
\label{Ejecución del proyecto}

\section{Etapa de análisis inicial}
Para la definición del sistema a desarrollar se trabajó con los futuros usuarios del sistema con el objetivo inicial de establecer los requisitos del mismo. En particular, debido a que los integrantes de mayor jerarquía en la organización son los principales interesados en su uso, y quienes mayores certezas tienen respecto al sistema requerido, se mantuvo una comunicación periódica con los mismos, a saber: la gerente general, el encargado de mantenimiento/control de reposición de la empresa, y el encargado de producción.\\
\indent De lo anterior se obtuvieron las definiciones que se describen a continuación.

\subsection{Requerimientos del sistema e historias de usuario}
\label{requerimientos_del_sistema_e_historias_de_usuario}
Las \hyperlink{US}{US} definidas inicialmente se listan en la tabla \ref{tabla_dependencias_iniciales_us} a modo de resumen con sus respectivas dependencias de desarrollo (para un mayor detalle en cuanto a los refinamientos correspondientes de las \hyperlink{US}{US}, véase el \hyperlink{apendice_a}{Apéndice A}). En este caso particular, debido a que el proyecto consta de un solo release, la prioridad en cuanto al valor de negocio puede ser obviada. Con esta metodología de priorización de \hyperlink{US}{US} se intentó facilitar las tareas de integración al reducir la utilización de componentes de tipo Stub.

\begin{table*}[h!]
\centering
\begin{tabular}{ |p{0.5cm}|p{9cm}|c|  }
\hline
\verb|#|& \textbf{Nombre de \hyperlink{US}{US}}& \textbf{Dependencia} \\
\hline
\textbf{1} & Inicio de sesión & - \\
\hline
\textbf{2} & Administrar usuarios & 1 \\
\hline
\textbf{3} & Administrar insumo & 4 \\
\hline
\textbf{4} & Administrar tipo de insumo & 1 \\
\hline
\textbf{5} & Modificar cantidad de insumo & 3 y 6 \\
\hline
\textbf{6} & Administrar depósitos & 7 \\
\hline
\textbf{7} & Administrar sucursales & 1 \\
\hline
\textbf{8} & Administrar proyecto & 10 \\
\hline
\textbf{9} & Agregar insumo a proyecto & 8 y 3 \\
\hline
\textbf{10} & Administrar tipos de proyectos & 1 \\
\hline
\textbf{11} & Editar tarea & 8 \\
\hline
\textbf{12} & Asignar tarea a empleado & 11\\
\hline
\textbf{13} & Agregar comentario a tarea & 12 \\
\hline
\textbf{14} & Modificar estado de tarea & 11 \\
\hline
\textbf{15} & Finalizar proyecto & 8 \\
\hline
\textbf{16} & Buscar proyecto & 8 \\
\hline
\textbf{17} & Visualizar tareas asignadas & 12 \\
\hline
\textbf{18} & Administrar clientes & 1 \\
\hline
\end{tabular}
\caption{Dependencias de desarrollo entre las \protect\hyperlink{US}{US} definidas.}
\label{tabla_dependencias_iniciales_us}
\end{table*}

\indent Lo anterior puede a su vez visualizarse en la figura \ref{diagrama_dependencias_US}, donde se muestra un diagrama con las dependencias señaladas en la tabla \ref{tabla_dependencias_iniciales_us}.

\begin{figure}[h]
\begin{center}
	\setlength{\fboxsep}{10pt} % Ajusta el padding
	\fbox{
	\begin{tikzpicture}[scale=1.1]
		\GraphInit[vstyle=Normal]
		\SetVertexNormal[Shape=circle,LineWidth = 1pt]
		\tikzset{EdgeStyle/.append style = {color = blue!70, line width=1pt}}
		\renewcommand*{\VertexLineWidth}{1pt}%vertex thickness
		\renewcommand*{\EdgeLineWidth}{1pt}% edge thickness
		\GraphInit[vstyle=Normal]
		
		\Vertex[LabelOut,Lpos=270,L=$us_1$,x=0,y=4]{R1}
		
		\Vertex[LabelOut,Lpos=270,L=$us_2$,x=2,y=8]{R2}
		\Vertex[LabelOut,Lpos=270,L=$us_7$,x=2,y=6]{R7}
		\Vertex[LabelOut,Lpos=270,L=$us_4$,x=2,y=4]{R4}
		\Vertex[LabelOut,Lpos=270,L=$us_{10}$,x=2,y=2]{R10}
		\Vertex[LabelOut,Lpos=270,L=$us_{18}$,x=2,y=0]{R18}
		
		\Vertex[LabelOut,Lpos=270,L=$us_6$,x=4,y=6]{R6}
		\Vertex[LabelOut,Lpos=270,L=$us_3$,x=4,y=4]{R3}
		\Vertex[LabelOut,Lpos=270,L=$us_8$,x=4,y=2]{R8}
		
		\Vertex[LabelOut,Lpos=270,L=$us_5$,x=6,y=5]{R5}
		\Vertex[LabelOut,Lpos=270,L=$us_9$,x=6,y=3]{R9}
		\Vertex[LabelOut,Lpos=270,L=$us_{16}$,x=6,y=2]{R16}
		\Vertex[LabelOut,Lpos=270,L=$us_{11}$,x=6,y=1]{R11}
		\Vertex[LabelOut,Lpos=270,L=$us_{15}$,x=6,y=0]{R15}
		
		\Vertex[LabelOut,Lpos=270,L=$us_{12}$,x=8,y=2]{R12}
		\Vertex[LabelOut,Lpos=270,L=$us_{14}$,x=8,y=0]{R14}
		
		\Vertex[LabelOut,Lpos=270,L=$us_{13}$,x=10,y=3]{R13}
		\Vertex[LabelOut,Lpos=270,L=$us_{17}$,x=10,y=1]{R17}
		
		%%%%%%%%%%%%%%%%%%%%%%%%%%%%%%
		
		\Edge (R1)(R2)
		\Edge (R1)(R4)
		\Edge (R1)(R7)
		\Edge (R1)(R10)
		\Edge (R1)(R18)
		
		\Edge (R7)(R6)
		\Edge (R4)(R3)
		\Edge (R10)(R8)
		
		\Edge (R6)(R5)
		\Edge (R3)(R5)
		\Edge (R3)(R9)
		\Edge (R8)(R9)
		\Edge (R8)(R16)
		\Edge (R8)(R11)
		\Edge (R8)(R15)
		
		\Edge (R11)(R12)
		\Edge (R11)(R14)
		
		\Edge (R12)(R13)
		\Edge (R12)(R17)
		
	\end{tikzpicture}
	}
	\caption{Diagrama de dependencias entre las \protect\hyperlink{US}{US} definidas.}
	\label{diagrama_dependencias_US}
\end{center}
\end{figure}


Cabe aclarar que se intentó expresar las historias de usuario basándose en el criterio \hyperlink{INVEST}{INVEST}\footnote{Bajo este criterio, buenas \hyperlink{US}{US} tienen las características: Independiente, Negociable, Valuable, Estimable, Pequeña. El concepto fue tomado de: Pokharel, Prabhat - Vaidya, Pramesh. \textit{<<A Study of User Story in Practice>>}, Octubre 2020. Págs. 1-2.}, así es que, con dicho fin se dieron por aceptados aspectos como por ejemplo los perfiles definidos (los cuales se describen en la subsección \ref{descripcion_usuarios_perfiles}) o el carácter multiplataforma del sistema. De este modo se evita incurrir al menos en el error de definir en una \hyperlink{US}{US} el alcance de los privilegios del superusuario o la mención de que lo válido en una plataforma lo es en la restante, lo que haría que esas \hyperlink{US}{US} fuesen dependientes de las demás, y a su vez implicando la dificultosa estimación de la cantidad de \textit{<<points>>} necesarios para ser completadas.\\
\indent Asimismo, se intentó seguir con el criterio de escritura <<usuario, meta, valor>>\footnote{\textit{<<user, goal, value>>}: cada \hyperlink{US}{US} debe especificar un rol, un objetivo y un valor; El valor solo se consigue cuando el objetivo es alcanzado. Esta estructura estándar de \hyperlink{US}{US} fue tomada de: Pokharel, Prabhat - Vaidya, Pramesh. \textit{<<A Study of User Story in Practice>>}, Octubre 2020. Pág. 1.} en un lenguaje natural semiformal\footnote{Yanche Ari Kustiawan, Tek Yong Lim. \textit{<<User Stories in Requirements Elicitation: A Systematic Literature Review>>}, Agosto 2023. Pág. 1.}, siempre que no resulte redundante, para homogeneizar la estructura de las \hyperlink{US}{US}.


\subsection{Estimación de tiempos y riesgos del proyecto}
En la sección \ref{Metodología de desarrollo y coordinación del equipo} se ha mencionado que los tiempos estimados provienen de asignaciones de esfuerzos relativos entre distintas \hyperlink{US}{US}, medidas en \textit{points}, así como de la definición de una velocidad de trabajo, medida en \textit{tiempo/point}, la cual irá variando de acuerdo a los tiempos empíricos ajustados a partir de la iteración previa a una iteración dada (en el caso de la primera iteración, este valor debe ser estimado por el equipo de desarrollo).\\
\indent El \textit{Plan de gestión de riesgos} se muestra en el \hyperlink{apendice_d}{Apéndice D}.

\subsection{Descripción general de usuario y perfiles}
\label{descripcion_usuarios_perfiles}

\begin{itemize}
\item \textbf{Gerente general:} planifica las tareas a realizar así como evalúa fechas estimadas de finalización de los trabajos. Deberá contar con permisos de superusuario.
\item \textbf{Encargado de producción:} realiza el control de los procesos de producción. Debe contar con autorización sobre los proyectos y tareas de producción registrados.
\item \textbf{Encargado de mantenimiento:} está encargado del control de stock en los depósitos. Debe tener autorización sobre los valores de inventarios (reducción, aumento, seteo, establecimiento de parámetros como unidades de medida, registro de nuevos insumos, etc.)
\item \textbf{Encargado de reposición:} registra la entrada de material e instrumentos en los depósitos así como su salida, estas se dan en contexto de adquisición así como de traslado para su uso dentro de la misma sucursal o entre sucursales distintas. Se prevé que solo tenga autorizaciones de modificación de valores en las cantidades almacenadas de los insumos registrados en los inventarios.
\item \textbf{Empleado de producción:}
responsable de ejecutar tareas asignadas dentro de los proyectos de la empresa. Puede registrar información sobre el estado de dichas tareas para el seguimiento y control del proceso productivo.
\end{itemize}

\subsection{Uso de plataformas del sistema} Por cuestiones de comodidad, se solicitó que al menos el perfil de gerente general pueda acceder a través de PC al sistema, mientras que el resto de los trabajadores con perfil en el sistema tengan la posibilidad de ingresar de manera móvil de modo de poder registrar sus acciones sin necesidad de recurrir al uso de una PC como en el primer caso.\\
\indent Sin embargo, debido a la posibilidad de un acceso distinto al mencionado, a pesar de tenerse actualmente perfiles asociados con formas tentativas de uso de las plataformas del sistema, los perfiles no deben tener una restricción en cuanto al método de acceso del usuario. Es decir, el usuario independientemente del perfil debe poder ingresar al sistema tanto mediante una PC como mediante el uso de un dispositivo móvil.

\subsection{Arquitectura del sistema}
\indent Dado el tamaño y las necesidades operativas de la empresa, se optó por una arquitectura de tipo 2-tier (cliente-servidor), en la cual tanto la aplicación de escritorio como la aplicación móvil contienen la lógica de presentación y de negocio, y acceden directamente a la base de datos ubicada en un servidor remoto contratado. En la figura \ref{diagrama_cliente_servidor} se muestra un diagrama de la arquitectura elegida.

\begin{figure}[H]
	\centering
	{%
		\setlength{\fboxsep}{0pt}%
		\setlength{\fboxrule}{0.5pt}%
		\fbox{\includegraphics[scale = 0.2]{Cliente_servidor.png}}%
	}%
	\caption{Diagrama de la arquitectura cliente-servidor.}
	\label{diagrama_cliente_servidor}
\end{figure}

La decisión de elegir este tipo de arquitectura se fundamenta en que se prevé que el sistema sea utilizado de forma simultánea por un número reducido de usuarios (no más de cinco en promedio), lo cual hace innecesaria la complejidad adicional de una capa intermedia.\\
\indent La arquitectura 2-tier permite un desarrollo más ágil y una estructura más simple de mantenimiento, resultando adecuada para organizaciones de tamaño pequeño, como la presente, donde la escalabilidad horizontal (por ejemplo, distribuir la carga en varios servidores) no es una exigencia inmediata.

\subsection{Tecnologías seleccionadas y versiones}
En el \hyperlink{apendice_f}{Apéndice F} se describen las tecnologías empleadas en el desarrollo del proyecto, junto con sus versiones correspondientes.

\subsection{Estructura general del sistema (diagrama de clases)}
En el \hyperlink{apendice_g}{Apéndice G} (figura \ref{diagrama_de_clases}) se muestra el diagrama de clases que define la estructura del sistema así como la descripción de cada clase para mayor claridad.

\subsection{Estructura de la base de datos}
Considerando la experiencia previa del equipo en el uso de bases de datos relacionales, se optó por su implementación en este caso. En el \hyperlink{apendice_g}{Apéndice G} (figura \ref{diagrama_ER}) se muestra el diagrama de entidad-relación (notación de Chen) que define la estructura de la base de datos del sistema. A su vez, en base a esto se muestran las definiciones de las tablas resultantes.


\section{Preparación inicial del proyecto}
\indent La semana inicial fue dedicada a la preparación del proyecto, a saber: instalación de las herramientas necesarias para el trabajo, coordinación del equipo, últimos detalles acordados con el cliente, etc.\\
\indent Algo a remarcar, y que influyó en el flujo esperado de trabajo del equipo, fue un incidente sufrido en el servidor de la base de datos, ya que por simplicidad en un inicio se pensó en trabajar temporalmente sin la implementación de SSL sobre las conexiones, teniéndose en cuenta que el servidor fue destinado puramente a desarrollo. Esto dio la posibilidad de un ataque por el cual se debió reinstalar el DBMS PostgreSQL\footnote{Web oficial de PostgreSQL, \url{https://www.postgresql.org/}. Consultado el 28/07/2024.}, esta vez implementando SSL. Por lo tanto el inicio del desarrollo debió ser retrasado hasta el día viernes 12/07/2024 (en tanto que en la planificación inicial se había indicado que la fecha de inicio sería el día lunes 08/07/2024).

\section{Iteración I}
\label{descripcion_iteracion_I}
\subsection{Planificación}
Las \hyperlink{US}{US} definidas se listan en la tabla \ref{tabla_dependencias_us} a modo de resumen con sus respectivas dependencias de desarrollo (para un mayor detalle en cuanto a los refinamientos correspondientes de las \hyperlink{US}{US}, véase el \hyperlink{apendice_a_I}{Apéndice A - Iteración I}). En este caso particular, debido a que el proyecto consta de un solo release, la prioridad en cuanto al valor de negocio puede ser obviada. Con esta metodología de priorización de \hyperlink{US}{US} se intentó facilitar las tareas de integración al reducir la utilización de componentes de tipo stub. Las \hyperlink{US}{US} que se muestran señaladas con color gris son las planificadas para ser realizadas en la primera iteración.

\begin{table*}[h!]
	\centering
	\begin{tabular}{ |p{0.5cm}|p{9cm}|c|  }
		\hline
		\verb|#|& \textbf{Nombre de \hyperlink{US}{US}}& \textbf{Dependencia} \\
		\hline
		\textbf{1} & \cellcolor{marca_US_realizada}Inicio de sesión & - \\
		\hline
		\textbf{2} & \cellcolor{marca_US_realizada}Administrar usuarios & 1 \\
		\hline
		\textbf{3} & Administrar insumo & 4 \\
		\hline
		\textbf{4} & Administrar tipo de insumo & 1 \\
		\hline
		\textbf{5} & Modificar cantidad de insumo & 3 y 6 \\
		\hline
		\textbf{6} & \cellcolor{marca_US_realizada}Administrar depósitos & 7 \\
		\hline
		\textbf{7} & \cellcolor{marca_US_realizada}Administrar sucursales & 1 \\
		\hline
		\textbf{8} & Administrar proyecto & 10 \\
		\hline
		\textbf{9} & Agregar insumo a proyecto & 8 y 3 \\
		\hline
		\textbf{10} & Administrar tipos de proyectos & 1 \\
		\hline
		\textbf{11} & Editar tarea & 8 \\
		\hline
		\textbf{12} & Asignar tarea a empleado & 11\\
		\hline
		\textbf{13} & Agregar comentario a tarea & 12 \\
		\hline
		\textbf{14} & Modificar estado de tarea & 11 \\
		\hline
		\textbf{15} & Finalizar proyecto & 8 \\
		\hline
		\textbf{16} & Buscar proyecto & 8 \\
		\hline
		\textbf{17} & Visualizar tareas asignadas & 12 \\
		\hline
		\textbf{18} & \cellcolor{marca_US_realizada}Administrar clientes & 1 \\
		\hline
	\end{tabular}
	\caption{Dependencias de desarrollo entre las \protect\hyperlink{US}{US} definidas.}
	\label{tabla_dependencias_us}
\end{table*}

\subsection{Comparación entre estimaciones y tiempos de ejecución}
En base a las dependencias mostradas en la tabla \ref{tabla_dependencias_us} se definieron distintas responsabilidades para los miembros del equipo de desarrollo, las cuales son mostradas en la tabla \ref{tabla_desarrollo_iter_1}.\\
\indent El valor seleccionado de velocidad inicial fue de 8 horas de trabajo individual por \hyperlink{SP}{SP} con base en estimaciones de tiempos a criterio de los desarrolladores para las \hyperlink{US}{US} en apariencia más sencillas. Si se tiene en cuenta que cada integrante del equipo de desarrollo realizará 20 horas de trabajo semanales, y que para cada iteración se ha convenido en emplear un tiempo de tres semanas, se tienen 7,5 \hyperlink{SP}{SP} asignables a cada miembro por iteración.

\begin{table*}[h!]
	\centering
	\begin{tabular}{ |p{0.5cm}|p{6cm}|c|c|c|c| }
		\hline
		\verb|#|& \textbf{Nombre de \hyperlink{US}{US}}& \textbf{Encargado} & \textbf{Estimado [\hyperlink{SP}{SP}]} & \textbf{Real [\hyperlink{SP}{SP}]} \\
		\hline
		\textbf{1} & Inicio de sesión & \cellcolor{marca_US_emir}Emir S. & 3 & 2 \\
		\hline
		\textbf{2} & Administrar usuarios & \cellcolor{marca_US_emir}Emir S. & 2 & 1,75 \\
		\hline
		\textbf{6} & Administrar depósitos & \cellcolor{marca_US_luciano}Luciano S. & 3 & 2,25\\
		\hline
		\textbf{7} & Administrar sucursales & \cellcolor{marca_US_luciano}Luciano S. & 3 & 1,75 \\
		\hline
		\textbf{18} & Administrar clientes & \cellcolor{marca_US_emir}Emir S. & 2 & 2 \\
		\hline
	\end{tabular}
	\caption{División de tareas de las \protect\hyperlink{US}{US} desarrolladas en la iteración I.}
	\label{tabla_desarrollo_iter_1}
\end{table*}

En base a los datos mostrados en la tabla \ref{tabla_desarrollo_iter_1}, con el valor de velocidad inicial de 8 horas de trabajo individual por \hyperlink{SP}{SP}, se infiere que en esta primera iteración se han realizado 13 \hyperlink{SP}{SP} (estimación inicial) en 78 horas de trabajo individual total real (9,75 \hyperlink{SP}{SP} reales). Se recalcula por lo tanto la velocidad de desarrollo como el cociente entre la cantidad de horas individuales totales trabajadas y la cantidad de \hyperlink{SP}{SP} estimadas, obteniéndose el nuevo valor de 6 horas de trabajo individual por \hyperlink{SP}{SP}. Tal valor es utilizado en la iteración II para realizar la planificación de trabajo correspondiente.\\
\indent En la tabla \ref{tabla_dif_horas_estim_iter_1} se muestran los errores de estimación por \hyperlink{US}{US} desarrollada si es que la velocidad inicial de desarrollo hubiese sido la calculada recientemente (6 horas).

\begin{table*}[h!]
	\centering
	\captionsetup{justification=centering,margin=1.5cm}
	\begin{tabular}{ |p{0.5cm}|l|c|c|c|c| }
		\hline
		\verb|#|& \textbf{Nombre de \hyperlink{US}{US}}& \textbf{Estimado[hs]} & \textbf{Real [hs]} & \textbf{Desviación [hs]} \\
		\hline
		\textbf{1} & Inicio de sesión & 18 & 16 & \cellcolor{diferencia_estimacion_positiva}2 \\
		\hline
		\textbf{2} & Administrar usuarios & 12 & 14 & \cellcolor{diferencia_estimacion_negativa}-2 \\
		\hline
		\textbf{6} & Administrar depósitos & 18 & 18 & \cellcolor{diferencia_estimacion_positiva}0 \\
		\hline
		\textbf{7} & Administrar sucursales & 18 & 14 & \cellcolor{diferencia_estimacion_positiva}4 \\
		\hline
		\textbf{18} & Administrar clientes & 12 & 16 & \cellcolor{diferencia_estimacion_negativa}-4 \\
		\hline
	\end{tabular}
	\caption{Diferencias entre horas estimadas y reales de desarrollo para las \protect\hyperlink{US}{US} pertenecientes a la iteración I con el nuevo valor calculado de velocidad.}
	\label{tabla_dif_horas_estim_iter_1}
\end{table*}

\indent Asimismo podemos notar que hay algunas diferencias a tener en cuenta entre lo estimado y lo real. Parte de estas discrepancias viene de no haber tenido en cuenta la similitud entre \hyperlink{US}{US}, como es el caso de \textit{Administrar depósitos} y \textit{Administrar sucursales}, esto es uno de los aspectos a tener en cuenta en las iteraciones posteriores. Por otro lado, es de esperar que la primera iteración sea la más afectada por la falta de habilidad de los desarrolladores en la utilización de las herramientas empleadas en el proyecto, lo que hace suponer que el tiempo de desarrollo de las \hyperlink{US}{US} de las posteriores iteraciones debería tender a disminuir.

\section{Iteración II}
\label{descripcion_iteracion_II}
\subsection{Planificación}
Las \hyperlink{US}{US} resultantes de lo descrito para el caso de la iteración I se muestran en la tabla \ref{tabla_dependencias_us_it2} (además de algunas más que se creyó conveniente agregar) junto a sus dependencias de desarrollo (para un mayor detalle en cuanto a los refinamientos correspondientes de las \hyperlink{US}{US}, véase el \hyperlink{apendice_a_II}{Apéndice A - Iteración II}). Las \hyperlink{US}{US} que se muestran señaladas con color verde son las ya realizadas en la iteración precedente, y en color gris se indican las planificadas para ser realizadas en la presente iteración.

\begin{table*}[h!]
	\centering
	\begin{tabular}{ |p{0.5cm}|p{9cm}|c|  }
		\hline
		\verb|#|& \textbf{Nombre de \hyperlink{US}{US}}& \textbf{Dependencia} \\
		\hline
		\textbf{1} & \cellcolor{marca_US_realizada_anterior}Inicio de sesión & - \\
		\hline
		\textbf{2} & \cellcolor{marca_US_realizada_anterior}Administrar usuarios & 1 \\
		\hline
		\textbf{3} & \cellcolor{marca_US_realizada}Administrar insumo & 4 \\
		\hline
		\textbf{4} & \cellcolor{marca_US_realizada}Administrar tipo de insumo & 1 \\
		\hline
		\textbf{5} & \cellcolor{marca_US_realizada}Modificar cantidad de insumo & 3 y 6 \\
		\hline
		\textbf{6} & \cellcolor{marca_US_realizada_anterior}Administrar depósitos & 7 \\
		\hline
		\textbf{7} & \cellcolor{marca_US_realizada_anterior}Administrar sucursales & 1 \\
		\hline
		\textbf{8} & Administrar proyecto & 10 \\
		\hline
		\textbf{9} & Agregar insumo a proyecto & 8 y 3 \\
		\hline
		\textbf{10} & Administrar tipos de proyectos & 23 \\
		\hline
		\textbf{11} & Editar tarea & 8 \\
		\hline
		\textbf{12} & Asignar tarea a empleado & 11\\
		\hline
		\textbf{13} & Agregar comentario a tarea & 12 \\
		\hline
		\textbf{14} & Modificar estado de tarea & 11 \\
		\hline
		\textbf{15} & Finalizar proyecto & 8 \\
		\hline
		\textbf{16} & Buscar proyecto & 8 \\
		\hline
		\textbf{17} & Visualizar tareas asignadas & 12 \\
		\hline
		\textbf{18} & \cellcolor{marca_US_realizada_anterior}Administrar clientes & 1 \\
		\hline
		\textbf{19} & Eliminar usuario & 2 y 12 \\
		\hline
		\textbf{20} & Eliminar depósito & 6 y 3 \\
		\hline
		\textbf{21} & Eliminar sucursal & 20 \\
		\hline
		\textbf{22} & Eliminar cliente & 8 \\
		\hline
		\textbf{23} & \cellcolor{marca_US_realizada}Administrar tipo de tarea & 1 
		\\
		\hline
		\textbf{24} & \cellcolor{marca_US_realizada}Visualizar novedades & 1 
		\\
		\hline
		\textbf{25} & Visualizar proyecto & 8 
		\\
		\hline
	\end{tabular}
	\caption{Dependencias de desarrollo actualizadas entre las \protect\hyperlink{US}{US} definidas.}
	\label{tabla_dependencias_us_it2}
\end{table*}

\subsection{Comparación entre estimaciones y tiempos de ejecución}
En base a las dependencias mostradas en la tabla \ref{tabla_dependencias_us_it2} se definieron distintas responsabilidades para los miembros del equipo de desarrollo, las cuales son mostradas en la tabla \ref{tabla_desarrollo_iter_2}. Cabe mencionar que para esta iteración se decidió agregar tres \hyperlink{US}{US} (denotados con los números 23, 24 y 25), las cuales provienen de necesidades del cliente no tenidas en cuenta en un principio (23 y 25) así como de \hyperlink{US}{US} previamente definidas y que se creyó conveniente dividir para seguir de manera adecuada con el criterio \hyperlink{INVEST}{INVEST} (24).\\
\indent El valor ajustado de velocidad al finalizar la iteración I había sido de 6 horas de trabajo individual por \hyperlink{SP}{SP}. Si se tiene en cuenta que cada integrante del equipo de desarrollo realiza 20 horas de trabajo semanales, y que para cada iteración se ha convenido en emplear un tiempo de tres semanas, se tienen 10 \hyperlink{SP}{SP} asignables a cada miembro para esta iteración.

\begin{table*}[h!]
	\centering
	\begin{tabular}{ |p{0.5cm}|p{6cm}|c|c|c|c| }
		\hline
		\verb|#|& \textbf{Nombre de \hyperlink{US}{US}}& \textbf{Encargado} & \textbf{Estimado [\hyperlink{SP}{SP}]} & \textbf{Real [\hyperlink{SP}{SP}]} \\
		\hline
		\textbf{3} & Administrar insumo & \cellcolor{marca_US_luciano}Luciano S. & 3 & 2,75 \\
		\hline
		\textbf{4} & Administrar tipo de insumo & \cellcolor{marca_US_luciano}Luciano S. & 3 & 2,5 \\
		\hline
		\textbf{5} & Modificar cantidad de insumo & \cellcolor{marca_US_luciano}Luciano S. & 3 & 4,67\\
		\hline
		\textbf{23} & Administrar tipo de tarea & \cellcolor{marca_US_emir}Emir S. & 3 & 4,17 \\
		\hline
		\textbf{24} & Visualizar novedades & \cellcolor{marca_US_emir}Emir S. & 3 & 4,5 \\
		\hline
	\end{tabular}
	\caption{División de tareas de las \protect\hyperlink{US}{US} desarrolladas en la iteración II.}
	\label{tabla_desarrollo_iter_2}
\end{table*}

De acuerdo con los datos mostrados en la tabla \ref{tabla_desarrollo_iter_2}, con el valor de velocidad inicial de 6 horas de trabajo individual por \hyperlink{SP}{SP} (obtenida con base en la práctica de acuerdo a los datos ya abordados en la iteración I), se infiere que en esta segunda iteración se han realizado 15 \hyperlink{SP}{SP} (estimación inicial\renewcommand*{\thefootnote}{\fnsymbol{footnote}}\footnote[1]{Teniéndose en cuenta que, a pesar de poderse planificar más \hyperlink{US}{US} dentro de la iteración porque así lo permitiría el tiempo restante de la misma, en la práctica se vio que en general se tendió a subestimar el esfuerzo en este caso. De acuerdo a esto es que solo se pudo cumplir con las 15 \hyperlink{SP}{SP} teóricas informadas.}\renewcommand*{\thefootnote}{\arabic{footnote}}) en 111,54 horas de trabajo individual total real (18,59 \hyperlink{SP}{SP} reales). Se recalcula por lo tanto la velocidad de desarrollo como el cociente entre la cantidad de horas individuales totales trabajadas y la cantidad de \hyperlink{SP}{SP} estimadas, obteniéndose el nuevo valor de \begin{math}\lceil 7,44\rceil\end{math} horas (8 horas) de trabajo individual por \hyperlink{SP}{SP}. Este valor se condice con el propuesto al inicio de la iteración I aún si se lo indicó como una sobreestimación inicial, teniéndose en cuenta en ese punto principalmente la falta de experiencia de los miembros del equipo de desarrollo tanto en el uso de las herramientas empleadas en el proyecto como en la realización de estimaciones. De cualquier modo, aún si el valor a utilizar es de 8 horas, en realidad el valor empírico sigue dejando un margen, en este caso de unas 0,56 horas, al valor sobreestimado inicialmente. El valor de 8 horas es utilizado en la iteración III para realizar la planificación de trabajo correspondiente.\\
\indent En la tabla \ref{tabla_dif_horas_estim_iter_2} se muestran los errores de estimación por \hyperlink{US}{US} desarrollada si es que la velocidad inicial de desarrollo hubiese sido la calculada recientemente (8 horas).

\begin{table*}[h!]
	\centering
	\captionsetup{justification=centering,margin=1.5cm}
	\begin{tabular}{ |p{0.5cm}|l|c|c|c|c| }
		\hline
		\verb|#|& \textbf{Nombre de \hyperlink{US}{US}}& \textbf{Estimado[hs]} & \textbf{Real [hs]} & \textbf{Desviación [hs]} \\
		\hline
		\textbf{3} & Administrar insumo & 24 & 16,5 & \cellcolor{diferencia_estimacion_positiva}7,5 \\
		\hline
		\textbf{4} & Administrar tipo de insumo & 24 & 15 & \cellcolor{diferencia_estimacion_positiva}9 \\
		\hline
		\textbf{5} & Modificar cantidad de insumo & 24 & 28 & \cellcolor{diferencia_estimacion_negativa}-4 \\
		\hline
		\textbf{23} & Administrar tipo de tarea & 24 & 25 & \cellcolor{diferencia_estimacion_negativa}-1 \\
		\hline
		\textbf{24} & Visualizar novedades & 24 & 27 & \cellcolor{diferencia_estimacion_negativa}-3 \\
		\hline
	\end{tabular}
	\caption{Diferencias entre horas estimadas y reales de desarrollo para las \protect\hyperlink{US}{US} pertenecientes a la iteración II con el nuevo valor calculado de velocidad.}
	\label{tabla_dif_horas_estim_iter_2}
\end{table*}

\indent Es posible notar en este caso que, aún de haberse adoptado una velocidad más acorde a lo que finalmente se evidenció de manera práctica, las estimaciones seguirían desviándose de manera no menor en los casos de las \hyperlink{US}{US} 3 y 4, teniéndose en ambos casos valores de tiempos restantes de alrededor de una \hyperlink{US}{US} positiva restante (sobreestimación). Al margen del valor de velocidad utilizado, debería haber una proporción entre las estimaciones de esfuerzo plasmadas en la cantidad de \hyperlink{SP}{SP} asignados en cada caso y la complejidad de implementación de las \hyperlink{US}{US}. Una justificación razonable es que en los casos de las \hyperlink{US}{US} 5, 23 y 24 se debió utilizar una mayor cantidad de tiempo en el desarrollo por no haberse realizado algo similar con anterioridad. Claro está que quienes parecieran bien estimadas en este caso (ver tabla \ref{tabla_dif_horas_estim_iter_2}) son justamente las originalmente subestimadas, pero el problema, en vistas de tomarse un valor de velocidad nuevo basado en la práctica de la anterior iteración, está justamente en la falta de proporción entre las estimaciones de los  esfuerzos de las distintas \hyperlink{US}{US} como ya se ha mencionado.

\section{Iteración III}
\label{descripcion_iteracion_III}
\subsection{Planificación}
Las \hyperlink{US}{US} resultantes de lo descrito para el caso de la iteración II se muestran en la tabla \ref{tabla_dependencias_us_it3} (además de algunas adicionales que se creyó conveniente agregar) junto a sus dependencias de desarrollo (para un mayor detalle en cuanto a los refinamientos correspondientes de las \hyperlink{US}{US}, véase el \hyperlink{apendice_a_III}{Apéndice A - Iteración III}). Las \hyperlink{US}{US} que se muestran señaladas con color verde son las ya realizadas en las iteraciones precedentes y en color gris se indican las planificadas para ser realizadas en la presente iteración.

\begin{table*}[h!]
	\centering
	\begin{tabular}{ |p{0.5cm}|p{9cm}|c|  }
		\hline
		\verb|#|& \textbf{Nombre de \hyperlink{US}{US}}& \textbf{Dependencia} \\
		\hline
		\textbf{1} & \cellcolor{marca_US_realizada_anterior}Inicio de sesión & - \\
		\hline
		\textbf{2} & \cellcolor{marca_US_realizada_anterior}Administrar usuarios & 1 \\
		\hline
		\textbf{3} & \cellcolor{marca_US_realizada_anterior}Administrar insumo & 4 \\
		\hline
		\textbf{4} & \cellcolor{marca_US_realizada_anterior}Administrar tipo de insumo & 1 \\
		\hline
		\textbf{5} & \cellcolor{marca_US_realizada_anterior}Modificar cantidad de insumo & 3 y 6 \\
		\hline
		\textbf{6} & \cellcolor{marca_US_realizada_anterior}Administrar depósitos & 7 \\
		\hline
		\textbf{7} & \cellcolor{marca_US_realizada_anterior}Administrar sucursales & 1 \\
		\hline
		\textbf{8} & \cellcolor{marca_US_realizada}Administrar proyecto & 10 \\
		\hline
		\textbf{9} & Agregar insumo a proyecto & 8 y 3 \\
		\hline
		\textbf{10} & \cellcolor{marca_US_realizada}Administrar tipos de proyectos & 23 \\
		\hline
		\textbf{11} & Editar tarea & 8 \\
		\hline
		\textbf{12} & \cellcolor{marca_US_realizada}Asignar tarea a empleado & 11\\
		\hline
		\textbf{13} & Agregar comentario a tarea & 12 \\
		\hline
		\textbf{14} & Modificar estado de tarea & 11 \\
		\hline
		\textbf{15} & Finalizar proyecto & 8 \\
		\hline
		\textbf{16} & \cellcolor{marca_US_realizada}Buscar proyecto & 8 \\
		\hline
		\textbf{17} & \cellcolor{marca_US_realizada}Visualizar tareas asignadas & 12 \\
		\hline
		\textbf{18} & \cellcolor{marca_US_realizada_anterior}Administrar clientes & 1 \\
		\hline
		\textbf{19} & Eliminar usuario & 2 y 12 \\
		\hline
		\textbf{20} & Eliminar depósito & 6 y 3 \\
		\hline
		\textbf{21} & Eliminar sucursal & 20 \\
		\hline
		\textbf{22} & Eliminar cliente & 8 \\
		\hline
		\textbf{23} & \cellcolor{marca_US_realizada_anterior}Administrar tipo de tarea & 1 
		\\
		\hline
		\textbf{24} & \cellcolor{marca_US_realizada_anterior}Visualizar novedades & 1 
		\\
		\hline
		\textbf{25} & Visualizar proyecto & 8 
		\\
		\hline
		\textbf{26} & Eliminar insumo & 3 
		\\
		\hline
		\textbf{27} & Eliminar tipo de insumo & 4 
		\\
		\hline
		\textbf{28} & Eliminar tipo de tarea & 23 
		\\
		\hline
	\end{tabular}
	\caption{Dependencias de desarrollo actualizadas entre las \protect\hyperlink{US}{US} definidas.}
	\label{tabla_dependencias_us_it3}
\end{table*}

\subsection{Comparación entre estimaciones y tiempos de ejecución}
Se muestra en la tabla \ref{tabla_dependencias_us_it3} de forma análoga para este caso las dependencias de las \hyperlink{US}{US} definidas para llevar a cabo la iteración III. Del mismo modo, las responsabilidades definidas para el equipo de desarrollo pueden verse en la tabla \ref{tabla_desarrollo_iter_3}. En este caso se han agregado las \hyperlink{US}{US} 26, 27 y 28, las cuales originalmente formaban parte de las \hyperlink{US}{US} 3, 4 y 23 respectivamente.\\
\indent El valor ajustado de velocidad al finalizar la iteración II había sido de 8 horas de trabajo individual por \hyperlink{SP}{SP}. Si se tiene en cuenta que cada integrante del equipo de desarrollo realiza 20 horas de trabajo semanales, y que para cada iteración se ha convenido en emplear un tiempo de tres semanas, se tienen 7,5 \hyperlink{SP}{SP} asignables a cada miembro para esta iteración.

\begin{table*}[h!]
	\centering
	\begin{tabular}{ |p{0.5cm}|p{6cm}|c|c|c|c| }
		\hline
		\verb|#|& \textbf{Nombre de \hyperlink{US}{US}}& \textbf{Encargado} & \textbf{Estimado [\hyperlink{SP}{SP}]} & \textbf{Real [\hyperlink{SP}{SP}]} \\
		\hline
		\textbf{8} & Administrar proyecto & \cellcolor{marca_US_luciano}Luciano S. & 4 & 4,25 \\
		\hline
		\textbf{10} & Administrar tipos de proyectos & \cellcolor{marca_US_emir}Emir S. & 3 & 3,75 \\
		\hline
		\textbf{12} & Asignar tarea a empleado & \cellcolor{marca_US_emir}Emir S. & 2 & 1,75 \\
		\hline
		\textbf{16} & Buscar proyecto & \cellcolor{marca_US_luciano}Luciano S. & 3 & 2,5 \\
		\hline
		\textbf{17} & Visualizar tareas asignadas & \cellcolor{marca_US_emir}Emir S. & 2 & 1,5 \\
		\hline
	\end{tabular}
	\caption{División de tareas de las \protect\hyperlink{US}{US} desarrolladas en la iteración III.}
	\label{tabla_desarrollo_iter_3}
\end{table*}

De acuerdo con los datos mostrados en la tabla \ref{tabla_desarrollo_iter_3}, con el valor de velocidad inicial de 8 horas de trabajo individual por \hyperlink{SP}{SP} (obtenida en base a la práctica de acuerdo a los datos ya abordados en la iteración II), se infiere que en esta tercera iteración se han realizado 14 \hyperlink{SP}{SP} (estimación inicial) en 110 horas de trabajo individual total real (13,75 \hyperlink{SP}{SP} reales). Se recalcula por lo tanto la velocidad de desarrollo como el cociente entre la cantidad de horas individuales totales trabajadas y la cantidad de \hyperlink{SP}{SP} estimadas, obteniéndose el nuevo valor de \begin{math}\lceil 7,86\rceil\end{math} horas (8 horas) de trabajo individual por \hyperlink{SP}{SP}. Es decir que en este caso el valor de la velocidad real fue ligeramente superior al de la iteración II, en tanto que la velocidad inicial tomada para la siguiente (iteración IV) será, del mismo modo que en la presente, de 8 horas.\\
\indent En la tabla \ref{tabla_dif_horas_estim_iter_3} se muestran los errores de estimación por \hyperlink{US}{US} desarrollada tomando como velocidad inicial de desarrollo la nueva calculada (8 horas).

\begin{table*}[h!]
	\centering
	\captionsetup{justification=centering,margin=1.5cm}
	\begin{tabular}{ |p{0.5cm}|l|c|c|c|c| }
		\hline
		\verb|#|& \textbf{Nombre de \hyperlink{US}{US}}& \textbf{Estimado[hs]} & \textbf{Real [hs]} & \textbf{Desviación [hs]} \\
		\hline
		\textbf{8} & Administrar proyecto & 32 & 34 & \cellcolor{diferencia_estimacion_negativa}-2 \\
		\hline
		\textbf{10} & Administrar tipos de proyectos & 24 & 30 & \cellcolor{diferencia_estimacion_negativa}-6 \\
		\hline
		\textbf{12} & Asignar tarea a empleado & 16 & 14 & \cellcolor{diferencia_estimacion_positiva}2 \\
		\hline
		\textbf{16} & Buscar proyecto & 24 & 20 & \cellcolor{diferencia_estimacion_positiva}4 \\
		\hline
		\textbf{17} & Visualizar tareas asignadas & 16 & 12 & \cellcolor{diferencia_estimacion_positiva}4 \\
		\hline
	\end{tabular}
	\caption{Diferencias entre horas estimadas y reales de desarrollo para las \protect\hyperlink{US}{US} pertenecientes a la iteración III con el nuevo valor calculado de velocidad.}
	\label{tabla_dif_horas_estim_iter_3}
\end{table*}

\indent Considerando que en ningún caso se obtuvo una desviación de al menos 1 \hyperlink{SP}{SP}, en esta iteración las estimaciones han sido en general bastante aceptables.\\
\indent El valor de la velocidad calculada creció ligeramente manteniéndose por debajo de las 8 horas, pero esta vez con un margen de 0,14 horas, lo cual indica que, de mantenerse la complejidad en los desarrollos a realizar, se tiende a pensar que el valor podría permanecer estabilizado. Sin embargo, esto no es esperable ya que, al menos bajo un análisis rápido de las \hyperlink{US}{US} definidas y no desarrolladas hasta esta iteración, la complejidad debería ir bajando en tanto las \hyperlink{US}{US} abarcadas en las iteraciones II y III parecerían ser las más críticas. En adición, se añade a lo mencionado el uso de herramientas más específicas en las tecnologías empleadas sobre las cuales no se tenía completo conocimiento por parte del equipo de desarrollo, aspecto que debería influir de menor manera en las iteraciones subsiguientes.

\section{Iteración IV}
\label{descripcion_iteracion_IV}
\subsection{Planificación}
Las \hyperlink{US}{US} resultantes de lo descrito para el caso de la iteración III se muestran en la tabla \ref{tabla_dependencias_us_it4} junto a sus dependencias de desarrollo (para un mayor detalle en cuanto a los refinamientos correspondientes de las \hyperlink{US}{US}, véase el \hyperlink{apendice_a_IV}{Apéndice A - Iteración IV}). Las \hyperlink{US}{US} que se muestran señaladas con color verde son las ya realizadas en las iteraciones precedentes y en color gris se indican las planificadas para ser realizadas en la presente iteración.

\begin{table*}[h!]
	\centering
	\begin{tabular}{ |p{0.5cm}|p{9cm}|c|  }
		\hline
		\verb|#|& \textbf{Nombre de \hyperlink{US}{US}}& \textbf{Dependencia} \\
		\hline
		\textbf{1} & \cellcolor{marca_US_realizada_anterior}Inicio de sesión & - \\
		\hline
		\textbf{2} & \cellcolor{marca_US_realizada_anterior}Administrar usuarios & 1 \\
		\hline
		\textbf{3} & \cellcolor{marca_US_realizada_anterior}Administrar insumo & 4 \\
		\hline
		\textbf{4} & \cellcolor{marca_US_realizada_anterior}Administrar tipo de insumo & 1 \\
		\hline
		\textbf{5} & \cellcolor{marca_US_realizada_anterior}Modificar cantidad de insumo & 3 y 6 \\
		\hline
		\textbf{6} & \cellcolor{marca_US_realizada_anterior}Administrar depósitos & 7 \\
		\hline
		\textbf{7} & \cellcolor{marca_US_realizada_anterior}Administrar sucursales & 1 \\
		\hline
		\textbf{8} & \cellcolor{marca_US_realizada_anterior}Administrar proyecto & 10 \\
		\hline
		\textbf{9} & \cellcolor{marca_US_realizada}Agregar insumo a proyecto & 8 y 3 \\
		\hline
		\textbf{10} & \cellcolor{marca_US_realizada_anterior}Administrar tipos de proyectos & 23 \\
		\hline
		\textbf{11} & \cellcolor{marca_US_realizada}Editar tarea & 8 \\
		\hline
		\textbf{12} & \cellcolor{marca_US_realizada_anterior}Asignar tarea a empleado & 11\\
		\hline
		\textbf{13} & \cellcolor{marca_US_realizada}Agregar comentario a tarea & 12 \\
		\hline
		\textbf{14} & \cellcolor{marca_US_realizada}Modificar estado de tarea & 11 \\
		\hline
		\textbf{15} & \cellcolor{marca_US_realizada}Finalizar proyecto & 8 \\
		\hline
		\textbf{16} & \cellcolor{marca_US_realizada_anterior}Buscar proyecto & 8 \\
		\hline
		\textbf{17} & \cellcolor{marca_US_realizada_anterior}Visualizar tareas asignadas & 12 \\
		\hline
		\textbf{18} & \cellcolor{marca_US_realizada_anterior}Administrar clientes & 1 \\
		\hline
		\textbf{19} & Eliminar usuario & 2 y 12 \\
		\hline
		\textbf{20} & Eliminar depósito & 6 y 3 \\
		\hline
		\textbf{21} & Eliminar sucursal & 20 \\
		\hline
		\textbf{22} & Eliminar cliente & 8 \\
		\hline
		\textbf{23} & \cellcolor{marca_US_realizada_anterior}Administrar tipo de tarea & 1 
		\\
		\hline
		\textbf{24} & \cellcolor{marca_US_realizada_anterior}Visualizar novedades & 1 
		\\
		\hline
		\textbf{25} & \cellcolor{marca_US_realizada}Visualizar proyecto & 8 
		\\
		\hline
		\textbf{26} & Eliminar insumo & 3 
		\\
		\hline
		\textbf{27} & Eliminar tipo de insumo & 4 
		\\
		\hline
		\textbf{28} & Eliminar tipo de tarea & 23 
		\\
		\hline
		\textbf{29} & Eliminar proyecto & 8 
		\\
		\hline
	\end{tabular}
	\caption{Dependencias de desarrollo actualizadas entre las \protect\hyperlink{US}{US} definidas.}
	\label{tabla_dependencias_us_it4}
\end{table*}

\subsection{Comparación entre estimaciones y tiempos de ejecución}
Se muestra en la tabla \ref{tabla_dependencias_us_it4} de forma análoga para este caso las dependencias de las \hyperlink{US}{US} definidas para llevar a cabo la iteración IV. Del mismo modo, las responsabilidades definidas para el equipo de desarrollo pueden verse en la tabla \ref{tabla_desarrollo_iter_4}. En este caso se ha agregado la \hyperlink{US}{US} 29 (\textit{Eliminar proyecto}), la cual proviene de la \hyperlink{US}{US} 8.\\
\indent El valor ajustado de velocidad al finalizar la iteración III había sido de 8 horas de trabajo individual por \hyperlink{SP}{SP}. Si se tiene en cuenta que cada integrante del equipo de desarrollo realiza 20 horas de trabajo semanales, y que para cada iteración se ha convenido en emplear un tiempo de tres semanas, se tienen 7,5 \hyperlink{SP}{SP} asignables a cada miembro para esta iteración.

\begin{table*}[h!]
	\centering
	\begin{tabular}{ |p{0.5cm}|p{6cm}|c|c|c|c| }
		\hline
		\verb|#|& \textbf{Nombre de \hyperlink{US}{US}}& \textbf{Encargado} & \textbf{Estimado [\hyperlink{SP}{SP}]} & \textbf{Real [\hyperlink{SP}{SP}]} \\
		\hline
		\textbf{9} & Agregar insumo a proyecto & \cellcolor{marca_US_luciano}Luciano S. & 3 & 2,1875 \\
		\hline
		\textbf{11} & Editar tarea & \cellcolor{marca_US_luciano}Luciano S. & 2 & 2,875 \\
		\hline
		\textbf{13} & Agregar comentario a tarea & \cellcolor{marca_US_emir}Emir S. & 2 & 1,25 \\
		\hline
		\textbf{14} & Modificar estado de tarea & \cellcolor{marca_US_luciano}Luciano S. & 2 & 1,1875 \\
		\hline
		\textbf{15} & Finalizar proyecto & \cellcolor{marca_US_emir}Emir S. & 2 & 3,0625 \\
		\hline
		\textbf{25} & Visualizar proyecto & \cellcolor{marca_US_emir}Emir S. & 3 & 2,125 \\
		\hline
	\end{tabular}
	\caption{División de tareas de las \protect\hyperlink{US}{US} desarrolladas en la iteración IV.}
	\label{tabla_desarrollo_iter_4}
\end{table*}

De acuerdo con los datos mostrados en la tabla \ref{tabla_desarrollo_iter_4}, con el valor de velocidad inicial de 8 horas de trabajo individual por \hyperlink{SP}{SP} (obtenida en base a la práctica de acuerdo a los datos ya abordados en la iteración III), se infiere que en esta cuarta iteración se han realizado 14 \hyperlink{SP}{SP} (estimación inicial) en 101,5 horas de trabajo individual total real (12,69 \hyperlink{SP}{SP} reales). Se recalcula por lo tanto la velocidad de desarrollo como el cociente entre la cantidad de horas individuales totales trabajadas y la cantidad de \hyperlink{SP}{SP} estimadas, obteniéndose el nuevo valor de \begin{math}\lceil 7,25\rceil\end{math} horas (8 horas) de trabajo individual por \hyperlink{SP}{SP}. Es decir que en este caso el valor de la velocidad real fue ligeramente inferior al de la iteración III, guardando aún más cercanía con el valor obtenido en la iteración II y aventajando a este último por muy poco. La velocidad inicial tomada para la siguiente (iteración V) será, del mismo modo que en la presente, de 8 horas.\\
\indent En la tabla \ref{tabla_dif_horas_estim_iter_4} se muestran los errores de estimación por \hyperlink{US}{US} desarrollada tomando como velocidad inicial de desarrollo la nueva calculada (8 horas).

\begin{table*}[h!]
	\centering
	\captionsetup{justification=centering,margin=1.5cm}
	\begin{tabular}{ |p{0.5cm}|l|c|c|c|c| }
		\hline
		\verb|#|& \textbf{Nombre de \hyperlink{US}{US}}& \textbf{Estimado[hs]} & \textbf{Real [hs]} & \textbf{Desviación [hs]} \\
		\hline
		\textbf{9} & Agregar insumo a proyecto & 24 & 17,5 & \cellcolor{diferencia_estimacion_positiva}6,5 \\
		\hline
		\textbf{11} & Editar tarea & 16 & 23 & \cellcolor{diferencia_estimacion_negativa}-7 \\
		\hline
		\textbf{13} & Agregar comentario a tarea & 16 & 10 & \cellcolor{diferencia_estimacion_positiva}6 \\
		\hline
		\textbf{14} & Modificar estado de tarea & 16 & 9,5 & \cellcolor{diferencia_estimacion_positiva}6,5 \\
		\hline
		\textbf{15} & Finalizar proyecto & 16 & 24,5 & \cellcolor{diferencia_estimacion_negativa}-8,5 \\
		\hline
		\textbf{25} & Visualizar proyecto & 24 & 17 & \cellcolor{diferencia_estimacion_positiva}7 \\
		\hline
	\end{tabular}
	\caption{Diferencias entre horas estimadas y reales de desarrollo para las \protect\hyperlink{US}{US} pertenecientes a la iteración IV con el nuevo valor calculado de velocidad.}
	\label{tabla_dif_horas_estim_iter_4}
\end{table*}

\indent Es posible notar que no solo existe un caso de una desviación mayor a 1 \hyperlink{SP}{SP}, sino que también, en general las desviaciones fueron en valor absoluto mayores a las de la iteración III. En todos los casos se obtuvieron desviaciones superiores a 1/2 \hyperlink{SP}{SP}, algo solo observado en una de las \hyperlink{US}{US} de la iteración III.\\
\indent El valor de la velocidad calculada, como ya fuera mencionado, disminuyó aunque no de manera significativa, dejando invariante la velocidad práctica en el valor de 8 horas. En virtud del análisis de los valores obtenidos para el caso de la iteración III, se esperaba una disminución de la velocidad ya que se considera que las iteraciones II y III contenían las \hyperlink{US}{US} de mayor criticidad y complejidad. En este último aspecto, sí es cierto que el desarrollo se facilitó en cuanto al uso de las herramientas al tenerse ya ejemplos de formas de implementación de código para funcionalidades relativas a \hyperlink{US}{US} ya tratadas, sin embargo, la complejidad para los casos particulares de \textit{Editar tarea} y \textit{Visualizar proyecto}, hicieron que la disminución no fuera significativa.\\
\indent Aún si no dejan de ser inusuales los errores en las estimaciones, se compensaron entre subestimaciones y sobreestimaciones dejando, en promedio, 10,5 horas a favor (sobreestimación). Asimismo, a este respecto cabe reconocer que habría de esperar en este caso al menos que las estimaciones particulares no se hubieran desviado como en ninguna otra iteración anterior (a simple vista es notoria una desviación media superior a las de las iteraciones previas), teniendo en cuenta que el equipo de trabajo debería haber adquirido a través de las distintas iteraciones mejores habilidades en las estimaciones. Estas discrepancias inusuales, se cree, pueden haber tenido origen en que para las estimaciones no se tuvo en cuenta el aspecto de integración y congruencia que se debe cumplir en las \hyperlink{US}{US} tratadas para este caso, esto teniendo en cuenta que para los casos anteriores las \hyperlink{US}{US} tenían incidencias en el uso del sistema por la importancia de sus funcionalidades, mas no interferían marcadamente de manera inicial en el flujo de trabajo del usuario, no debiéndose dar demasiada importancia a la interacción entre distintas pantallas para lograr una buena usabilidad del sistema.

\section{Iteración V}
\label{descripcion_iteracion_V}
\subsection{Planificación}
Las \hyperlink{US}{US} resultantes de lo descrito para el caso de la iteración IV se muestran en la tabla \ref{tabla_dependencias_us_it5} junto a sus dependencias de desarrollo (para un mayor detalle en cuanto a los refinamientos correspondientes de las \hyperlink{US}{US}, véase el \hyperlink{apendice_a_V}{Apéndice A - Iteración V}). Las \hyperlink{US}{US} que se muestran señaladas con color verde son las ya realizadas en las iteraciones precedentes y en color gris se indican las planificadas para ser realizadas en la presente iteración.

\begin{table*}[h!]
	\centering
	\begin{tabular}{ |p{0.5cm}|p{10cm}|c|  }
		\hline
		\verb|#|& \textbf{Nombre de \hyperlink{US}{US}}& \textbf{Dependencia} \\
		\hline
		\textbf{1} & \cellcolor{marca_US_realizada_anterior}Inicio de sesión & - \\
		\hline
		\textbf{2} & \cellcolor{marca_US_realizada_anterior}Administrar usuarios & 1 \\
		\hline
		\textbf{3} & \cellcolor{marca_US_realizada_anterior}Administrar insumo & 4 \\
		\hline
		\textbf{4} & \cellcolor{marca_US_realizada_anterior}Administrar tipo de insumo & 1 \\
		\hline
		\textbf{5} & \cellcolor{marca_US_realizada_anterior}Modificar cantidad de insumo & 3 y 6 \\
		\hline
		\textbf{6} & \cellcolor{marca_US_realizada_anterior}Administrar depósitos & 7 \\
		\hline
		\textbf{7} & \cellcolor{marca_US_realizada_anterior}Administrar sucursales & 1 \\
		\hline
		\textbf{8} & \cellcolor{marca_US_realizada_anterior}Administrar proyecto & 10 \\
		\hline
		\textbf{9} & \cellcolor{marca_US_realizada_anterior}Agregar insumo a proyecto & 8 y 3 \\
		\hline
		\textbf{10} & \cellcolor{marca_US_realizada_anterior}Administrar tipos de proyectos & 23 \\
		\hline
		\textbf{11} & \cellcolor{marca_US_realizada_anterior}Editar tarea & 8 \\
		\hline
		\textbf{12} & \cellcolor{marca_US_realizada_anterior}Asignar tarea a empleado & 11\\
		\hline
		\textbf{13} & \cellcolor{marca_US_realizada_anterior}Agregar comentario a tarea & 12 \\
		\hline
		\textbf{14} & \cellcolor{marca_US_realizada_anterior}Modificar estado de tarea & 11 \\
		\hline
		\textbf{15} & \cellcolor{marca_US_realizada_anterior}Finalizar proyecto & 8 \\
		\hline
		\textbf{16} & \cellcolor{marca_US_realizada_anterior}Buscar proyecto & 8 \\
		\hline
		\textbf{17} & \cellcolor{marca_US_realizada_anterior}Visualizar tareas asignadas & 12 \\
		\hline
		\textbf{18} & \cellcolor{marca_US_realizada_anterior}Administrar clientes & 1 \\
		\hline
		\textbf{19} & \cellcolor{marca_US_realizada}Eliminar usuario & 2 y 12 \\
		\hline
		\textbf{20} & \cellcolor{marca_US_realizada}Eliminar depósito & 6 y 3 \\
		\hline
		\textbf{21} & \cellcolor{marca_US_realizada}Eliminar sucursal & 20 \\
		\hline
		\textbf{22} & \cellcolor{marca_US_realizada}Eliminar cliente & 8 \\
		\hline
		\textbf{23} & \cellcolor{marca_US_realizada_anterior}Administrar tipo de tarea & 1 
		\\
		\hline
		\textbf{24} & \cellcolor{marca_US_realizada_anterior}Visualizar novedades & 1 
		\\
		\hline
		\textbf{25} & \cellcolor{marca_US_realizada_anterior}Visualizar proyecto & 8 
		\\
		\hline
		\textbf{26} & \cellcolor{marca_US_realizada}Eliminar insumo & 3 
		\\
		\hline
		\textbf{27} & \cellcolor{marca_US_realizada}Eliminar tipo de insumo & 4 
		\\
		\hline
		\textbf{28} & \cellcolor{marca_US_realizada}Eliminar tipo de tarea & 23 
		\\
		\hline
		\textbf{29} & \cellcolor{marca_US_realizada}Eliminar proyecto & 8 
		\\
		\hline
		\textbf{30} & \cellcolor{marca_US_realizada}Eliminar tipo de proyecto & 10 
		\\
		\hline
		\textbf{31} & \cellcolor{marca_US_realizada}Generar reporte de insumos disponibles en cada depósito & 5 
		\\
		\hline
	\end{tabular}
	\caption{Dependencias de desarrollo actualizadas entre las \protect\hyperlink{US}{US} definidas.}
	\label{tabla_dependencias_us_it5}
\end{table*}

\subsection{Comparación entre estimaciones y tiempos de ejecución}
Se muestra en la tabla \ref{tabla_dependencias_us_it5} de forma análoga para este caso las dependencias de las \hyperlink{US}{US} definidas para llevar a cabo la iteración V. Del mismo modo, las responsabilidades definidas para el equipo de desarrollo pueden verse en la tabla \ref{tabla_desarrollo_iter_5}. En este caso se han agregado las \hyperlink{US}{US} 30 (\textit{Eliminar tipo de proyecto}) y 31 (\textit{Generar reporte de insumos disponibles en cada depósito}), las cuales provienen de las \hyperlink{US}{US} 10 y de la necesidad del cliente de generar un registro del estado de los depósitos (algo que se hizo evidente luego de la implementación de las \hyperlink{US}{US} 20 y 21, \textit{Eliminar depósito} y \textit{Eliminar sucursal}) respectivamente.\\
\indent El valor ajustado de velocidad al finalizar la iteración IV había sido de 8 horas de trabajo individual por \hyperlink{SP}{SP}. Si se tiene en cuenta que cada integrante del equipo de desarrollo realiza 20 horas de trabajo semanales, y que para cada iteración se ha convenido en emplear un tiempo de tres semanas, se tienen 7,5 \hyperlink{SP}{SP} asignables a cada miembro para esta iteración.

\begin{table*}[h!]
	\centering
	\begin{tabular}{ |p{0.5cm}|p{6cm}|c|c|c|c| }
		\hline
		\verb|#|& \textbf{Nombre de \hyperlink{US}{US}}& \textbf{Encargado} & \textbf{Estimado [\hyperlink{SP}{SP}]} & \textbf{Real [\hyperlink{SP}{SP}]} \\
		\hline
		\textbf{19} & Eliminar usuario & \cellcolor{marca_US_emir}Emir S. & 1 & 0,875 \\
		\hline
		\textbf{20} & Eliminar depósito & \cellcolor{marca_US_luciano}Luciano S. & 1 & 0,75 \\
		\hline
		\textbf{21} & Eliminar sucursal & \cellcolor{marca_US_luciano}Luciano S. & 1 & 0,625 \\
		\hline
		\textbf{22} & Eliminar cliente & \cellcolor{marca_US_emir}Emir S. & 0,5 & 0,75 \\
		\hline
		\textbf{26} & Eliminar insumo & \cellcolor{marca_US_luciano}Luciano S. & 0,5 & 0,375 \\
		\hline
		\textbf{27} & Eliminar tipo de insumo & \cellcolor{marca_US_luciano}Luciano S. & 0,5 & 0,375 \\
		\hline
		\textbf{28} & Eliminar tipo de tarea & \cellcolor{marca_US_emir}Emir S. & 1 & 1,25 \\
		\hline
		\textbf{29} & Eliminar proyecto & \cellcolor{marca_US_emir}Emir S. & 1 & 1 \\
		\hline
		\textbf{30} & Eliminar tipo de proyecto & \cellcolor{marca_US_emir}Emir S. & 0,5 & 0,375 \\
		\hline
		\textbf{31} & Generar reporte de insumos disponibles en cada depósito & \cellcolor{marca_US_luciano}Luciano S. & 1 & 1,625 \\
		\hline
	\end{tabular}
	\caption{División de tareas de las \protect\hyperlink{US}{US} desarrolladas en la iteración V.}
	\label{tabla_desarrollo_iter_5}
\end{table*}

De acuerdo con los datos mostrados en la tabla \ref{tabla_desarrollo_iter_5}, con el valor de velocidad inicial de 8 horas de trabajo individual por \hyperlink{SP}{SP} (obtenida en base a la práctica de acuerdo a los datos ya abordados en la iteración IV), se infiere que en esta quinta iteración se han realizado 8 \hyperlink{SP}{SP} (estimación inicial) en 64 horas de trabajo individual total real (8 \hyperlink{SP}{SP} reales). Se recalcula por lo tanto la velocidad de desarrollo como el cociente entre la cantidad de horas individuales totales trabajadas y la cantidad de \hyperlink{SP}{SP} estimadas, obteniéndose el nuevo valor de \begin{math}\lceil 8\rceil\end{math} horas (8 horas) de trabajo individual por \hyperlink{SP}{SP}. Es decir que en este caso el valor de la velocidad real fue superior al de la iteración IV, pero sin modificar el valor de cantidad de horas a emplear por \hyperlink{US}{US} en una eventual siguiente iteración (VI), que en tal caso debería ser a su vez de 8 horas.\\
\indent En la tabla \ref{tabla_dif_horas_estim_iter_5} se muestran los errores de estimación por \hyperlink{US}{US} desarrollada tomando como velocidad inicial de desarrollo la nueva calculada (8 horas).

\begin{table*}[h!]
	\centering
	\captionsetup{justification=centering,margin=1.5cm}
	\begin{tabular}{ |p{0.5cm}|l|c|c|c|c| }
		\hline
		\verb|#|& \textbf{Nombre de \hyperlink{US}{US}}& \textbf{Estimado[hs]} & \textbf{Real [hs]} & \textbf{Desviación [hs]} \\
		\hline
		\textbf{19} & Eliminar usuario & 8 & 7 & \cellcolor{diferencia_estimacion_positiva}1 \\
		\hline
		\textbf{20} & Eliminar depósito & 8 & 6 & \cellcolor{diferencia_estimacion_positiva}2 \\
		\hline
		\textbf{21} & Eliminar sucursal & 8 & 5 & \cellcolor{diferencia_estimacion_positiva}3 \\
		\hline
		\textbf{22} & Eliminar cliente & 4 & 6 & \cellcolor{diferencia_estimacion_negativa}-2 \\
		\hline
		\textbf{26} & Eliminar insumo & 4 & 3 & \cellcolor{diferencia_estimacion_positiva}1 \\
		\hline
		\textbf{27} & Eliminar tipo de insumo & 4 & 3 & \cellcolor{diferencia_estimacion_positiva}1 \\
		\hline
		\textbf{28} & Eliminar tipo de tarea & 8 & 10 & \cellcolor{diferencia_estimacion_negativa}-2 \\
		\hline
		\textbf{29} & Eliminar proyecto & 8 & 8 & \cellcolor{diferencia_estimacion_positiva}0 \\
		\hline
		\textbf{30} & Eliminar tipo de proyecto & 4 & 3 & \cellcolor{diferencia_estimacion_positiva}1 \\
		\hline
		\textbf{31} & \parbox[c]{6cm}{\vspace{2pt} Generar reporte de insumos disponibles en cada depósito \vspace{2pt}} & 8 & 13 & \cellcolor{diferencia_estimacion_negativa}-5 \\
		\hline
	\end{tabular}
	\caption{Diferencias entre horas estimadas y reales de desarrollo para las \protect\hyperlink{US}{US} pertenecientes a la iteración V con el nuevo valor calculado de velocidad.}
	\label{tabla_dif_horas_estim_iter_5}
\end{table*}

\indent Considerando que en ningún caso se obtuvo una desviación de al menos 1 \hyperlink{SP}{SP}, en esta iteración las estimaciones han sido en general bastante aceptables. Asimismo, se debe tener en cuenta que, casi en la totalidad de los casos, las \hyperlink{US}{US} tratadas en la iteración han tenido relación mayormente con eliminaciones en la base de datos. Solo en la \hyperlink{US}{US} 31 las complejidades provinieron de las propias de la actividad de desarrollo, y, en este caso particular, de generar un reporte, para lo que no se tenía hasta el momento ningún antecedente. Es por esto que no es inesperado que ninguna de las estimaciones haya excedido el valor de 1 \hyperlink{SP}{SP}, y, debido a la baja complejidad de las \hyperlink{US}{US}, las estimaciones hayan sido relativamente buenas.\\
\indent El valor de la velocidad calculada, como ya fuera mencionado, se mantuvo. De hecho, en este caso se obtuvo un valor exacto de 8 horas, dejando a su vez invariante la velocidad práctica, aunque en los anteriores casos se tomara necesariamente el entero superior (techo) del valor calculado. De cualquier modo, no se ha hecho uso en este caso del tiempo total de la iteración, utilizándose solo 64 de las 120 horas disponibles.

\section{Modificaciones finales}
Luego de concluida la iteración V, fue posible obtener un diagrama final de dependencias entre las \hyperlink{US}{US} definidas a lo largo del proyecto. Este puede verse en la figura \ref{diagrama_dependencias_US_final}, y a su vez contrastarse con el previamente realizado con las \hyperlink{US}{US} definidas antes del inicio de la iteración I, mostrado en la figura \ref{diagrama_dependencias_US}.

\begin{figure}[h]
	\begin{center}
		\setlength{\fboxsep}{10pt} % Ajusta el padding
		\fbox{
			\begin{tikzpicture}[scale=1.1]
				\GraphInit[vstyle=Normal]
				\SetVertexNormal[Shape=circle,LineWidth = 1pt]
				\tikzset{EdgeStyle/.append style = {color = blue!70, line width=1pt}}
				\renewcommand*{\VertexLineWidth}{1pt}%vertex thickness
				\renewcommand*{\EdgeLineWidth}{1pt}% edge thickness
				\GraphInit[vstyle=Normal]
				
				\Vertex[LabelOut,Lpos=270,L=$us_1$,x=0,y=2]{R1}
				
				\Vertex[LabelOut,Lpos=270,L=$us_2$,x=2,y=8]{R2}
				\Vertex[LabelOut,Lpos=270,L=$us_7$,x=2,y=6]{R7}
				\Vertex[LabelOut,Lpos=270,L=$us_4$,x=2,y=4]{R4}
				\Vertex[LabelOut,Lpos=270,L=$us_{18}$,x=2,y=2]{R18}
				\Vertex[LabelOut,Lpos=270,L=$us_{23}$,x=2,y=0]{R23}
				\Vertex[LabelOut,Lpos=270,L=$us_{24}$,x=2,y=-4]{R24}
				
				\Vertex[LabelOut,Lpos=270,L=$us_6$,x=4,y=6]{R6}
				\Vertex[LabelOut,Lpos=270,L=$us_3$,x=4,y=4]{R3}
				\Vertex[LabelOut,Lpos=270,L=$us_{27}$,x=4,y=3]{R27}
				\Vertex[LabelOut,Lpos=270,L=$us_{10}$,x=4,y=0]{R10}
				\Vertex[LabelOut,Lpos=270,L=$us_{28}$,x=4,y=-2]{R28}	\Vertex[LabelOut,Lpos=270,L=$us_{30}$,x=6,y=-1]{R30}
				
				\Vertex[LabelOut,Lpos=270,L=$us_{20}$,x=6,y=6]{R20}
				\Vertex[LabelOut,Lpos=270,L=$us_5$,x=6,y=5]{R5}
				\Vertex[LabelOut,Lpos=270,L=$us_{26}$,x=6,y=4]{R26}
				\Vertex[LabelOut,Lpos=270,L=$us_8$,x=6,y=1]{R8}
				
				\Vertex[LabelOut,Lpos=270,L=$us_{21}$,x=8,y=6]{R21}
				\Vertex[LabelOut,Lpos=270,L=$us_{31}$,x=8,y=5]{R31}
				\Vertex[LabelOut,Lpos=270,L=$us_9$,x=8,y=3]{R9}
				\Vertex[LabelOut,Lpos=270,L=$us_{16}$,x=8,y=2]{R16}
				\Vertex[LabelOut,Lpos=270,L=$us_{11}$,x=8,y=1]{R11}
				\Vertex[LabelOut,Lpos=270,L=$us_{15}$,x=8,y=0]{R15}
				\Vertex[LabelOut,Lpos=270,L=$us_{22}$,x=8,y=-1]{R22}
				\Vertex[LabelOut,Lpos=270,L=$us_{25}$,x=8,y=-2]{R25}
				\Vertex[LabelOut,Lpos=270,L=$us_{29}$,x=8,y=-3]{R29}
				
				\Vertex[LabelOut,Lpos=270,L=$us_{12}$,x=10,y=2]{R12}
				\Vertex[LabelOut,Lpos=270,L=$us_{14}$,x=10,y=0]{R14}
				
				\Vertex[LabelOut,Lpos=270,L=$us_{19}$,x=12,y=8]{R19}
				\Vertex[LabelOut,Lpos=270,L=$us_{13}$,x=12,y=3]{R13}
				\Vertex[LabelOut,Lpos=270,L=$us_{17}$,x=12,y=1]{R17}
				
				%%%%%%%%%%%%%%%%%%%%%%%%%%%%%%
				
				\Edge (R1)(R2)
				\Edge (R1)(R7)
				\Edge (R1)(R4)
				\Edge (R1)(R18)
				\Edge (R1)(R23)
				\Edge (R1)(R24)
				
				\Edge (R7)(R6)
				\Edge (R4)(R3)
				\Edge (R4)(R27)
				\Edge (R23)(R10)
				\Edge (R23)(R28)
				
				\Edge (R6)(R20)
				\Edge (R6)(R5)
				\Edge (R3)(R20)
				\Edge (R3)(R5)
				\Edge (R3)(R26)
				\Edge (R10)(R8)
				\Edge (R10)(R30)
				
				\Edge (R20)(R21)
				\Edge (R5)(R31)
				\Edge (R3)(R9)
				\Edge (R8)(R9)
				\Edge (R8)(R16)
				\Edge (R8)(R11)
				\Edge (R8)(R15)
				\Edge (R8)(R22)
				\Edge (R8)(R25)
				\Edge (R8)(R29)
				
				\Edge (R11)(R12)
				\Edge (R11)(R14)
				
				\Edge (R2)(R19)
				\Edge (R12)(R19)
				\Edge (R12)(R13)
				\Edge (R12)(R17)
				
			\end{tikzpicture}
		}
		\caption{Diagrama de dependencias entre las \protect\hyperlink{US}{US} definidas hasta la iteración V.}
		\label{diagrama_dependencias_US_final}
	\end{center}
\end{figure}

Asimismo, se realizaron modificaciones que se creyeron pertinentes, las cuales tuvieron mayormente relación con la homogeneidad del formato esperado en el sistema, así como con aspectos tratados de manera laxa al momento de realizarse las iteraciones.\\
\indent De un modo general pueden listarse las siguientes clases de modificaciones llevadas a cabo:
\begin{itemize}
	\item Verificación del formato de entrada de datos mediante el uso de expresiones regulares.
	\item Verificación de propiedades deseadas en el ingreso de datos del usuario. Por ejemplo, que en la estructura dada a un tipo de proyecto, los tipos de tarea que la conforman junto con sus interdependencias puedan ser representados en forma de grafo dirigido acíclico (DAG) conexo.
	\item Adición de modales de confirmación en los casos en que se crea necesario para confirmar modificaciones consideradas más críticas.
	\item Adición de campos en tablas de la base de datos, para facilitar un manejo más simple y directo de los datos obtenidos en las consultas del sistema.
	\item Mejoras en la estética de las pantallas así como en la presentación de los datos para lograr homogeneidad en el sistema. Por ejemplo, la señalización con color rojo de los botones que producen la supresión de registros en la base de datos.
\end{itemize}
\indent Estas modificaciones se realizaron entre el 10/03/2025 y el 21/03/2025, tomando alrededor de 40 horas de trabajo individual aproximadamente. Con fines comparativos, teniendo en cuenta que en una iteración (de acuerdo a lo establecido en nuestro caso) se cuenta con 120 horas de trabajo individual, es posible decir que esto equivalió a un tiempo de trabajo del de alrededor de un tercio de una iteración común.

\chapter{Control de calidad}
\label{Control de calidad}
Si bien inicialmente, en lo relativo al control de calidad, se había previsto únicamente la realización de pruebas \begin{math}\alpha\end{math} (\textit{Alpha}), lo llevado a la práctica estrictamente no puede considerarse como tal en este caso, ya que las pruebas se asemejaron más a una inspección superficial por parte del usuario, orientada a detectar aspectos a tener en cuenta para las siguientes etapas por parte del equipo de desarrollo.\\
\indent Por otro lado, si bien en gran medida las pruebas unitarias se esperaba que fueran abarcadas en las iteraciones de desarrollo (por estar intrínsecamente implícitas en el proceso de desarrollo de software), no se tomó en cuenta la carencia de pruebas de integración en ese marco. Más aún, por la dinámica que empíricamente se pudo observar, el equipo de desarrollo tendió a realizar pruebas unitarias muy laxas, por lo que incluso esto debió realizarse con mayor rigurosidad en una etapa posterior a la ejecución de todas las iteraciones tratadas en el presente trabajo.\\
\indent Durante la actividad mencionada se tomó nota de los distintos errores hallados, de modo de poder clasificarlos de acuerdo a las etapas de desarrollo particulares (iteraciones) y analizar brevemente tales correspondencias. Las descripciones de tales errores se omiten en este capítulo ya que no se cree de relevancia puntualizar en tal aspecto en este caso, pero para mayor detalle estas fueron abordadas en el \hyperlink{apendice_h}{Apéndice H}.

\indent En la figura \ref{grafico_errores_iteracion} se muestra un gráfico de barras donde se plasman los resultados de la actividad mencionada con anterioridad, con la cantidad de errores observados durante las pruebas realizadas y atribuibles a cada una de las iteraciones.

\begin{figure}[h]
	\centering
	\begin{tikzpicture}
		\begin{axis}[ 
			ybar,
			bar width=20pt,
			ylabel={Cantidad de errores},
			xlabel={Iteración},
			symbolic x coords={1, 2, 3, 4, 5},
			xtick=data,
			nodes near coords,
			every node near coord/.append style={font=\small},
			enlarge x limits=0.2,
			ymin=0,
			grid=both,
			every axis plot/.append style={draw=black, thick},
			axis line style={draw=black, thick},
			]
			\addplot[
			fill=orange!60,
			postaction={
				pattern=north east lines, % Se dibuja el borde
				draw=orange, thick,
				% Elimina el borde izquierdo solo en la primera barra
				each nth plot 1/.append style={postaction={draw=none}},
			}
			] coordinates {(1,3) (2,10) (3,8) (4,13) (5,3)};
		\end{axis}
	\end{tikzpicture}
	\caption{Gráfico de la cantidad de errores evidenciados en cada iteración.}
	\label{grafico_errores_iteracion}
\end{figure}

\indent Al analizar los errores encontrados, es posible notar que en general se debieron a carencias en cuanto a restricciones a cumplir en la lógica del código implementado. Se han tenido problemas de verificaciones (condiciones bajo las cuales es posible realizar una acción) en todas las iteraciones, mientras que los errores de secuencias entre las pantallas solo fueron halladas en las iteraciones II, III y IV, y esto claramente se debe a que las iteraciones I y V corresponden a \hyperlink{US}{US} donde se ejecutan acciones sobre pantallas <<aisladas>> (con poca o nula interacción en su secuencia con otras).\\
\indent En las iteraciones II, III y IV se observaron la mayor cantidad de errores, lo que puede ser justificado con la complejidad de tales iteraciones. Sin embargo, se mencionó que las iteraciones de mayor criticidad eran las II y III, mientras que la iteración IV fue en la que mayor cantidad de errores fueron encontrados. De cualquier modo, en este caso los errores mencionados no se deben en gran medida a restricciones no tenidas en cuenta, sino a cuestiones de reutilización de funciones previas, como es el caso del lanzamiento de novedades ante determinada condición (relacionada con la \hyperlink{US}{US} 24, desarrollada en la iteración II, \textit{Visualizar novedades}), o cuestiones de acceso vinculadas con los perfiles de usuario, lo que tiene relación con temas de integración (ambos casos dependientes de la comunicación entre los integrantes del equipo de desarrollo, tema que será tratado en la sección \ref{comparacion_metodologia_implementada}).\\
\indent La búsqueda y corrección de estos errores se realizaron entre el 24/03/2025 y el 25/04/2025, insumiendo alrededor de 80 horas de trabajo individual aproximadamente. A modo de comparación, teniéndose, de acuerdo a nuestro caso, 120 horas de trabajo individual asignables en una iteración, se puede decir que esto tiene su equivalencia con el tiempo de alrededor de dos tercios de una iteración común.


\chapter{Comparación entre teoría y práctica}

\section{En la metodología implementada}
\label{comparacion_metodologia_implementada}
Se considera que, si bien el análisis presentado en la sección \ref{Caracteristicas_generales_del_proyecto} acerca de las características del proyecto fue correcto, hubieron aspectos que en la práctica tuvieron incidencia en el desarrollo normal de las actividades:
\begin{itemize}
	\item La división de las tareas de desarrollo en \hyperlink{US}{US}, si bien facilitó el trabajo del equipo haciendo que cada integrante se enfoque en aspectos específicos, también evidenció la ausencia de una correcta comunicación interna. Esto tuvo principal impacto en la integración, ya que, a grandes rasgos, la especialización hizo perder de vista el desarrollo global así como el flujo de trabajo esperado.
	\item En la sección \ref{Modelo de desarrollo del software utilizado} se mencionó la premisa de \textit{Comunicación constante con el cliente}. En este sentido, si bien en general la disponibilidad del cliente fue buena, hubo en principio un error al no dejar un tiempo entre iteraciones para poder realizar la planificación de acuerdo a la reunión que se tuviera con este, ya que era claro que, en cualquier caso, la reunión sería realizada dentro de los tiempos de ejecución de la iteración siguiente, algo que se debió haber intentado evitar desde el principio.\\
	Por otro lado, como fuera ya mencionado en el capítulo \ref{Control de calidad}, las reuniones trataron en mayor medida temas superficiales del sistema, por lo que no se pudieron obtener mayores precisiones relativas a nuevos requerimientos para ser tratados en las siguientes iteraciones (no al menos de una manera exhaustiva, como cabría esperar). Esto es resultado a su vez de una falta de comunicación constate con el cliente durante la ejecución de las iteraciones, lo que hubiese facilitado al cliente inmiscuirse en mayor grado en los distintos aspectos relevantes para el desarrollo del sistema. Esto último le hubiese permitido brindar mayores precisiones de acuerdo a los requerimientos esperados.
	\item Debido a la falta de un sistema de alcance similar al tratado en el presente trabajo, los requisitos no estuvieron completamente definidos desde un inicio. Sin embargo, las modificaciones en cuanto a los requisitos planteados de manera inicial no fueron significativos, y se debieron principalmente a aspectos no tenidos en cuenta al comenzar con el proyecto. En este orden de ideas, y teniendo en cuenta en retrospectiva la libertad que el cliente terminaría brindando, no parece haberse hecho demasiado uso de la posibilidad de modificaciones en los requisitos durante la ejecución del proyecto, posibilidad que se tiene como resultado de la aplicación de una metodología iterativa basada en metodologías ágiles. Es decir, que tal vez la implementación de una metodología clásica, con un análisis más detallado de los requisitos al inicio del proyecto, hubiese sido igualmente (o posiblemente más) adecuada para su ejecución.
\end{itemize}

\section{En la gestión de riesgos}
En el \hyperlink{apendice_d}{Apéndice D} se muestra el \textit{Plan de gestión de riesgos}. En particular, como será tratado con mayor detalle en la sección \ref{en_las_estimaciones_de_tiempo}, en la iteración IV se materializó el riesgo 4 (a saber, \textit{Imposibilidad de trabajar la cantidad de horas pactadas}), mientras que en la iteración V fue el riesgo 7 el ocurrido (a saber, \textit{Falta de disponibilidad por parte del cliente}). Ambos casos son los que no pudieron ser correctamente mitigados mediante la implementación del plan de respuesta a riesgos, por lo que a continuación se abordan ambos casos:
\begin{itemize}
	\item \underline{Imposibilidad de trabajar la cantidad de horas pactadas:}\\
	Al elaborar el \textit{Plan de gestión de riesgos} no se tuvo en cuenta la posibilidad de que en algún momento ninguno de los integrantes del equipo pudiera hacerse cargo de la iteración ejecutada. Dado que ambos integrantes se encontraron, durante gran parte del tiempo de ejecución del presente proyecto, involucrados con la realización de otro proyecto de software cuya atención debió ser prioritaria durante la ejecución de la iteración IV, no pudo hacerse frente de manera adecuada a la ocurrencia del riesgo 4.\\
	Lo previsto, por estar más acorde con la teoría\footnote{\textit{<<La iteración termina en la fecha especificada, aún si todas las \hyperlink{US}{US} no están hechas>>} (traducción del texto original). Martin, Robert C. \textit{<<Agile Software Development: Principles, Patterns, and Practices>>}.  Pearson Education, 2003. Pág. 21.}, habría sido asumir una velocidad de trabajo inferior, de modo que en el tiempo de la iteración se realizaran una menor cantidad de \hyperlink{US}{US}. De cualquier modo, esto hubiese ocasionado un aumento en la cantidad de iteraciones, dando por resultado algo similar a lo finalmente ocurrido. Además, en los plazos de la iteración IV difícilmente se podría haber llevado una correcta coordinación del equipo, ya que la disponibilidad de cada miembro fue variando de manera considerable a lo largo del periodo. De hecho, en sentido estricto, de tender a cero la cantidad de \hyperlink{US}{US} realizadas en alguna iteración, se perdería un poco el análisis de la evolución de la velocidad del equipo a través de las iteraciones con la premisa de un tiempo de trabajo constante (calculado en 120 horas por iteración para el equipo); en este sentido fue que se decidió dejar invariante el tiempo de trabajo en detrimento del tiempo (teóricamente) constante de la iteración.
	\item \underline{Falta de disponibilidad por parte del cliente:}\\
	Ya que el miembro de la empresa que principalmente estuvo al tanto del estado del proyecto así como de la definición de los requisitos del mismo fue el gerente general, su ausencia durante el mes de enero ocasionó un retraso en la ejecución de la iteración V. Esto pudo haber sido evitado en mayor medida proponiendo a su vez otro miembro que tuviera conocimiento de lo relativo al proyecto. Sin embargo, ya que el gerente general es el principal interesado en la ejecución del proyecto, y dado que es quien tiene mayor conocimiento de todos los procesos realizados en la empresa, tal acción parece igualmente de difícil ejecución. Esto guarda relación estrecha con la manera en que la información es manejada por el cliente, así como con la manera en que se realizan habitualmente las tareas en la empresa, por lo que no es claro que se pudiera haber hecho una planificación acorde para evitar la incidencia del riesgo en los tiempos del proyecto, aún de haberse tenido en cuenta esta posibilidad particular.
\end{itemize}

\section{En las estimaciones de tiempo}
\label{en_las_estimaciones_de_tiempo}
A continuación, se describen las discrepancias surgidas durante la ejecución del proyecto respecto de las distintas etapas previstas inicialmente (estas tienen relación con la planificación de las \hyperlink{US}{US} propuestas al inicio, las cuales pueden verse en el \hyperlink{apendice_e}{Apéndice E}):\\
\noindent\underline{Preparación Inicial:}
\begin{itemize}
	\item Fechas teóricas desde el 01/07/2024 al 07/07/2024.
	\item Fechas reales desde el 01/07/2024 al 11/07/2024.\\\\
	Algo a remarcar, y que influyó en el flujo esperado de trabajo del equipo, fue el incidente sufrido en el servidor de la base de datos, ocurrido al trabajar inicialmente sin la implementación de SSL sobre las conexiones. Esto dio la posibilidad de un ataque por el cual se debió reinstalar el DBMS PostgreSQL, esta vez implementando SSL. Por lo tanto el inicio del desarrollo debió ser retrasado hasta el día viernes 12/07/2024 (en tanto que en la planificación inicial se había indicado que la fecha de inicio sería el día lunes 08/07/2024).
\end{itemize}
\noindent\underline{Iteración I:}
\begin{itemize}
	\item Fechas teóricas desde el 08/07/2024 al 28/07/2024.
	\item Fechas reales desde el 12/07/2024 al 28/07/2024.\\\\
	Al margen de lo ya mencionado (el inicio tardío de la iteración por lo acontecido en la semana de preparación inicial), en este caso no se observó algo anormal en la ejecución de la iteración. Si bien el inicio se dio después de lo esperado, la iteración se finalizó de la forma planificada.\\
	La correspondiente reunión con el cliente, que tuvo el objetivo de hacer la revisión por parte de este último de lo desarrollado, se llevó a cabo el día 31/07/2024.
\end{itemize}
\underline{Iteración II:}
\begin{itemize}
	\item Fechas teóricas desde el 29/07/2024 al 18/08/2024.
	\item Fechas reales desde el 05/08/2024 al 25/08/2024.\\\\
	No se observó algo anormal en la ejecución de la iteración.\\
	La reunión con el cliente se llevó a cabo el día 29/08/2024.
\end{itemize}
\underline{Iteración III:}
\begin{itemize}
	\item Fechas teóricas desde el 19/08/2024 al 08/09/2024.
	\item Fechas reales desde el 02/09/2024 al 22/09/2024.\\\\
	No se observó algo anormal en la ejecución de la iteración.\\
	La reunión con el cliente se llevó a cabo el día 25/09/2024.
\end{itemize}
\underline{Iteración IV:}
\begin{itemize}
	\item Fechas teóricas desde el 09/09/2024 al 22/09/2024.
	\item Fechas reales desde el 15/11/2024 al 28/12/2024.\\\\
	Desde este punto, la discrepancia entre los tiempos de ejecución teóricos y prácticos de las iteraciones se hace notoria (lo que contribuyó en el desplazamiento del inicio de las iteraciones posteriores en mayor medida incluso que la semana intermedia dejada entre iteraciones). Se debió a que ambos miembros del equipo de desarrollo se encontraban prestando servicio en la misma institución (situación que se mantuvo durante todo el proyecto), donde, a su vez, estaban vinculados con otro proyecto de software que anteriormente había sido dejado en segundo plano con el fin de priorizar el presente. En adición, hacia fines del año 2024 tal proyecto requirió aún más tiempo de trabajo, debido a su nivel de avance.\\
	Cabe mencionar que, si bien el tiempo de ejecución de la iteración aumentó notablemente, no fue ese a su vez el caso del tiempo de trabajo (que se corresponde con el esfuerzo empleado sobre la iteración y es tratado en la sección \ref{comparacion_estimaciones_de_esfuerzo}), que se mantuvo invariante aún si fue distribuido sobre el tiempo de la iteración de manera inusualmente heterogénea.\\
	Debido a la falta de disponibilidad del gerente general durante el mes de enero, en este caso la reunión no pudo realizarse en una fecha cercana a la de finalización de la iteración. La reunión con el cliente se llevó a cabo el día 13/02/2025.
\end{itemize}
\underline{Iteración V:}
\begin{itemize}
	\item Inicialmente no planificada.
	\item Fechas reales desde el 17/02/2025 al 01/03/2025.\\\\
	No se observó algo anormal en la ejecución de la iteración.\\
	La reunión con el cliente se llevó a cabo el día 06/03/2025.
\end{itemize}
\indent En tanto hubieron diferencias entre los tiempos previstos inicialmente y los reales, también hay variaciones entre la diagramación resultante de la etapa de planificación inicial y la observada en la práctica. En las figuras \ref{diagrama de gantt 1} y \ref{diagrama de gantt 2} se muestran respectivamente los cronogramas correspondientes a estos dos casos.

\begin{figure}[H]
	\centering
	{%
		\setlength{\fboxsep}{0pt}%
		\setlength{\fboxrule}{0.5pt}%
		\fbox{\includegraphics[scale = 0.6]{Diagrama_Gantt_1.png}}%
	}%
	\caption{Diagrama de Gantt del proyecto propuesto inicialmente.}
	\label{diagrama de gantt 1}
\end{figure}

\begin{figure}[H]
	\centering
	{%
		\setlength{\fboxsep}{0pt}%
		\setlength{\fboxrule}{0.5pt}%
		\fbox{\includegraphics[scale = 0.6]{Diagrama_Gantt_2.png}}%
	}%
	\caption{Diagrama de Gantt del proyecto obtenido empíricamente.}
	\label{diagrama de gantt 2}
\end{figure}

La fecha de inicio, que se corresponde con la fecha de comienzo de la preparación inicial fue respetada (01/07/2024), por lo que la discrepancia en las estimaciones temporales puede analizarse solo observando la diferencia en los tiempos de finalización de las actividades por medio de las cuales se consideró en ambos casos que se podrían cumplir con los objetivos del proyecto (\hyperlink{apendice_c}{Apéndice C}). Tales fechas fueron el 22/09/2024, fecha teórica, y el 25/04/2025, fecha práctica. Debe considerarse, sin embargo, que en la teoría se tuvieron en cuenta solo 4 iteraciones de las 5 que terminarían siendo, esto de acuerdo a las estimaciones iniciales y las \hyperlink{US}{US} consideradas en tal caso. Es por esto último que una correcta comparación debería realizarse añadiendo una iteración más al cronograma teórico, manteniendo la consistencia con los criterios de ese momento. Cabe aclarar que no hay forma de realizar analogías entre las distintas \hyperlink{US}{US} consideradas para la cuarta iteración teórica y la quinta práctica (las últimas iteraciones de los escenarios comparados), pero en ambos casos la duración fue de alrededor de dos semanas, por lo que se cree que simplemente agregar una iteración completa de tres semanas de trabajo al cronograma teórico no perjudicaría sustancialmente el análisis. De este modo, las fechas a comparar serían el 13/10/2024 y el 25/04/2025.\\
\indent La duración teórica es entonces de 105 días, mientras que la práctica de 299 días. Por lo tanto el error relativo porcentual de la estimación fue de 184,76\% aproximadamente.

\section{En las estimaciones de esfuerzo}
\label{comparacion_estimaciones_de_esfuerzo}
En las secciones \ref{descripcion_iteracion_I}, \ref{descripcion_iteracion_II}, \ref{descripcion_iteracion_III}, \ref{descripcion_iteracion_IV} y \ref{descripcion_iteracion_V} se describieron las ejecuciones de las distintas iteraciones. A su vez, se hizo en cada caso un análisis de las diferencias entre los tiempos estimados y los reales de haberse tomado como tiempo empleado por \hyperlink{SP}{SP} el nuevo calculado de acuerdo a lo obtenido en la práctica para la \hyperlink{US}{US} dada (ver tablas \ref{tabla_dif_horas_estim_iter_1}, \ref{tabla_dif_horas_estim_iter_2}, \ref{tabla_dif_horas_estim_iter_3}, \ref{tabla_dif_horas_estim_iter_4} y \ref{tabla_dif_horas_estim_iter_5}). Esto tuvo como objetivo eliminar los errores de asignación de velocidad para el equipo, el cual se iría adecuando de manera práctica a través de las distintas iteraciones (convergiendo en el mejor de los casos a una velocidad adecuada para todo el proyecto, siempre que se cumpla la premisa de tener condiciones no demasiado disímiles entre iteraciones). Se evidenció que, en general, se tendió a subestimar el esfuerzo, ya que la suma de desviaciones (considerando signo) de tiempo medidos en cantidad de horas fueron de 0hs; 8,5hs; 2hs; 10,5hs; 0hs. En tanto las nuevas velocidades de las iteraciones fueron de 6hs, 8hs, 8hs, 8hs, 8hs, es posible notar que en las iteraciones II y IV hubo una subestimación de al menos un \hyperlink{SP}{SP}.\\
En la tabla \ref{tabla_estadisticas_desviaciones} se muestran las desviaciones de cada iteración y los correspondientes errores relativos (respecto del total de horas estimadas). Puede notarse que el valor máximo de error es inferior al 10\%, en tanto que el error relativo promedio es de alrededor del 3,65\%. Estos valores son ciertamente aceptables, más aún teniendo en cuenta la falta de experiencia en estimaciones similares por parte del equipo de desarrollo.

\begin{table*}[h!]
	\centering
	\captionsetup{justification=centering,margin=1.5cm}
	\begin{tabular}{ |c|c|c|c| }
		\hline
		\textbf{Iteración} & \textbf{Estimado[hs]} & \textbf{Desviación[hs]} & \textbf{Error \%} \\
		\hline
		\textbf{1} & 78 & 0 & 0,00\% \\
		\hline
		\textbf{2} & 120 & 8,5 & 7,08\% \\
		\hline
		\textbf{3} & 112 & 2 & 1,79\% \\
		\hline
		\textbf{4} & 112 & 10,5 & 9,38\% \\
		\hline
		\textbf{5} & 64 & 0 & 0,00\% \\
		\hline
		\multicolumn{4}{|c|}{\textbf{Promedio}} \\
		\hline
		\multicolumn{1}{|c|}{-} & 97,2 & 4,2 & 3,65\% \\
		\hline
	\end{tabular}
	\caption{Valores de desviaciones y errores relativos por iteración.}
	\label{tabla_estadisticas_desviaciones}
\end{table*}

\section{En el control de calidad}
En el capítulo \ref{Control de calidad} se dijo que las pruebas \begin{math}\alpha\end{math} no serían suficientes para asegurar la calidad del producto desarrollado, así como las pruebas unitarias realizadas durante la ejecución de las iteraciones. Por lo que se debió optar por adicionar pruebas unitarias y de integración en etapas posteriores a las correspondientes a las iteraciones realizadas.

\chapter{Evaluación de la calidad del proyecto}

El PMI\textsuperscript{\tiny\textregistered} (Project Management Institute) ha definido dos condiciones que se deben cumplir para asegurar que un proyecto sea de calidad\footnote{Moreno Monsalve, Nelson - Sánchez Ayala, Luz - Velosa García, José. \textit{<<Introducción a la Gerencia de Proyectos, Conceptos y Aplicaciones>>}. Ediciones EAN, 2018. Pág. 95.}:
\begin{enumerate}
\item El tiempo previsto así como los recursos asignados y los costos presupuestados deben ser cumplidos.
\item El resultado del proyecto debe concordar con las especificaciones que han definido los stakeholders.
\end{enumerate}

En base a estas dos condiciones, se tienen dos resultados disímiles:
\begin{enumerate}
	\item El proyecto aporta experiencia así como un beneficio académico para el equipo de desarrollo. La empresa interesada no tuvo más que costos de tiempos relacionados con disponibilidad para la atención de consultas que tuvieran como fin la definición de requerimientos para el sistema, así como la validación de lo desarrollado. Como se mencionó, ese esfuerzo estuvo muy debajo del esperado, ya que las consultas hacia el cliente no fueron tan habituales como lo indica la teoría. Por otro lado, los tiempos planificados inicialmente fueron incumplidos en gran medida. El plan de gestión de riesgos no fue suficiente para contrarrestar los inconvenientes que ocasionaron los mayores retrasos en los tiempos de ejecución del proyecto, y todo indica que se debió prever una menor disponibilidad por parte de los desarrolladores, cuyo tiempo de trabajo individual teórico fue de 20 horas semanales, no cumpliéndose con esto en todas las iteraciones. Aún así, de haberse ajustado la velocidad del equipo para estos casos, el resultado no podría haber variado significativamente, ya que hubieron momentos donde la velocidad debió tender a cero de haberse respetado el tiempo de iteración de tres semanas.\\
	Los tiempos estimados fueron poco realistas al no dejar espacio entre iteraciones para planificar la siguiente, considerando que la hipótesis de una retroalimentación rápida y constante con el cliente es difícil, a menos que el compromiso de todos los actores clave del proyecto sea muy alto.\\
	Luego de la adición de una iteración al plan teórico inicial, que no había sido en ese punto tenida en cuenta, y que fue necesaria posteriormente, se obtuvo un error relativo porcentual del 184,76\% entre el plan empírico y el teórico.\\
	Podemos asegurar entonces que este punto no se cumplió de manera satisfactoria en absoluto.
	\item El cliente se mostró conforme con el resultado. Sin embargo, teniendo en cuenta las diferencias en los tiempos de entrega mencionadas en el punto 1, es probable que su conformidad haya estado influida por el hecho de que la realización del proyecto no implicó ningún costo económico para él. De cualquier modo, se considera que los requerimientos (tanto iniciales como los surgidos durante el desarrollo) del cliente fueron abordados de manera satisfactoria y de acuerdo a sus indicaciones. Por otro lado, como fuera mencionado en la sección \ref{comparacion_metodologia_implementada}, teniendo en cuenta la libertad que el cliente terminaría brindando, la posibilidad de realizar modificaciones durante el desarrollo no fue aprovechada en gran medida, ya que no resultó necesario.
\end{enumerate}

Solo es posible concluir que el proyecto fue de baja calidad. Sin embargo, es importante aclarar que con esto no debe inferirse necesariamente que el producto de software entregado al cliente haya sido de baja calidad, lo cual, en todo caso, tiene más relación con el punto 2.

\chapter{Conclusiones}
\indent En el presente informe se detallaron las actividades llevadas a cabo para el diseño e implementación de un sistema destinado al control del inventario y de la producción para Imprenta Lux S.A.\\
\indent De modo de abordar la situación problemática a tratar mediante la implementación del sistema de información, en el cual se centra el documento, se hizo una descripción tanto de la estructura física como jerárquica de la empresa. A su vez, para una mayor contextualización, se analizaron los procesos de producción de interés para el proyecto. En base a esto fueron descritos los problemas que tienen lugar en la organización relativos a los sectores implicados. Tales problemas, en mayor medida fueron identificados por tener relación con la falta de centralización de la información, lo que posibilitaría un acceso rápido para la planificación de proyectos de producción. De este modo, se planteó una solución que se basara en dicha centralización, que permitiera al gerente general, encargado de la planificación de los proyectos y distribución de los recursos de la empresa, poder abordar los requerimientos del cliente de manera más eficiente.\\
\indent Debido a las características del proyecto, se creyó conveniente emplear una metodología con características ágiles, en tanto se priorizó la flexibilidad en el desarrollo debido a probables requerimientos cambiantes, ya que inicialmente estos no estaban absolutamente definidos. Sin embargo, teniendo en cuenta la libertad ofrecida finalmente por el cliente, la posibilidad de realizar modificaciones no fue aprovechada como se esperaba. Por lo tanto, se piensa que una metodología clásica, basada en un análisis inicial más detallado en cuanto a los requisitos, hubiera sido igualmente (si no es que tal vez más) útil.\\
\indent Cuatro iteraciones teóricas de desarrollo fueron previstas de manera inicial, comenzando el 01/07/2024 y finalizando el 22/09/2024, las cuales de acuerdo a la teoría de las metodologías ágiles podrían variar en número en base a posibles modificaciones en los requerimientos (por adición de \hyperlink{US}{US}). Los tiempos teóricos de desarrollo mostraron ser altamente discordantes con los prácticos, teniendo principal incidencia la subestimación de tiempos necesarios para la comunicación con el cliente, así como del esfuerzo necesario para abordar proyectos profesionales independientes del presente proyecto por parte de los miembros del equipo de desarrollo. En la práctica, el desarrollo tuvo lugar entre el 01/07/2024 y el 01/03/2025, con cinco iteraciones. En un posterior análisis, se obtuvo un valor del 184,76\% de error relativo porcentual para el caso real respecto del tiempo de desarrollo teórico (añadiéndosele a este último una iteración más correspondiente a la no planificada inicialmente), lo cual evidenció una baja calidad del proyecto de acuerdo a las definiciones del PMI\textsuperscript{\tiny\textregistered} (Project Management Institute). A pesar de esto, se considera que las especificaciones de los stakeholders han sido concordantes con las características del producto de software resultante.\\
\indent Actualmente (23/05/2025), el cliente no se encuentra utilizando la totalidad del sistema, sino solo la parte destinada a la gestión de proyectos. Los relevamientos iniciales sobre la disponibilidad de insumos en los depósitos constituyen un esfuerzo inicial considerable, al igual que el posterior control regular de inventarios. Sin embargo, ambos procesos resultan fundamentales para aprovechar plenamente las funcionalidades del sistema implementado. En cualquier caso, el aprovechamiento completo del sistema dependerá de cuán relevante considere la empresa la solución que este propone a los problemas detectados para considerar justificado este esfuerzo.\\
\indent En cuanto al valor aportado por el proyecto al desarrollo profesional de quienes elaboran este documento, se entiende que fue importante, especialmente por el uso de herramientas poco habituales para ambos. Tanto el trabajo con un lenguaje de programación no empleado de forma regular, como el desarrollo de una aplicación móvil además de la de escritorio, y la coordinación del equipo así como la interacción con el cliente, se espera que constituyan una referencia útil para futuros proyectos. Si bien no se preveía que las estimaciones realizadas fueran del todo acertadas debido a la falta de experiencia, sí hubo una buena concordancia si es que solo se toma en cuenta lo trabajado y no los intervalos ociosos dentro y entre iteraciones. A su vez, y a pesar de lo anterior, también parece de relevancia en futuros proyectos no subestimar la importancia de la prueba constante durante el desarrollo, que viene de un intento inconsciente por aumentar la velocidad del equipo (percibir una mayor cantidad de points finalizados por iteración).

%%% quitamos el número del capítulo de la cabecera %%%
\renewcommand{\sectionmark}[1]{\markright{\slshape\nouppercase{#1}}}

%%% quitamos el número del capítulo de la cabecera %%%


\chapter*{\hypertarget{apendice_a}{}Apéndice A\\Refinamiento de las \hyperlink{US}{US}}
\markboth{Apéndice A - Refinamiento de las US}{Apéndice A - Refinamiento de las US}
\sectionmark{Apéndice A - Refinamiento de las US}
\addcontentsline{toc}{chapter}{A. Refinamiento de las US}

%%%%%%%%%%%%%%%%%%%%%%%%%%%%%%%%%%%%%%%%%
\newcounter{counter_tbl_A}  \setcounter{counter_tbl_A}{1}
%%%%%%%%%%%%%%%%%%%%%%%%%%%%%%%%%%%%%%%%%

\hypertarget{apendice_a_I}{}
\section*{Iteración I}
\subsection*{Inicio de sesión}
Los usuarios deben identificarse para acceder al sistema y poder realizar cualquier acción sobre el mismo.

% custom commands
\newcolumntype{L}[1]{>{\raggedright\let\newline\\\arraybackslash\hspace{0pt}}p{#1}}
\newcolumntype{C}[1]{>{\centering\let\newline\\\arraybackslash\hspace{0pt}}p{#1}}
\newcolumntype{R}[1]{>{\raggedleft\let\newline\\\arraybackslash\hspace{0pt}}p{#1}}

%%%%%%%%%%%%%%%
% this line fixes the vertical padding of text inside the cells
\renewcommand{\arraystretch}{1.4}

\begin{table*}[h!]
	\centering
	\begin{tabularx}
		{\textwidth}{|L{6cm}|L{8cm}|}
		\hline 
		\textbf{Número:} 1 & \textbf{Nombre:} Inicio de sesión \\ \hline
		\multicolumn{2}{| X |}{\textbf{Usuario:} Todos. }\\ \hline
		\textbf{Iteración Asignada:} 1 & \textbf{Puntos Estimados:} 3 \\ \hline
		\textbf{Prioridad en Negocio:} Baja & \multirow{2}{*}{}{\textbf{Puntos Reales:} 2 }\\
		\cline{1-1} \textbf{Riesgo en Desarrollo:} Bajo & \\ \hline
		\multicolumn{2}{| X |}{\textbf{Descripción:} }\\
		
		\hline
		\multicolumn{2}{| X |}{El usuario puede iniciar sesión en el sistema ingresando sus credenciales (nombre de usuario y contraseña).\newline
			El usuario puede finalizar su sesión una vez que ingrese al sistema.}\\
		
		\hline
		\multicolumn{2}{| X |}
		{\textbf{Observaciones:} - }\\
		
		\hline
	\end{tabularx}
	\caption*{Tabla A.\arabic{counter_tbl_A}: Refinamiento de \hyperlink{US}{US} 1.}
	\addcontentsline{lot}{table}{A.\arabic{counter_tbl_A}. Refinamiento de US 1.}\stepcounter{counter_tbl_A}
\end{table*}
%%%%%%%%%%%%%%%


\pagebreak

\subsection*{Administrar usuarios}
Solo el gerente general debe poder agregar nuevos usuarios al sistema. Los usuarios del sistema serán únicamente los empleados de la empresa, por lo que no se espera que estos varíen de manera frecuente. Además, el gerente debe poder editar la información de los usuarios  si así lo desea.

%%%%%%%%%%%%%%%
% this line fixes the vertical padding of text inside the cells
\renewcommand{\arraystretch}{1.4}

\begin{table*}[h!]
	\centering
	\begin{tabularx}
		{\textwidth}{|L{6cm}|L{8cm}|}
		\hline 
		\textbf{Número:} 2 & \textbf{Nombre:} Administrar usuarios \\ \hline
		\multicolumn{2}{| X |}{\textbf{Usuario:} Gerente general. }\\ \hline
		\textbf{Iteración Asignada:} 1 & \textbf{Puntos Estimados:} 2 \\ \hline
		\textbf{Prioridad en Negocio:} Baja & \multirow{2}{*}{}{\textbf{Puntos Reales:} 1,75 }\\
		\cline{1-1} \textbf{Riesgo en Desarrollo:} Bajo & \\ \hline
		\multicolumn{2}{| X |}{\textbf{Descripción:} }\\
		
		\hline
		\multicolumn{2}{| X |}{El gerente general puede agregar nuevos usuarios al sistema.\newline
			El gerente general puede editar la informaciín de los usuarios existentes.\newline
			Al agregar un nuevo usuario se debe indicar la información de acceso del usuario: nombre y contraseña. También se debe indicar el perfil de usuario del que se trata, el cual puede ser: gerente general, encargado de mantenimiento, encargado de reposición, encargado de producción, empleado de producción.\newline
			El gerente general puede indicar el nombre y apellido del empleado asociado al usuario.}\\
		
		\hline
		\multicolumn{2}{| X |}
		{\textbf{Observaciones:} - }\\
		
		\hline
	\end{tabularx}
	\caption*{Tabla A.\arabic{counter_tbl_A}: Refinamiento de \hyperlink{US}{US} 2.}
	\addcontentsline{lot}{table}{A.\arabic{counter_tbl_A}. Refinamiento de US 2.}\stepcounter{counter_tbl_A}
\end{table*}
%%%%%%%%%%%%%%%


\pagebreak

\subsection*{Administrar de\smash{p}ósitos}
La empresa no está limitada a posibles futuras ampliaciones físicas o nuevas disposiciones de gestión, por lo que el encargado de mantenimiento debe poder agregar nuevos depósitos.\\
\indent Se debe tener en consideración que la empresa cuenta con varios depósitos, los cuales pueden incluso estar ubicados en una misma sucursal. Por ello es que el encargado de mantenimiento también debe poder designar los distintos depósitos existentes para poder diferenciarlos.

%%%%%%%%%%%%%%%
% this line fixes the vertical padding of text inside the cells
\renewcommand{\arraystretch}{1.4}

\begin{table*}[h!]
	\centering
	\begin{tabularx}
		{\textwidth}{|L{6cm}|L{8cm}|}
		\hline 
		\textbf{Número:} 6 & \textbf{Nombre:} Administrar depósitos \\ \hline
		\multicolumn{2}{| X |}{\textbf{Usuario:} Gerente general, Encargado de mantenimiento. }\\ \hline
		\textbf{Iteración Asignada:} 1 & \textbf{Puntos Estimados:} 3 \\ \hline
		\textbf{Prioridad en Negocio:} Baja & \multirow{2}{*}{}{\textbf{Puntos Reales:} 2,25 }\\
		\cline{1-1} \textbf{Riesgo en Desarrollo:} Bajo & \\ \hline
		\multicolumn{2}{| X |}{\textbf{Descripción:} }\\
		
		\hline
		\multicolumn{2}{| X |}{El encargado de mantenimiento puede agregar un nuevo depósito.\newline
			El encargado de mantenimiento puede editar un depósito existente en el sistema.\newline
			Los depósitos cuentan con los siguientes atributos:
			\setlist{nolistsep}
			\begin{itemize}[noitemsep]
				\item Nombre: designación corta para diferenciar al depósito.
				\item Descripción: designación larga que puede servir para diferenciar más claramente al depósito y evitar confusiones.
			\end{itemize}
			El campo <<Descripción>> es opcional y, por lo tanto, puede ser omitido.\newline
			Cada depósito está asociado a una y solo una sucursal.}\\
		
		\hline
		\multicolumn{2}{| X |}
		{\textbf{Observaciones:} - }\\
		
		\hline
	\end{tabularx}
	\caption*{Tabla A.\arabic{counter_tbl_A}: Refinamiento de \hyperlink{US}{US} 6.}
	\addcontentsline{lot}{table}{A.\arabic{counter_tbl_A}. Refinamiento de US 6.}\stepcounter{counter_tbl_A}
\end{table*}
%%%%%%%%%%%%%%%


\pagebreak

\subsection*{Administrar sucursales}
El gerente general debe poder añadir nuevos establecimientos o sucursales de la empresa, ya que es posible que en el futuro la disposición geográfica de la empresa cambie.

%%%%%%%%%%%%%%%
% this line fixes the vertical padding of text inside the cells
\renewcommand{\arraystretch}{1.4}

\begin{table*}[h!]
	\centering
	\begin{tabularx}
		{\textwidth}{|L{6cm}|L{8cm}|}
		\hline 
		\textbf{Número:} 7 & \textbf{Nombre:} Administrar sucursales \\ \hline
		\multicolumn{2}{| X |}{\textbf{Usuario:} Gerente general, Encargado de mantenimiento. }\\ \hline
		\textbf{Iteración Asignada:} 1 & \textbf{Puntos Estimados:} 3 \\ \hline
		\textbf{Prioridad en Negocio:} Baja & \multirow{2}{*}{}{\textbf{Puntos Reales:} 1,75 }\\
		\cline{1-1} \textbf{Riesgo en Desarrollo:} Bajo & \\ \hline
		\multicolumn{2}{| X |}{\textbf{Descripción:} }\\
		
		\hline
		\multicolumn{2}{| X |}{El gerente general puede añadir nuevas sucursales.\newline
			El gerente general puede editar las sucursales existentes.\newline
			Las sucursales constan del siguiente atributo:
			\setlist{nolistsep}
			\begin{itemize}[noitemsep]
				\item Nombre: designación corta para diferenciar la sucursal.
				\item Dirección: es la ubicación de la sucursal (calle, altura y ciudad si fuese necesario). 
		\end{itemize}}\\
		
		\hline
		\multicolumn{2}{| X |}
		{\textbf{Observaciones:} - }\\
		
		\hline
	\end{tabularx}
	\caption*{Tabla A.\arabic{counter_tbl_A}: Refinamiento de \hyperlink{US}{US} 7.}
	\addcontentsline{lot}{table}{A.\arabic{counter_tbl_A}. Refinamiento de US 7.}\stepcounter{counter_tbl_A}
\end{table*}
%%%%%%%%%%%%%%%


\pagebreak

\subsection*{Administrar clientes}
El gerente general debe poder llevar registro de los clientes de la empresa y sus datos de contacto.

%%%%%%%%%%%%%%%
% this line fixes the vertical padding of text inside the cells
\renewcommand{\arraystretch}{1.4}

\begin{table*}[h!]
	\centering
	\begin{tabularx}
		{\textwidth}{|L{6cm}|L{8cm}|}
		\hline 
		\textbf{Número:} 18 & \textbf{Nombre:} Administrar clientes \\ \hline
		\multicolumn{2}{| X |}{\textbf{Usuario:} Gerente general. }\\ \hline
		\textbf{Iteración Asignada:} 1 & \textbf{Puntos Estimados:} 2 \\ \hline
		\textbf{Prioridad en Negocio:} Baja & \multirow{2}{*}{}{\textbf{Puntos Reales:} 2 }\\
		\cline{1-1} \textbf{Riesgo en Desarrollo:} Bajo & \\ \hline
		\multicolumn{2}{| X |}{\textbf{Descripción:} }\\
		
		\hline
		\multicolumn{2}{| X |}{El gerente general puede agregar nuevos clientes al sistema.\newline
			El gerente general puede editar la información de los clientes existentes.\newline
			Los clientes pueden contar con los siguientes atributos de contacto:
			\setlist{nolistsep}
			\begin{itemize}[noitemsep]
				\item Nombre.
				\item Dirección.
				\item Teléfono.
				\item Email.
			\end{itemize}
			No se pueden guardar clientes sin nombre.}\\
		
		\hline
		\multicolumn{2}{| X |}
		{\textbf{Observaciones:} - }\\
		
		\hline
	\end{tabularx}
	\caption*{Tabla A.\arabic{counter_tbl_A}: Refinamiento de \hyperlink{US}{US} 18.}
	\addcontentsline{lot}{table}{A.\arabic{counter_tbl_A}. Refinamiento de US 18.}\stepcounter{counter_tbl_A}
\end{table*}
%%%%%%%%%%%%%%%


\pagebreak

\hypertarget{apendice_a_II}{}
\section*{Iteración II}
\subsection*{Administrar insumo}
El encargado de mantenimiento debe poder registrar o modificar un insumo de características particulares. En caso de alcanzarse el valor de punto de pedido del insumo se debe notificar al encargado de mantenimiento.

% custom commands
\newcolumntype{L}[1]{>{\raggedright\let\newline\\\arraybackslash\hspace{0pt}}p{#1}}
\newcolumntype{C}[1]{>{\centering\let\newline\\\arraybackslash\hspace{0pt}}p{#1}}
\newcolumntype{R}[1]{>{\raggedleft\let\newline\\\arraybackslash\hspace{0pt}}p{#1}}

%%%%%%%%%%%%%%%
% this line fixes the vertical padding of text inside the cells
\renewcommand{\arraystretch}{1.4}

\begin{table*}[h!]
	\centering
	\begin{tabularx}
		{\textwidth}{|L{6cm}|L{8cm}|}
		\hline 
		\textbf{Número:} 3 & \textbf{Nombre:} Administrar insumo \\ \hline
		\multicolumn{2}{| X |}{\textbf{Usuario:} Gerente general, Encargado de mantenimiento. }\\ \hline
		\textbf{Iteración Asignada:} 2 & \textbf{Puntos Estimados:} 3 \\ \hline
		\textbf{Prioridad en Negocio:} Baja & \multirow{2}{*}{}{\textbf{Puntos Reales:} 2,75 }\\
		\cline{1-1} \textbf{Riesgo en Desarrollo:} Bajo & \\ \hline
		\multicolumn{2}{| X |}{\textbf{Descripción:} }\\
		
		\hline
		\multicolumn{2}{| X |}{\begin{small}El encargado de mantenimiento puede registrar un nuevo insumo.\newline
				El encargado de mantenimiento puede modificar los insumos previamente registrados en el sistema.\newline
				El insumo representa un tipo particular de distribución de un producto de mercado. Ejemplos: papel obra de 80g 72cmx102cm Boreal, una tinta Cyan Process de marca Brancher de 1 Kg.\newline
				El insumo a administrar consta de las siguientes características identificativas:
				\setlist{nolistsep}
				\begin{itemize}[noitemsep]
					\item Tipo de insumo (indica el tipo de producto de mercado genérico al que pertenece el insumo).
					\item Descripción (modo de identificar, ya sea por característica distintiva, marca, etc.).
					\item Unidad (unidad de medida adoptada para determinar la cantidad del insumo).
					\item Medida (cantidad de insumo medida en unidades \textit{Unidad}).
					\item Cliente propietario (indica si el insumo es propiedad de Imprenta Lux S.A. o de uno de sus clientes).
					\item Punto de pedido (es la cantidad mínima de insumo con que debe contar la empresa para cumplir con sus trabajos regulares hasta reponer stock).
				\end{itemize}
				La unidad de un insumo debe coincidir con la unidad de su tipo de insumo asociado.
		\end{small}}\\
		
		\hline
		\multicolumn{2}{| X |}
		{\textbf{Observaciones:} - }\\
		
		\hline
	\end{tabularx}
	\caption*{Tabla A.\arabic{counter_tbl_A}: Refinamiento de \hyperlink{US}{US} 3.}
	\addcontentsline{lot}{table}{A.\arabic{counter_tbl_A}. Refinamiento de US 3.}\stepcounter{counter_tbl_A}
\end{table*}
%%%%%%%%%%%%%%%


\pagebreak

\subsection*{Administrar tipo de insumo}
El encargado de mantenimiento debe poder buscar insumos que sean equivalentes o alternativos en caso de faltante imprevisto de un insumo particular (ya sea por demoras en la reposición o la imposibilidad de conseguir stock en el mercado). En caso de alcanzarse el valor de punto de pedido del tipo de insumo se debe notificar al encargado de mantenimiento.

%%%%%%%%%%%%%%%
% this line fixes the vertical padding of text inside the cells
\renewcommand{\arraystretch}{1.4}

\begin{table*}[h!]
	\centering
	\begin{tabularx}
		{\textwidth}{|L{6cm}|L{8cm}|}
		\hline 
		\textbf{Número:} 4 & \textbf{Nombre:} Administrar tipo de insumo \\ \hline
		\multicolumn{2}{| X |}{\textbf{Usuario:} Gerente general, Encargado de mantenimiento.}\\ \hline
		\textbf{Iteración Asignada:} 2 & \textbf{Puntos Estimados:} 3 \\ \hline
		\textbf{Prioridad en Negocio:} Baja & \multirow{2}{*}{}{\textbf{Puntos Reales:} 2,5 }\\
		\cline{1-1} \textbf{Riesgo en Desarrollo:} Bajo & \\ \hline
		\multicolumn{2}{| X |}{\textbf{Descripción:} }\\
		
		\hline
		\multicolumn{2}{| X |}{El encargado de mantenimiento puede registrar tipos de insumos, los cuales sirven para agrupar insumos equivalentes.\newline
			El encargado de mantenimiento puede editar las características de los tipos de insumos.\newline
			El tipo de insumo representa un producto de mercado genérico independientemente de la marca y las características propias de la distribución (unidades y medidas). Ejemplos: <<Papel obra de 80g>> (podría variar las dimensiones del papel), <<Tinta Cyan Process>> (podría variar en su peso de distribución).\newline
			El tipo de insumo a administrar consta de las características identificativas:
			\setlist{nolistsep}
			\begin{itemize}[noitemsep]
				\item Nombre (designación usual con la cual se conoce al tipo de insumo).
				\item Unidad (unidad de medida adoptada para determinar cantidades relacionadas a un tipo de insumo).
				\item Punto de pedido (es la cantidad mínima de tipo de insumo con que debe contar la empresa para cumplir con sus trabajos regulares hasta reponer stock).
			\end{itemize}
		}\\
		
		\hline
		\multicolumn{2}{| X |}
		{\textbf{Observaciones:} - }\\
		
		\hline
	\end{tabularx}
	\caption*{Tabla A.\arabic{counter_tbl_A}: Refinamiento de \hyperlink{US}{US} 4.}
	\addcontentsline{lot}{table}{A.\arabic{counter_tbl_A}. Refinamiento de US 4.}\stepcounter{counter_tbl_A}
\end{table*}
%%%%%%%%%%%%%%%


\pagebreak

\subsection*{Modificar cantidad de insumo}
Se deberá poder llevar registro de los insumos disponibles en cada depósito. El encargado de reposición deberá registrar el ingreso y egreso de insumos en los distintos depósitos de la empresa de manera periódica a medida que estos sean solicitados para cumplir con los distintos trabajos. Por su parte, el encargado de mantenimiento debe poder ajustar los valores reales de los distintos insumos en caso de notar discrepancias durante algún control de inventario.

%%%%%%%%%%%%%%%
% this line fixes the vertical padding of text inside the cells
\renewcommand{\arraystretch}{1.4}

\begin{table*}[h!]
	\centering
	\begin{tabularx}
		{\textwidth}{|L{6cm}|L{8cm}|}
		\hline 
		\textbf{Número:} 5 & \textbf{Nombre:} Modificar cantidad de insumo \\ \hline
		\multicolumn{2}{| X |}{\textbf{Usuario:} Gerente general, Encargado de mantenimiento, Encargado de reposición.}\\ \hline
		\textbf{Iteración Asignada:} 2 & \textbf{Puntos Estimados:} 3 \\ \hline
		\textbf{Prioridad en Negocio:} Media & \multirow{2}{*}{}{\textbf{Puntos Reales:} 4,67 }\\
		\cline{1-1} \textbf{Riesgo en Desarrollo:} Medio & \\ \hline
		\multicolumn{2}{| X |}{\textbf{Descripción:} }\\
		
		\hline
		\multicolumn{2}{| X |}{Cada depósito de la empresa tiene asociada una cuenta de los insumos que se tiene a disposición en cada uno de ellos.\newline
		El encargado de mantenimiento y el de reposición pueden modificar la cantidad de los insumos disponibles en los depósitos.\newline
		El encargado de reposición solo puede modificar los valores aumentándolos o decrementándolos (solo establecer cantidades relativas a la disponibilidad previa en los inventarios), mientras que el encargado de mantenimiento puede establecer una cantidad absoluta (independiente del valor previo).\newline
		Al modificarse la cantidad disponible de un insumo se comprueba si se ha alcanzado el PP del insumo o del tipo de insumo. En caso que así sea se notifica al encargado de reposición indicando de qué insumo o tipo de insumo se trata.
		}\\
		
		\hline
		\multicolumn{2}{| X |}
		{\textbf{Observaciones:} - }\\
		
		\hline
	\end{tabularx}
	\caption*{Tabla A.\arabic{counter_tbl_A}: Refinamiento de \hyperlink{US}{US} 5.}
	\addcontentsline{lot}{table}{A.\arabic{counter_tbl_A}. Refinamiento de US 5.}\stepcounter{counter_tbl_A}
\end{table*}
%%%%%%%%%%%%%%%


\pagebreak

\subsection*{Administrar tipo de tarea}
El encargado de producción debe poder agregar otros tipos de tareas que no sean los que se consideraron hasta el momento, de manera que las tareas actuales se puedan subdividir o agregar otras completamente nuevas para poder abarcar nuevos productos finales y sus correspondientes procesos de producción.

%%%%%%%%%%%%%%%
% this line fixes the vertical padding of text inside the cells
\renewcommand{\arraystretch}{1.4}

\begin{table*}[h!]
	\centering
	\begin{tabularx}
		{\textwidth}{|L{6cm}|L{8cm}|}
		\hline 
		\textbf{Número:} 23 & \textbf{Nombre:} Administrar tipo de tarea \\ \hline
		\multicolumn{2}{| X |}{\textbf{Usuario:} Gerente general, Encargado de producción.}\\ \hline
		\textbf{Iteración Asignada:} 2 & \textbf{Puntos Estimados:} 3 \\ \hline
		\textbf{Prioridad en Negocio:} Baja & \multirow{2}{*}{}{\textbf{Puntos Reales:} 4,17 }\\
		\cline{1-1} \textbf{Riesgo en Desarrollo:} Medio & \\ \hline
		\multicolumn{2}{| X |}{\textbf{Descripción:} }\\
		
		\hline
		\multicolumn{2}{| X |}{
			El encargado de producción puede agregar nuevos tipos de tareas.\newline
			El encargado de producción puede editar tipos de tareas.\newline
			Un tipo de tarea representa a un conjunto de tareas que sean similares bajo algún criterio (no es una tarea particular que pueda ser completada).\newline
			Cada tipo de tarea puede constar de cero, una o más propiedades que caractericen a las tareas de dicho tipo.\newline
			La modificación de un tipo de tarea no repercute en las tareas de dicho tipo ni en sus propiedades.
		}\\
		
		\hline
		\multicolumn{2}{| X |}
		{\textbf{Observaciones:} - }\\
		
		\hline
	\end{tabularx}
	\caption*{Tabla A.\arabic{counter_tbl_A}: Refinamiento de \hyperlink{US}{US} 23.}
	\addcontentsline{lot}{table}{A.\arabic{counter_tbl_A}. Refinamiento de US 23.}\stepcounter{counter_tbl_A}
\end{table*}
%%%%%%%%%%%%%%%


\pagebreak

\subsection*{Visualizar novedades}
Los usuarios deberán poder recibir y visualizar novedades del sistema tras la ocurrencia de ciertos eventos. Si el usuario está conectado se le debe notificar que tiene una nueva novedad.

%%%%%%%%%%%%%%%
% this line fixes the vertical padding of text inside the cells
\renewcommand{\arraystretch}{1.4}

\begin{table*}[h!]
	\centering
	\begin{tabularx}
		{\textwidth}{|L{6cm}|L{8cm}|}
		\hline 
		\textbf{Número:} 24 & \textbf{Nombre:} Visualizar novedades \\ \hline
		\multicolumn{2}{| X |}{\textbf{Usuario:}  Todos.}\\ \hline
		\textbf{Iteración Asignada:} 2 & \textbf{Puntos Estimados:} 3 \\ \hline
		\textbf{Prioridad en Negocio:} Baja & \multirow{2}{*}{}{\textbf{Puntos Reales:} 4,5 }\\
		\cline{1-1} \textbf{Riesgo en Desarrollo:} Bajo & \\ \hline
		\multicolumn{2}{| X |}{\textbf{Descripción:} }\\
		
		\hline
		\multicolumn{2}{| X |}{
			Cada usuario puede listar las novedades que lo tengan como destinatario.\newline
			Cada novedad tiene un único destinatario.\newline
			Las novedades no visualizadas previamente se muestran primero y se señalan de manera distinta a aquellas que el sistema ya haya mostrado.\newline
			Cada novedad consta de las siguientes características:
			\setlist{nolistsep}
			\begin{itemize}[noitemsep]
				\item Un título (que resuma de qué tipo de novedad se trata).
				\item Un cuerpo (que proporcione información más detallada de la causa de la novedad).
				\item Una fecha de creación (que indique el momento en que se generó la novedad).
			\end{itemize}
			Si el destinatario de una novedad está conectado al sistema en el momento en que esta se genere se le debe notificar al usuario correspondiente que tiene una nueva novedad en un plazo no mayor a 10 minutos desde que se generó dicha novedad.\newline
			Al ingresar al sistema se le muestra al usuario la cantidad de novedades que lo tienen como destinatario y que no ha visualizado previamente.
		}\\
		
		\hline
		\multicolumn{2}{| X |}
		{\textbf{Observaciones:} - }\\
		
		\hline
	\end{tabularx}
	\caption*{Tabla A.\arabic{counter_tbl_A}: Refinamiento de \hyperlink{US}{US} 24.}
	\addcontentsline{lot}{table}{A.\arabic{counter_tbl_A}. Refinamiento de US 24.}\stepcounter{counter_tbl_A}
\end{table*}
%%%%%%%%%%%%%%%


\pagebreak

\hypertarget{apendice_a_III}{}
\section*{Iteración III}
\subsection*{Administrar proyecto}
El encargado de producción debe poder registrar nuevos proyectos, así como también editar proyectos previamente registrados. Estos deben incluir datos administrativos básicos como el nombre o designación del proyecto, el cliente y la fecha límite de entrega del producto.

% custom commands
\newcolumntype{L}[1]{>{\raggedright\let\newline\\\arraybackslash\hspace{0pt}}p{#1}}
\newcolumntype{C}[1]{>{\centering\let\newline\\\arraybackslash\hspace{0pt}}p{#1}}
\newcolumntype{R}[1]{>{\raggedleft\let\newline\\\arraybackslash\hspace{0pt}}p{#1}}

%%%%%%%%%%%%%%%
% this line fixes the vertical padding of text inside the cells
\renewcommand{\arraystretch}{1.4}

\begin{table*}[h!]
	\centering
	\begin{tabularx}
		{\textwidth}{|L{6cm}|L{8cm}|}
		\hline 
		\textbf{Número:} 8 & \textbf{Nombre:} Administrar proyecto \\ \hline
		\multicolumn{2}{| X |}{\textbf{Usuario:} Gerente general, Encargado de producción.}\\ \hline
		\textbf{Iteración Asignada:} 3 & \textbf{Puntos Estimados:} 4 \\ \hline
		\textbf{Prioridad en Negocio:} Alta & \multirow{2}{*}{}{\textbf{Puntos Reales:} 4,25 }\\
		\cline{1-1} \textbf{Riesgo en Desarrollo:} Alto & \\ \hline
		\multicolumn{2}{| X |}{\textbf{Descripción:} }\\
		
		\hline
		\multicolumn{2}{| X |}{El encargado de producción puede agregar un nuevo proyecto.\newline
			El encargado de producción puede modificar un proyecto existente en el sistema.\newline
			Los proyectos constan de los siguientes atributos:
			\setlist{nolistsep}
			\begin{itemize}[noitemsep]
				\item Título: es una descripción corta que identifica al proyecto.
				\item Tipo de proyecto: indica el formato de producto requerido. Puede tratarse de un almanaque, libro, revista o volante/afiche, u otro que se haya agregado.
				\item Fecha de entrega: es la fecha límite de entrega acordada con el cliente.
				\item Fecha de inicio: es la fecha en que se agregó el proyecto al sistema.
				\item Fecha de finalización: es la fecha en la cual se da por finalizado el proyecto.
				\item Estado del proyecto: indica el estado en que se encuentra el proyecto, el cual puede ser iniciado, pendiente, parado, finalizado o cancelado.
			\end{itemize}
			Cada proyecto está relacionado con un y solo un cliente.\newline
			Los proyectos se relacionan a un conjunto de tareas predefinidas en un orden específico de acuerdo al tipo de proyecto del que se trate cada uno.\newline
			Al agregar o modificar un proyecto el encargado de producción puede especificar para cada tarea los valores de sus propiedades (pueden no especificarse). Cada una de estas propiedades puede tener asociado un conjunto de valores recomendados para facilitar la elección al usuario.
		}\\
		
		\hline
		\multicolumn{2}{| X |}
		{\textbf{Observaciones:} - }\\
		
		\hline
	\end{tabularx}
	\caption*{Tabla A.\arabic{counter_tbl_A}: Refinamiento de \hyperlink{US}{US} 8.}
	\addcontentsline{lot}{table}{A.\arabic{counter_tbl_A}. Refinamiento de US 8.}\stepcounter{counter_tbl_A}
\end{table*}
%%%%%%%%%%%%%%%


\pagebreak

\subsection*{Administrar tipos de proyectos}
El encargado de producción debe poder agregar nuevos tipos de proyecto e indicar una secuencia de tareas distinta de la secuencia de los tipos de proyectos genéricos (almanaque, libro, revista o volante/afiche).

%%%%%%%%%%%%%%%
% this line fixes the vertical padding of text inside the cells
\renewcommand{\arraystretch}{1.4}

\begin{table*}[h!]
	\centering
	\begin{tabularx}
		{\textwidth}{|L{6cm}|L{8cm}|}
		\hline 
		\textbf{Número:} 10 & \textbf{Nombre:} Administrar tipos de proyectos \\ \hline
		\multicolumn{2}{| X |}{\textbf{Usuario:} Gerente general, Encargado de producción.}\\ \hline
		\textbf{Iteración Asignada:} 3 & \textbf{Puntos Estimados:} 3 \\ \hline
		\textbf{Prioridad en Negocio:} Media & \multirow{2}{*}{}{\textbf{Puntos Reales:} 3,75 }\\
		\cline{1-1} \textbf{Riesgo en Desarrollo:} Alto & \\ \hline
		\multicolumn{2}{| X |}{\textbf{Descripción:} }\\
		
		\hline
		\multicolumn{2}{| X |}{El encargado de producción puede agregar nuevos tipos de proyectos.\newline
			El encargado de producción puede editar tipos de proyectos.\newline
			Existen 4 tipos de proyectos por defecto, los cuales son: almanaque, libro, revista, volante/afiche.\newline
			Según el tipo de proyecto por defecto, las tareas asociadas serán las siguientes:
			\setlist{nolistsep}
			\begin{itemize}[noitemsep]
				\item Almanaque: impresión, corte, plegado, envarillado/ perforado, empaquetado.
				\item Libro: impresión, encuadernado, corte, plegado, prensado, corte, empaquetado.
				\item Revista: impresión, corte, plegado, prensado, corte, empaquetado.
				\item Volante o afiche: impresión, corte, plegado, prensado, corte, empaquetado.
			\end{itemize}
			Además de los 4 tipos de proyecto por defecto se pueden agregar nuevos con sus respectivas secuencias de tareas y un nombre que los identifique.\newline
			Al agregar o editar la secuencia de tareas de un tipo de proyecto se puede especificar para cada tarea si depende de la finalización de alguna otra tarea para iniciarse.\newline
		}\\
		
		\hline
		\multicolumn{2}{| X |}
		{\textbf{Observaciones:} - }\\
		
		\hline
	\end{tabularx}
	\caption*{Tabla A.\arabic{counter_tbl_A}: Refinamiento de \hyperlink{US}{US} 10.}
	\addcontentsline{lot}{table}{A.\arabic{counter_tbl_A}. Refinamiento de US 10.}\stepcounter{counter_tbl_A}
\end{table*}
%%%%%%%%%%%%%%%


\pagebreak

\subsection*{Asignar tarea a empleado}
El encargado de producción debe poder designar a los empleados que se encargarán de realizar las distintas tareas del proceso de producción.

%%%%%%%%%%%%%%%
% this line fixes the vertical padding of text inside the cells
\renewcommand{\arraystretch}{1.4}

\begin{table*}[h!]
	\centering
	\begin{tabularx}
		{\textwidth}{|L{6cm}|L{8cm}|}
		\hline 
		\textbf{Número:} 12 & \textbf{Nombre:} Asignar tarea a empleado \\ \hline
		\multicolumn{2}{| X |}{\textbf{Usuario:} Gerente general, Encargado de producción.}\\ \hline
		\textbf{Iteración Asignada:} 3 & \textbf{Puntos Estimados:} 2 \\ \hline
		\textbf{Prioridad en Negocio:} Media & \multirow{2}{*}{}{\textbf{Puntos Reales:} 1,75 }\\
		\cline{1-1} \textbf{Riesgo en Desarrollo:} Bajo & \\ \hline
		\multicolumn{2}{| X |}{\textbf{Descripción:} }\\
		
		\hline
		\multicolumn{2}{| X |}{El encargado de producción puede asignar a un empleado de producción a una tarea para indicar que este debe realizarla.\newline
			El encargado de producción puede asignarse a sí mismo una tarea.\newline
			El encargado de producción puede desasignar a un empleado de producción previamente asignado a una tarea.\newline
			Una tarea puede tener un único empleado de producción asignado.\newline
			Una tarea puede no tener empleados asignados.}\\
		
		\hline
		\multicolumn{2}{| X |}
		{\textbf{Observaciones:} - }\\
		
		\hline
	\end{tabularx}
	\caption*{Tabla A.\arabic{counter_tbl_A}: Refinamiento de \hyperlink{US}{US} 12.}
	\addcontentsline{lot}{table}{A.\arabic{counter_tbl_A}. Refinamiento de US 12.}\stepcounter{counter_tbl_A}
\end{table*}
%%%%%%%%%%%%%%%


\pagebreak

\subsection*{Buscar proyecto}
El encargado de producción debe poder buscar los proyectos tanto por título, estado o cliente del proyecto.

%%%%%%%%%%%%%%%
% this line fixes the vertical padding of text inside the cells
\renewcommand{\arraystretch}{1.4}

\begin{table*}[h!]
	\centering
	\begin{tabularx}
		{\textwidth}{|L{6cm}|L{8cm}|}
		\hline 
		\textbf{Número:} 16 & \textbf{Nombre:} Buscar proyecto \\ \hline
		\multicolumn{2}{| X |}{\textbf{Usuario:} Gerente general, Encargado de producción.}\\ \hline
		\textbf{Iteración Asignada:} 3 & \textbf{Puntos Estimados:} 3 \\ \hline
		\textbf{Prioridad en Negocio:} Media & \multirow{2}{*}{}{\textbf{Puntos Reales:} 2,5 }\\
		\cline{1-1} \textbf{Riesgo en Desarrollo:} Bajo & \\ \hline
		\multicolumn{2}{| X |}{\textbf{Descripción:} }\\
		
		\hline
		\multicolumn{2}{| X |}{El encargado de producción puede buscar proyectos indicando de manera completa o parcial el título de un proyecto, su estado o el cliente que lo encargó. Dichos parámetros de búsqueda pueden usarse en conjunto.}\\
		
		\hline
		\multicolumn{2}{| X |}
		{\textbf{Observaciones:} - }\\
		
		\hline
	\end{tabularx}
	\caption*{Tabla A.\arabic{counter_tbl_A}: Refinamiento de \hyperlink{US}{US} 16.}
	\addcontentsline{lot}{table}{A.\arabic{counter_tbl_A}. Refinamiento de US 16.}\stepcounter{counter_tbl_A}
\end{table*}
%%%%%%%%%%%%%%%


\pagebreak

\subsection*{Visualizar tareas asignadas}
Los empleados de producción deben poder visualizar la lista de tareas que se le han asignado y que aún no se han realizado. Además, deben poder visualizar un mayor detalle de una tarea específica y del proyecto del cual forma parte.

%%%%%%%%%%%%%%%
% this line fixes the vertical padding of text inside the cells
\renewcommand{\arraystretch}{1.4}

\begin{table*}[h!]
	\centering
	\begin{tabularx}
		{\textwidth}{|L{6cm}|L{8cm}|}
		\hline 
		\textbf{Número:} 17 & \textbf{Nombre:} Visualizar tareas asignadas \\ \hline
		\multicolumn{2}{| X |}{\textbf{Usuario:} Gerente general, Encargado de producción, Empleado de producción.}\\ \hline
		\textbf{Iteración Asignada:} 3 & \textbf{Puntos Estimados:} 2 \\ \hline
		\textbf{Prioridad en Negocio:} Media & \multirow{2}{*}{}{\textbf{Puntos Reales:} 1,5 }\\
		\cline{1-1} \textbf{Riesgo en Desarrollo:} Bajo & \\ \hline
		\multicolumn{2}{| X |}{\textbf{Descripción:} }\\
		
		\hline
		\multicolumn{2}{| X |}{
			Los empleados de producción pueden listar las tareas que les han sido asignadas.\newline
			Los empleados de producción pueden ver un mayor detalle de las tareas que se les han asignado. Para una tarea se muestra:
			\setlist{nolistsep}
			\begin{itemize}[noitemsep]
				\item El nombre del proyecto al que pertenece.
				\item El tipo de tarea.
				\item El estado de la tarea.
				\item Las propiedades de la tarea y sus correspondientes valores.
			\end{itemize}
		}\\
		\hline
		\multicolumn{2}{| X |}
		{\textbf{Observaciones:} - }\\
		
		\hline
	\end{tabularx}
	\caption*{Tabla A.\arabic{counter_tbl_A}: Refinamiento de \hyperlink{US}{US} 17.}
	\addcontentsline{lot}{table}{A.\arabic{counter_tbl_A}. Refinamiento de US 17.}\stepcounter{counter_tbl_A}
\end{table*}
%%%%%%%%%%%%%%%


\pagebreak

\hypertarget{apendice_a_IV}{}
\section*{Iteración IV}
\subsection*{Agregar insumo a proyecto}
Con fines de planeación y gestión de proyectos, el encargado de producción deberá poder solicitar la reserva de determinada cantidad de un material/insumo para un proyecto determinado. A su vez, deberá poder seleccionar los depósitos de los cuales reservará cada insumo para hacer frente a la solicitud, pudiendo incluso obtener la cantidad requerida de un insumo de diferentes depósitos.

% custom commands
\newcolumntype{L}[1]{>{\raggedright\let\newline\\\arraybackslash\hspace{0pt}}p{#1}}
\newcolumntype{C}[1]{>{\centering\let\newline\\\arraybackslash\hspace{0pt}}p{#1}}
\newcolumntype{R}[1]{>{\raggedleft\let\newline\\\arraybackslash\hspace{0pt}}p{#1}}

%%%%%%%%%%%%%%%
% this line fixes the vertical padding of text inside the cells
\renewcommand{\arraystretch}{1.4}

\begin{table*}[h!]
	\centering
	\begin{tabularx}
		{\textwidth}{|L{6cm}|L{8cm}|}
		\hline 
		\textbf{Número:} 9 & \textbf{Nombre:} Agregar insumo a proyecto \\ \hline
		\multicolumn{2}{| X |}{\textbf{Usuario:} Gerente general, Encargado de producción. }\\ \hline
		\textbf{Iteración Asignada:} 4 & \textbf{Puntos Estimados:} 3 \\ \hline
		\textbf{Prioridad en Negocio:} Media & \multirow{2}{*}{}{\textbf{Puntos Reales:} 2,1875 }\\
		\cline{1-1} \textbf{Riesgo en Desarrollo:} Medio & \\ \hline
		\multicolumn{2}{| X |}{\textbf{Descripción:} }\\
		
		\hline
		\multicolumn{2}{| X |}{El encargado de producción busca un insumo de entre los registrados previamente. Se tiene en cuenta no solo la pertenencia actual de los insumos respecto de los depósitos sino también de los clientes (o la misma empresa Imprenta Lux S.A., de ser el caso).\newline
		Se muestran las cantidades del insumo seleccionado disponible en los distintos depósitos.\newline
		Al seleccionarse una determinada cantidad del insumo para ser reservado para usarse en el proyecto, la cantidad seleccionada debe descontarse de la registrada en los depósitos. No es posible seleccionar una cantidad de insumo superior a la disponible en los depósitos de la empresa.\newline
		Si se alcanza el PP de algún insumo o tipo de insumo debe notificarse al encargado de reposición indicando de qué insumo o tipo de insumo se trata.\newline
		No es posible realizar la reserva de un insumo no perteneciente al cliente relacionado con el proyecto, o, alternativamente, a la empresa Imprenta Lux S.A. (en este último caso la reserva puede realizarse con independencia del cliente).}\\
		
		\hline
		\multicolumn{2}{| X |}
		{\textbf{Observaciones:} - }\\
		
		\hline
	\end{tabularx}
	\caption*{Tabla A.\arabic{counter_tbl_A}: Refinamiento de \hyperlink{US}{US} 9.}
	\addcontentsline{lot}{table}{A.\arabic{counter_tbl_A}. Refinamiento de US 9.}\stepcounter{counter_tbl_A}
\end{table*}
%%%%%%%%%%%%%%%


\pagebreak

\subsection*{Editar tarea}
El encargado de producción debe poder editar las propiedades de cada tarea perteneciente a un proyecto dado.

% custom commands
\newcolumntype{L}[1]{>{\raggedright\let\newline\\\arraybackslash\hspace{0pt}}p{#1}}
\newcolumntype{C}[1]{>{\centering\let\newline\\\arraybackslash\hspace{0pt}}p{#1}}
\newcolumntype{R}[1]{>{\raggedleft\let\newline\\\arraybackslash\hspace{0pt}}p{#1}}

%%%%%%%%%%%%%%%
% this line fixes the vertical padding of text inside the cells
\renewcommand{\arraystretch}{1.4}

\begin{table*}[h!]
	\centering
	\begin{tabularx}
		{\textwidth}{|L{6cm}|L{8cm}|}
		\hline 
		\textbf{Número:} 11 & \textbf{Nombre:} Editar tarea \\ \hline
		\multicolumn{2}{| X |}{\textbf{Usuario:} Gerente general, Encargado de producción. }\\ \hline
		\textbf{Iteración Asignada:} 4 & \textbf{Puntos Estimados:} 2 \\ \hline
		\textbf{Prioridad en Negocio:} Baja & \multirow{2}{*}{}{\textbf{Puntos Reales:} 2,875 }\\
		\cline{1-1} \textbf{Riesgo en Desarrollo:} Bajo & \\ \hline
		\multicolumn{2}{| X |}{\textbf{Descripción:} }\\
		
		\hline
		\multicolumn{2}{| X |}{El encargado de producción puede listar las propiedades de las tareas de un proyecto dado. También puede de modificar los valores en cada caso de acuerdo a los valores sugeridos desde la base de datos, o incluso modificarlos de manera arbitraria (ingresar un nuevo valor inexistente hasta el momento en la base de datos).}\\
		
		\hline
		\multicolumn{2}{| X |}
		{\textbf{Observaciones:} - }\\
		
		\hline
	\end{tabularx}
	\caption*{Tabla A.\arabic{counter_tbl_A}: Refinamiento de \hyperlink{US}{US} 11.}
	\addcontentsline{lot}{table}{A.\arabic{counter_tbl_A}. Refinamiento de US 11.}\stepcounter{counter_tbl_A}
\end{table*}
%%%%%%%%%%%%%%%


\pagebreak

\subsection*{Agregar comentario a tarea}
Los empleados de producción deben poder dejar comentarios en las tareas que les hayan sido asignadas. Estos comentarios sirven para dar mayores detalles acerca de la tarea en cuestión además del estado de la misma.

% custom commands
\newcolumntype{L}[1]{>{\raggedright\let\newline\\\arraybackslash\hspace{0pt}}p{#1}}
\newcolumntype{C}[1]{>{\centering\let\newline\\\arraybackslash\hspace{0pt}}p{#1}}
\newcolumntype{R}[1]{>{\raggedleft\let\newline\\\arraybackslash\hspace{0pt}}p{#1}}

%%%%%%%%%%%%%%%
% this line fixes the vertical padding of text inside the cells
\renewcommand{\arraystretch}{1.4}

\begin{table*}[h!]
	\centering
	\begin{tabularx}
		{\textwidth}{|L{6cm}|L{8cm}|}
		\hline 
		\textbf{Número:} 13 & \textbf{Nombre:} Agregar comentario a tarea \\ \hline
		\multicolumn{2}{| X |}{\textbf{Usuario:} Gerente general, Encargado de producción, Empleado de producción. }\\ \hline
		\textbf{Iteración Asignada:} 4 & \textbf{Puntos Estimados:} 2 \\ \hline
		\textbf{Prioridad en Negocio:} Baja & \multirow{2}{*}{}{\textbf{Puntos Reales:} 1,25 }\\
		\cline{1-1} \textbf{Riesgo en Desarrollo:} Bajo & \\ \hline
		\multicolumn{2}{| X |}{\textbf{Descripción:} }\\
		
		\hline
		\multicolumn{2}{| X |}{Los empleados de producción pueden agregar un comentario en su/s tarea/s asignada/s.\newline
			Los empleados de producción que no estén asignados a una tarea pueden visualizar los comentarios de dicha tarea sin inconvenientes.\newline
			La edición de un comentario solo es aceptada para quien haya sido su creador.\newline
			El gerente general y el encargado de producción pueden añadir comentarios en todas las tareas así como editar los que por ellos hayan sido creados.
		}\\
		
		\hline
		\multicolumn{2}{| X |}
		{\textbf{Observaciones:} - }\\
		
		\hline
	\end{tabularx}
	\caption*{Tabla A.\arabic{counter_tbl_A}: Refinamiento de \hyperlink{US}{US} 13.}
	\addcontentsline{lot}{table}{A.\arabic{counter_tbl_A}. Refinamiento de US 13.}\stepcounter{counter_tbl_A}
\end{table*}
%%%%%%%%%%%%%%%


\pagebreak

\subsection*{Modificar estado de tarea}
El encargado de producción y los empleados de producción deben poder modificar el estado de una tarea perteneciente a un proyecto del cual forman parte. Al finalizar una tarea se debe habilitar el inicio de aquellas tareas que dependen de su finalización para comenzar.

% custom commands
\newcolumntype{L}[1]{>{\raggedright\let\newline\\\arraybackslash\hspace{0pt}}p{#1}}
\newcolumntype{C}[1]{>{\centering\let\newline\\\arraybackslash\hspace{0pt}}p{#1}}
\newcolumntype{R}[1]{>{\raggedleft\let\newline\\\arraybackslash\hspace{0pt}}p{#1}}

%%%%%%%%%%%%%%%
% this line fixes the vertical padding of text inside the cells
\renewcommand{\arraystretch}{1.4}

\begin{table*}[h!]
	\centering
	\begin{tabularx}
		{\textwidth}{|L{6cm}|L{8cm}|}
		\hline 
		\textbf{Número:} 14 & \textbf{Nombre:} Modificar estado de tarea \\ \hline
		\multicolumn{2}{| X |}{\textbf{Usuario:} Gerente general, Encargado de producción, Empleado de producción. }\\ \hline
		\textbf{Iteración Asignada:} 4 & \textbf{Puntos Estimados:} 2 \\ \hline
		\textbf{Prioridad en Negocio:} Baja & \multirow{2}{*}{}{\textbf{Puntos Reales:} 1,1875 }\\
		\cline{1-1} \textbf{Riesgo en Desarrollo:} Bajo & \\ \hline
		\multicolumn{2}{| X |}{\textbf{Descripción:} }\\
		
		\hline
		\multicolumn{2}{| X |}{El encargado de producción tanto como los empleados de producción asignados a una tarea pueden modificar el estado de la tarea.
			Las tareas que conforman los proyectos pueden tener cuatro estados:
			\setlist{nolistsep}
			\begin{itemize}[noitemsep]
			\item En espera: aún no ha sido iniciada.
			\item En proceso: la tarea se está realizando por los empleados de producción.
			\item Parada: son las tareas que se han detenido por algún motivo no especificado.
			\item Finalizada: indica que se han terminado las actividades correspondientes a la tarea en cuestión.
			\end{itemize}
			Si se modifica el estado de una tarea, se notifica al encargado de producción.\newline
			Si un usuario modifica el estado de una tarea finalizada, las tareas dependientes de esta última cambian al estado \textit{pendiente}.\newline
			Si un usuario modifica el estado de una tarea finalizada, y el proyecto al que pertenece se encuentra en estado \textit{finalizado}, el proyecto cambia al estado \textit{iniciado}.\newline
			Si un usuario modifica el estado de una tarea a \textit{iniciado}, y el proyecto se encontraba en un estado distinto al de \textit{iniciado}, el nuevo estado del proyecto también pasa al estado \textit{iniciado}.
		}\\
		
		\hline
		\multicolumn{2}{| X |}
		{\textbf{Observaciones:} - }\\
		
		\hline
	\end{tabularx}
	\caption*{Tabla A.\arabic{counter_tbl_A}: Refinamiento de \hyperlink{US}{US} 14.}
	\addcontentsline{lot}{table}{A.\arabic{counter_tbl_A}. Refinamiento de US 14.}\stepcounter{counter_tbl_A}
\end{table*}
%%%%%%%%%%%%%%%


\pagebreak

\subsection*{Finalizar proyecto}
El encargado de producción debe poder marcar un proyecto como finalizado e indicar la cantidad sobrante de cada insumo reservado en un inicio. Luego, las cantidades restantes deben poder ser emplazadas en los depósitos disponibles.

% custom commands
\newcolumntype{L}[1]{>{\raggedright\let\newline\\\arraybackslash\hspace{0pt}}p{#1}}
\newcolumntype{C}[1]{>{\centering\let\newline\\\arraybackslash\hspace{0pt}}p{#1}}
\newcolumntype{R}[1]{>{\raggedleft\let\newline\\\arraybackslash\hspace{0pt}}p{#1}}

%%%%%%%%%%%%%%%
% this line fixes the vertical padding of text inside the cells
\renewcommand{\arraystretch}{1.4}

\begin{table*}[h!]
	\centering
	\begin{tabularx}
		{\textwidth}{|L{6cm}|L{8cm}|}
		\hline 
		\textbf{Número:} 15 & \textbf{Nombre:} Finalizar proyecto \\ \hline
		\multicolumn{2}{| X |}{\textbf{Usuario:} Gerente general, Encargado de producción. }\\ \hline
		\textbf{Iteración Asignada:} 4 & \textbf{Puntos Estimados:} 2 \\ \hline
		\textbf{Prioridad en Negocio:} Medio & \multirow{2}{*}{}{\textbf{Puntos Reales:} 3,0625 }\\
		\cline{1-1} \textbf{Riesgo en Desarrollo:} Alto & \\ \hline
		\multicolumn{2}{| X |}{\textbf{Descripción:} }\\
		
		\hline
		\multicolumn{2}{| X |}{El encargado de producción puede marcar un proyecto como finalizado.\newline
		No es posible cambiar el estado de un proyecto a \textit{finalizado} si al menos una de sus tareas no ha sido finalizada. Si un usuario intenta cambiar el estado de un proyecto a \textit{finalizado} sin que todas sus tareas asociadas hayan sido finalizadas, el sistema pedirá confirmación para marcar las tareas restantes como finalizadas. En caso de negarse, el proyecto no se marca como finalizado.\newline
		No es posible cambiar el estado de un proyecto a \textit{finalizado} si existen insumos asignados a este. Si un usuario intenta cambiar el estado de un proyecto a \textit{finalizado} sin que todos sus insumos asociados hayan sido reubicados en los depósitos disponibles o marcados como utilizados, el sistema instará al usuario a tomar tales decisiones. En caso de no realizarse tales acciones para la totalidad de los insumos asignados al proyecto, este no se marca como finalizado. La reubicación por defecto de los insumos debe ser la inicial (los insumos reservados para el proyecto, por defecto deben poder reubicarse en sus depósitos de origen en cada caso).}\\
		
		\hline
		\multicolumn{2}{| X |}
		{\textbf{Observaciones:} - }\\
		
		\hline
	\end{tabularx}
	\caption*{Tabla A.\arabic{counter_tbl_A}: Refinamiento de \hyperlink{US}{US} 15.}
	\addcontentsline{lot}{table}{A.\arabic{counter_tbl_A}. Refinamiento de US 15.}\stepcounter{counter_tbl_A}
\end{table*}
%%%%%%%%%%%%%%%


\pagebreak

\subsection*{Visualizar proyecto}
El encargado de producción y los empleados de producción deben poder visualizar el estado del proyecto del cual forman parte (al tener asignada estos últimos al menos una tarea perteneciente a este). El encargado de producción debe poder modificar el estado del proyecto.

% custom commands
\newcolumntype{L}[1]{>{\raggedright\let\newline\\\arraybackslash\hspace{0pt}}p{#1}}
\newcolumntype{C}[1]{>{\centering\let\newline\\\arraybackslash\hspace{0pt}}p{#1}}
\newcolumntype{R}[1]{>{\raggedleft\let\newline\\\arraybackslash\hspace{0pt}}p{#1}}

%%%%%%%%%%%%%%%
% this line fixes the vertical padding of text inside the cells
\renewcommand{\arraystretch}{1.4}

\begin{table*}[h!]
	\centering
	\begin{tabularx}
		{\textwidth}{|L{6cm}|L{8cm}|}
		\hline 
		\textbf{Número:} 25 & \textbf{Nombre:} Visualizar proyecto \\ \hline
		\multicolumn{2}{| X |}{\textbf{Usuario:} Gerente general, Encargado de producción, Empleado de producción. }\\ \hline
		\textbf{Iteración Asignada:} 4 & \textbf{Puntos Estimados:} 3 \\ \hline
		\textbf{Prioridad en Negocio:} Alta & \multirow{2}{*}{}{\textbf{Puntos Reales:} 2,125 }\\
		\cline{1-1} \textbf{Riesgo en Desarrollo:} Bajo & \\ \hline
		\multicolumn{2}{| X |}{\textbf{Descripción:} }\\
		
		\hline
		\multicolumn{2}{| X |}{El encargado de producción tanto como los empleados de producción que tengan al menos una tarea asignada de un proyecto, pueden visualizar tanto las tareas como el estado de las mismas, así como el estado del proyecto.\newline
		Las tareas se mostrarán de acuerdo a la dependencia que guardan, no pudiéndose tener un estado distinto de \textit{pendiente} si es que la/s tarea/s de la/s que dependen no ha/n sido finalizada/s aún.\newline
		El encargado de producción puede modificar el estado del proyecto mostrado. En caso de modificarse el estado del proyecto a \textit{finalizado}, todas las tareas que pertenecen a este, deben a su vez pasar al estado \textit{finalizado}.}\\
		
		\hline
		\multicolumn{2}{| X |}
		{\textbf{Observaciones:} - }\\
		
		\hline
	\end{tabularx}
	\caption*{Tabla A.\arabic{counter_tbl_A}: Refinamiento de \hyperlink{US}{US} 25.}
	\addcontentsline{lot}{table}{A.\arabic{counter_tbl_A}. Refinamiento de US 25.}\stepcounter{counter_tbl_A}
\end{table*}
%%%%%%%%%%%%%%%


\pagebreak

\hypertarget{apendice_a_V}{}
\section*{Iteración V}
\subsection*{Eliminar usuario}
El gerente general debe poder eliminar usuarios del sistema.

% custom commands
\newcolumntype{L}[1]{>{\raggedright\let\newline\\\arraybackslash\hspace{0pt}}p{#1}}
\newcolumntype{C}[1]{>{\centering\let\newline\\\arraybackslash\hspace{0pt}}p{#1}}
\newcolumntype{R}[1]{>{\raggedleft\let\newline\\\arraybackslash\hspace{0pt}}p{#1}}

%%%%%%%%%%%%%%%
% this line fixes the vertical padding of text inside the cells
\renewcommand{\arraystretch}{1.4}

\begin{table*}[h!]
	\centering
	\begin{tabularx}
		{\textwidth}{|L{6cm}|L{8cm}|}
		\hline 
		\textbf{Número:} 19 & \textbf{Nombre:} Eliminar usuario \\ \hline
		\multicolumn{2}{| X |}{\textbf{Usuario:} Gerente general. }\\ \hline
		\textbf{Iteración Asignada:} 5 & \textbf{Puntos Estimados:} 1 \\ \hline
		\textbf{Prioridad en Negocio:} Baja & \multirow{2}{*}{}{\textbf{Puntos Reales:} 0,875 }\\
		\cline{1-1} \textbf{Riesgo en Desarrollo:} Medio & \\ \hline
		\multicolumn{2}{| X |}{\textbf{Descripción:} }\\
		
		\hline
		\multicolumn{2}{| X |}{El gerente general puede eliminar los usuarios del sistema de forma permanente.\newline
		Se requiere la confirmación del usuario para proceder con la eliminación.\newline
		Al eliminar un usuario también se eliminan las notificaciones que el usuario haya recibido.\newline
		Al eliminar un usuario se lo desasigna de las tareas no finalizadas a las que fue asignado. Si alguna de estas tareas asignadas al usuario fueron iniciadas deben marcarse como pendientes.\newline
		Si se elimina un usuario con tareas asignadas finalizadas, estas deben ser reasignadas al pseudousuario <<Usuario\_eliminado>>.\newline
		Si se elimina un usuario que ha realizado comentarios en tareas, estas deben ser marcadas como realizadas por el pseudousuario <<Usuario\_eliminado>>.\newline
		No se puede eliminar el pseudousuario <<Usuario\_eliminado>>.\newline
		Un usuario no puede eliminarse a sí mismo.
		}\\
		
		\hline
		\multicolumn{2}{| X |}
		{\textbf{Observaciones:} - }\\
		
		\hline
	\end{tabularx}
	\caption*{Tabla A.\arabic{counter_tbl_A}: Refinamiento de \hyperlink{US}{US} 19.}
	\addcontentsline{lot}{table}{A.\arabic{counter_tbl_A}. Refinamiento de US 19.}\stepcounter{counter_tbl_A}
\end{table*}
%%%%%%%%%%%%%%%


\pagebreak

\subsection*{Eliminar depósito}
El encargado de mantenimiento debe poder eliminar depósitos.

% custom commands
\newcolumntype{L}[1]{>{\raggedright\let\newline\\\arraybackslash\hspace{0pt}}p{#1}}
\newcolumntype{C}[1]{>{\centering\let\newline\\\arraybackslash\hspace{0pt}}p{#1}}
\newcolumntype{R}[1]{>{\raggedleft\let\newline\\\arraybackslash\hspace{0pt}}p{#1}}

%%%%%%%%%%%%%%%
% this line fixes the vertical padding of text inside the cells
\renewcommand{\arraystretch}{1.4}

\begin{table*}[h!]
	\centering
	\begin{tabularx}
		{\textwidth}{|L{6cm}|L{8cm}|}
		\hline 
		\textbf{Número:} 20 & \textbf{Nombre:} Eliminar depósito \\ \hline
		\multicolumn{2}{| X |}{\textbf{Usuario:} Gerente general, Encargado de mantenimiento. }\\ \hline
		\textbf{Iteración Asignada:} 5 & \textbf{Puntos Estimados:} 1 \\ \hline
		\textbf{Prioridad en Negocio:} Baja & \multirow{2}{*}{}{\textbf{Puntos Reales:} 0,75 }\\
		\cline{1-1} \textbf{Riesgo en Desarrollo:} Alto & \\ \hline
		\multicolumn{2}{| X |}{\textbf{Descripción:} }\\
		
		\hline
		\multicolumn{2}{| X |}{El encargado de mantenimiento puede eliminar los depósitos del sistema de forma permanente.\newline
		Se requiere la confirmación del usuario para proceder con la eliminación.\newline
		Se requiere que el usuario seleccione un depósito sustituto para que, al desasignar de un proyecto insumos provenientes del depósito a eliminar, se envíen a tal depósito sustituto.\newline
		Al eliminar un depósito también se eliminan las provisiones de insumos asociadas al depósito. Si tras la eliminación del depósito se alcanza el PP de algún insumo o tipo de insumo allí almacenado, se le notifica al encargado de reposición que se ha alcanzado el o los PP correspondientes.\newline
		No es posible eliminar un depósito si este es el único del sistema.
		}\\
		
		\hline
		\multicolumn{2}{| X |}
		{\textbf{Observaciones:} - }\\
		
		\hline
	\end{tabularx}
	\caption*{Tabla A.\arabic{counter_tbl_A}: Refinamiento de \hyperlink{US}{US} 20.}
	\addcontentsline{lot}{table}{A.\arabic{counter_tbl_A}. Refinamiento de US 20.}\stepcounter{counter_tbl_A}
\end{table*}
%%%%%%%%%%%%%%%


\pagebreak

\subsection*{Eliminar sucursal}
El encargado de mantenimiento debe poder eliminar sucursales.

% custom commands
\newcolumntype{L}[1]{>{\raggedright\let\newline\\\arraybackslash\hspace{0pt}}p{#1}}
\newcolumntype{C}[1]{>{\centering\let\newline\\\arraybackslash\hspace{0pt}}p{#1}}
\newcolumntype{R}[1]{>{\raggedleft\let\newline\\\arraybackslash\hspace{0pt}}p{#1}}

%%%%%%%%%%%%%%%
% this line fixes the vertical padding of text inside the cells
\renewcommand{\arraystretch}{1.4}

\begin{table*}[h!]
	\centering
	\begin{tabularx}
		{\textwidth}{|L{6cm}|L{8cm}|}
		\hline 
		\textbf{Número:} 21 & \textbf{Nombre:} Eliminar sucursal \\ \hline
		\multicolumn{2}{| X |}{\textbf{Usuario:} Gerente general, Encargado de mantenimiento. }\\ \hline
		\textbf{Iteración Asignada:} 5 & \textbf{Puntos Estimados:} 1 \\ \hline
		\textbf{Prioridad en Negocio:} Baja & \multirow{2}{*}{}{\textbf{Puntos Reales:} 0,625 }\\
		\cline{1-1} \textbf{Riesgo en Desarrollo:} Alto & \\ \hline
		\multicolumn{2}{| X |}{\textbf{Descripción:} }\\
		
		\hline
		\multicolumn{2}{| X |}{El encargado de mantenimiento puede eliminar las sucursales del sistema de forma permanente.\newline
		Se requiere la confirmación del usuario para proceder con la eliminación.\newline
		Se requiere que el usuario seleccione un depósito sustituto para que, al desasignar de un proyecto insumos provenientes de depósitos de la sucursal a eliminar, se envíen a tal depósito sustituto.\newline
		Al eliminar una sucursal también se eliminan los depósitos asociados.\newline
		Al eliminar una sucursal también se eliminan las provisiones de insumos asociadas a la sucursal. Si tras la eliminación de la sucursal se alcanza el PP de algún insumo o tipo de insumo allí almacenado, se le notifica al encargado de reposición que se ha alcanzado el o los PP correspondientes.\newline
		No es posible eliminar una sucursal si esta acción conlleva la eliminación de todos los depósitos del sistema.
		}\\
		
		\hline
		\multicolumn{2}{| X |}
		{\textbf{Observaciones:} - }\\
		
		\hline
	\end{tabularx}
	\caption*{Tabla A.\arabic{counter_tbl_A}: Refinamiento de \hyperlink{US}{US} 21.}
	\addcontentsline{lot}{table}{A.\arabic{counter_tbl_A}. Refinamiento de US 21.}\stepcounter{counter_tbl_A}
\end{table*}
%%%%%%%%%%%%%%%


\pagebreak

\subsection*{Eliminar cliente}
El gerente general debe poder eliminar clientes registrados en el sistema.

% custom commands
\newcolumntype{L}[1]{>{\raggedright\let\newline\\\arraybackslash\hspace{0pt}}p{#1}}
\newcolumntype{C}[1]{>{\centering\let\newline\\\arraybackslash\hspace{0pt}}p{#1}}
\newcolumntype{R}[1]{>{\raggedleft\let\newline\\\arraybackslash\hspace{0pt}}p{#1}}

%%%%%%%%%%%%%%%
% this line fixes the vertical padding of text inside the cells
\renewcommand{\arraystretch}{1.4}

\begin{table*}[h!]
	\centering
	\begin{tabularx}
		{\textwidth}{|L{6cm}|L{8cm}|}
		\hline 
		\textbf{Número:} 22 & \textbf{Nombre:} Eliminar cliente \\ \hline
		\multicolumn{2}{| X |}{\textbf{Usuario:} Gerente general }\\ \hline
		\textbf{Iteración Asignada:} 5 & \textbf{Puntos Estimados:} 0,5 \\ \hline
		\textbf{Prioridad en Negocio:} Baja & \multirow{2}{*}{}{\textbf{Puntos Reales:} 0,75 }\\
		\cline{1-1} \textbf{Riesgo en Desarrollo:} Medio & \\ \hline
		\multicolumn{2}{| X |}{\textbf{Descripción:} }\\
		
		\hline
		\multicolumn{2}{| X |}{El gerente general puede eliminar clientes.\newline
		Se requiere la confirmación del usuario para proceder con la eliminación.\newline
		Al eliminar un cliente se eliminan los proyectos relacionados a este, así como las tareas que los componen y los comentarios relacionados.\newline
		Al eliminar un cliente se eliminan los insumos de los que es propietario.\newline
		No es posible eliminar un cliente que tenga proyectos con reservas de insumos propiedad de Imprenta Lux S.A.
		}\\
		
		\hline
		\multicolumn{2}{| X |}
		{\textbf{Observaciones:} - }\\
		
		\hline
	\end{tabularx}
	\caption*{Tabla A.\arabic{counter_tbl_A}: Refinamiento de \hyperlink{US}{US} 22.}
	\addcontentsline{lot}{table}{A.\arabic{counter_tbl_A}. Refinamiento de US 22.}\stepcounter{counter_tbl_A}
\end{table*}
%%%%%%%%%%%%%%%


\pagebreak

\subsection*{Eliminar insumo}
El encargado de mantenimiento debe poder eliminar insumos.

% custom commands
\newcolumntype{L}[1]{>{\raggedright\let\newline\\\arraybackslash\hspace{0pt}}p{#1}}
\newcolumntype{C}[1]{>{\centering\let\newline\\\arraybackslash\hspace{0pt}}p{#1}}
\newcolumntype{R}[1]{>{\raggedleft\let\newline\\\arraybackslash\hspace{0pt}}p{#1}}

%%%%%%%%%%%%%%%
% this line fixes the vertical padding of text inside the cells
\renewcommand{\arraystretch}{1.4}

\begin{table*}[h!]
	\centering
	\begin{tabularx}
		{\textwidth}{|L{6cm}|L{8cm}|}
		\hline 
		\textbf{Número:} 26 & \textbf{Nombre:} Eliminar insumo \\ \hline
		\multicolumn{2}{| X |}{\textbf{Usuario:} Gerente general, Encargado de mantenimiento. }\\ \hline
		\textbf{Iteración Asignada:} 5 & \textbf{Puntos Estimados:} 0,5 \\ \hline
		\textbf{Prioridad en Negocio:} Baja & \multirow{2}{*}{}{\textbf{Puntos Reales:} 0,375 }\\
		\cline{1-1} \textbf{Riesgo en Desarrollo:} Bajo & \\ \hline
		\multicolumn{2}{| X |}{\textbf{Descripción:} }\\
		
		\hline
		\multicolumn{2}{| X |}{El encargado de mantenimiento puede eliminar los insumos registrados en el sistema de forma permanente.\newline
		Se requiere la confirmación del usuario para proceder con la eliminación.\newline
		El sistema advierte si es que hay reservas del insumo a eliminar asociadas a proyectos no finalizados, indicando de qué proyectos se trata antes de proceder con la eliminación.\newline
		Al eliminar un insumo también se eliminan las reservas de dicho insumo asociadas a proyectos.\newline
		Si tras la eliminación del insumo se alcanza el PP de su tipo de insumo, se le notifica al encargado de reposición que se ha alcanzado dicho PP.
		}\\
		
		\hline
		\multicolumn{2}{| X |}
		{\textbf{Observaciones:} - }\\
		
		\hline
	\end{tabularx}
	\caption*{Tabla A.\arabic{counter_tbl_A}: Refinamiento de \hyperlink{US}{US} 26.}
	\addcontentsline{lot}{table}{A.\arabic{counter_tbl_A}. Refinamiento de US 26.}\stepcounter{counter_tbl_A}
\end{table*}
%%%%%%%%%%%%%%%


\pagebreak

\subsection*{Eliminar tipo de insumo}
El encargado de mantenimiento debe poder eliminar tipos de insumos.

% custom commands
\newcolumntype{L}[1]{>{\raggedright\let\newline\\\arraybackslash\hspace{0pt}}p{#1}}
\newcolumntype{C}[1]{>{\centering\let\newline\\\arraybackslash\hspace{0pt}}p{#1}}
\newcolumntype{R}[1]{>{\raggedleft\let\newline\\\arraybackslash\hspace{0pt}}p{#1}}

%%%%%%%%%%%%%%%
% this line fixes the vertical padding of text inside the cells
\renewcommand{\arraystretch}{1.4}

\begin{table*}[h!]
	\centering
	\begin{tabularx}
		{\textwidth}{|L{6cm}|L{8cm}|}
		\hline 
		\textbf{Número:} 27 & \textbf{Nombre:} Eliminar tipo de insumo \\ \hline
		\multicolumn{2}{| X |}{\textbf{Usuario:} Gerente general, Encargado de mantenimiento. }\\ \hline
		\textbf{Iteración Asignada:} 5 & \textbf{Puntos Estimados:} 0,5 \\ \hline
		\textbf{Prioridad en Negocio:} Baja & \multirow{2}{*}{}{\textbf{Puntos Reales:} 0,375 }\\
		\cline{1-1} \textbf{Riesgo en Desarrollo:} Bajo & \\ \hline
		\multicolumn{2}{| X |}{\textbf{Descripción:} }\\
		
		\hline
		\multicolumn{2}{| X |}{El encargado de mantenimiento puede eliminar los tipos de insumos registrados en el sistema de forma permanente.\newline
		Se requiere la confirmación del usuario para proceder con la eliminación.\newline
		El sistema advierte si es que hay reservas de insumos del tipo de insumo a eliminar asociadas a proyectos no finalizados, indicando de qué proyectos se trata antes de proceder con la eliminación.\newline
		El sistema indica los insumos que serán eliminados en caso de proceder con la eliminación.\newline
		Al eliminar un tipo de insumo también se eliminan los insumos de dicho tipo de insumo.\newline
		Al eliminar un tipo de insumo también se eliminan las reservas de dicho tipo de insumo asociadas a proyectos.
		}\\
		
		\hline
		\multicolumn{2}{| X |}
		{\textbf{Observaciones:} - }\\
		
		\hline
	\end{tabularx}
	\caption*{Tabla A.\arabic{counter_tbl_A}: Refinamiento de \hyperlink{US}{US} 27.}
	\addcontentsline{lot}{table}{A.\arabic{counter_tbl_A}. Refinamiento de US 27.}\stepcounter{counter_tbl_A}
\end{table*}
%%%%%%%%%%%%%%%


\pagebreak

\subsection*{Eliminar tipo de tarea}
El encargado de producción debe poder eliminar los tipos de tareas registrados.

% custom commands
\newcolumntype{L}[1]{>{\raggedright\let\newline\\\arraybackslash\hspace{0pt}}p{#1}}
\newcolumntype{C}[1]{>{\centering\let\newline\\\arraybackslash\hspace{0pt}}p{#1}}
\newcolumntype{R}[1]{>{\raggedleft\let\newline\\\arraybackslash\hspace{0pt}}p{#1}}

%%%%%%%%%%%%%%%
% this line fixes the vertical padding of text inside the cells
\renewcommand{\arraystretch}{1.4}

\begin{table*}[h!]
	\centering
	\begin{tabularx}
		{\textwidth}{|L{6cm}|L{8cm}|}
		\hline 
		\textbf{Número:} 28 & \textbf{Nombre:} Eliminar tipo de tarea \\ \hline
		\multicolumn{2}{| X |}{\textbf{Usuario:} Gerente general, Encargado de producción. }\\ \hline
		\textbf{Iteración Asignada:} 5 & \textbf{Puntos Estimados:} 1 \\ \hline
		\textbf{Prioridad en Negocio:} Baja & \multirow{2}{*}{}{\textbf{Puntos Reales:} 1,25 }\\
		\cline{1-1} \textbf{Riesgo en Desarrollo:} Alto & \\ \hline
		\multicolumn{2}{| X |}{\textbf{Descripción:} }\\
		
		\hline
		\multicolumn{2}{| X |}{El encargado de producción puede eliminar tipos de tareas de forma permanente.\newline
		Se requiere la confirmación del usuario para proceder con la eliminación.\newline
		El sistema indica los tipos de proyectos que están relacionados al tipo de tarea a eliminar.\newline
		No es posible eliminar un tipo de tarea si está siendo usado para definir la secuencia de tareas de un tipo de proyecto.
		}\\
		
		\hline
		\multicolumn{2}{| X |}
		{\textbf{Observaciones:} - }\\
		
		\hline
	\end{tabularx}
	\caption*{Tabla A.\arabic{counter_tbl_A}: Refinamiento de \hyperlink{US}{US} 28.}
	\addcontentsline{lot}{table}{A.\arabic{counter_tbl_A}. Refinamiento de US 28.}\stepcounter{counter_tbl_A}
\end{table*}
%%%%%%%%%%%%%%%


\pagebreak

\subsection*{Eliminar proyecto}
El encargado de producción debe poder eliminar proyectos.

% custom commands
\newcolumntype{L}[1]{>{\raggedright\let\newline\\\arraybackslash\hspace{0pt}}p{#1}}
\newcolumntype{C}[1]{>{\centering\let\newline\\\arraybackslash\hspace{0pt}}p{#1}}
\newcolumntype{R}[1]{>{\raggedleft\let\newline\\\arraybackslash\hspace{0pt}}p{#1}}

%%%%%%%%%%%%%%%
% this line fixes the vertical padding of text inside the cells
\renewcommand{\arraystretch}{1.4}

\begin{table*}[h!]
	\centering
	\begin{tabularx}
		{\textwidth}{|L{6cm}|L{8cm}|}
		\hline 
		\textbf{Número:} 29 & \textbf{Nombre:} Eliminar proyecto \\ \hline
		\multicolumn{2}{| X |}{\textbf{Usuario:} Gerente general, Encargado de producción. }\\ \hline
		\textbf{Iteración Asignada:} 5 & \textbf{Puntos Estimados:} 1 \\ \hline
		\textbf{Prioridad en Negocio:} Baja & \multirow{2}{*}{}{\textbf{Puntos Reales:} 1 }\\
		\cline{1-1} \textbf{Riesgo en Desarrollo:} Medio & \\ \hline
		\multicolumn{2}{| X |}{\textbf{Descripción:} }\\
		
		\hline
		\multicolumn{2}{| X |}{El encargado de producción puede eliminar proyectos.\newline
		Se requiere la confirmación del usuario para proceder con la eliminación.\newline
		Si el proyecto tiene reservas de insumos relacionadas, el usuario debe seleccionar a qué depósito/s serán enviadas antes de proceder con la eliminación.\newline
		Al eliminar un proyecto también se eliminan las tareas y comentarios asociados a este.
		}\\
		
		\hline
		\multicolumn{2}{| X |}
		{\textbf{Observaciones:} - }\\
		
		\hline
	\end{tabularx}
	\caption*{Tabla A.\arabic{counter_tbl_A}: Refinamiento de \hyperlink{US}{US} 29.}
	\addcontentsline{lot}{table}{A.\arabic{counter_tbl_A}. Refinamiento de US 29.}\stepcounter{counter_tbl_A}
\end{table*}
%%%%%%%%%%%%%%%


\pagebreak

\subsection*{Eliminar tipo de proyecto}
El encargado de producción debe poder eliminar tipos de proyecto.

% custom commands
\newcolumntype{L}[1]{>{\raggedright\let\newline\\\arraybackslash\hspace{0pt}}p{#1}}
\newcolumntype{C}[1]{>{\centering\let\newline\\\arraybackslash\hspace{0pt}}p{#1}}
\newcolumntype{R}[1]{>{\raggedleft\let\newline\\\arraybackslash\hspace{0pt}}p{#1}}

%%%%%%%%%%%%%%%
% this line fixes the vertical padding of text inside the cells
\renewcommand{\arraystretch}{1.4}

\begin{table*}[h!]
	\centering
	\begin{tabularx}
		{\textwidth}{|L{6cm}|L{8cm}|}
		\hline 
		\textbf{Número:} 30 & \textbf{Nombre:} Eliminar tipo de proyecto \\ \hline
		\multicolumn{2}{| X |}{\textbf{Usuario:} Gerente general, Encargado de producción. }\\ \hline
		\textbf{Iteración Asignada:} 5 & \textbf{Puntos Estimados:} 0,5 \\ \hline
		\textbf{Prioridad en Negocio:} Baja & \multirow{2}{*}{}{\textbf{Puntos Reales:} 0,375 }\\
		\cline{1-1} \textbf{Riesgo en Desarrollo:} Bajo & \\ \hline
		\multicolumn{2}{| X |}{\textbf{Descripción:} }\\
		
		\hline
		\multicolumn{2}{| X |}{El encargado de producción puede eliminar tipos de proyecto.\newline
		Se requiere la confirmación del usuario para proceder con la eliminación.
		}\\
		
		\hline
		\multicolumn{2}{| X |}
		{\textbf{Observaciones:} - }\\
		
		\hline
	\end{tabularx}
	\caption*{Tabla A.\arabic{counter_tbl_A}: Refinamiento de \hyperlink{US}{US} 30.}
	\addcontentsline{lot}{table}{A.\arabic{counter_tbl_A}. Refinamiento de US 30.}\stepcounter{counter_tbl_A}
\end{table*}
%%%%%%%%%%%%%%%


\pagebreak

\subsection*{Generar reporte de insumos disponibles en cada depósito}
El encargado de mantenimiento debe poder generar y guardar una copia de la lista de los insumos disponibles actualmente en cada depósito.

% custom commands
\newcolumntype{L}[1]{>{\raggedright\let\newline\\\arraybackslash\hspace{0pt}}p{#1}}
\newcolumntype{C}[1]{>{\centering\let\newline\\\arraybackslash\hspace{0pt}}p{#1}}
\newcolumntype{R}[1]{>{\raggedleft\let\newline\\\arraybackslash\hspace{0pt}}p{#1}}

%%%%%%%%%%%%%%%
% this line fixes the vertical padding of text inside the cells
\renewcommand{\arraystretch}{1.4}

\begin{table*}[h!]
	\centering
	\begin{tabularx}
		{\textwidth}{|L{6cm}|L{8cm}|}
		\hline 
		\textbf{Número:} 31 & \textbf{Nombre:} Generar reporte de insumos disponibles en cada depósito \\ \hline
		\multicolumn{2}{| X |}{\textbf{Usuario:} Gerente general, Encargado de mantenimiento. }\\ \hline
		\textbf{Iteración Asignada:} 5 & \textbf{Puntos Estimados:} 1 \\ \hline
		\textbf{Prioridad en Negocio:} Baja & \multirow{2}{*}{}{\textbf{Puntos Reales:} 1,625 }\\
		\cline{1-1} \textbf{Riesgo en Desarrollo:} Bajo & \\ \hline
		\multicolumn{2}{| X |}{\textbf{Descripción:} }\\
		
		\hline
		\multicolumn{2}{| X |}{El encargado de mantenimiento puede generar una lista indicando, para cada depósito, la cantidad de cada insumo allí disponible.\newline
		La lista se genera en formato XLSX.\newline
		La existencia de cada depósito se muestra en hojas de cálculo separadas.\newline
		Los insumos disponibles se muestran en filas separadas, indicando su nombre/designación, cantidad disponible y la unidad de medida.\newline
		El usuario puede seleccionar el directorio local donde desea guardar el informe generado.
		}\\
		
		\hline
		\multicolumn{2}{| X |}
		{\textbf{Observaciones:} - }\\
		
		\hline
	\end{tabularx}
	\caption*{Tabla A.\arabic{counter_tbl_A}: Refinamiento de \hyperlink{US}{US} 31.}
	\addcontentsline{lot}{table}{A.\arabic{counter_tbl_A}. Refinamiento de US 31.}\stepcounter{counter_tbl_A}
\end{table*}
%%%%%%%%%%%%%%%


\pagebreak


\chapter*{\hypertarget{apendice_b}{}Apéndice B\\Mock-ups y pantallas desarrolladas}
\markboth{Apéndice B - Mock-ups y pantallas desarrolladas}{Apéndice B - Mock-ups y pantallas desarrolladas}
\sectionmark{Apéndice B - Mock-ups y pantallas desarrolladas}
\addcontentsline{toc}{chapter}{B. Mock-ups y pantallas desarrolladas}

%%%%%%%%%%%%%%%%%%%%%%%%%%%%%%%%%%%%%%%%%
\newcounter{counter_img_B}  \setcounter{counter_img_B}{1}
%%%%%%%%%%%%%%%%%%%%%%%%%%%%%%%%%%%%%%%%%

\section*{Iteración I}

\subsection*{Inicio de sesión (\hyperlink{US}{US} 1)}

\begin{figure}[H]
	\centering
	{%
		\setlength{\fboxsep}{0pt}%
		\setlength{\fboxrule}{0.5pt}%
		\fbox{\includegraphics[scale = 0.5]{pantallas/us01_mockup.jpg}}%
	}%
	\caption*{Figura B.\arabic{counter_img_B}: Mock-up \hyperlink{US}{US} 1.}
	\addcontentsline{lof}{figure}{B.\arabic{counter_img_B}. Mock-up US 1.}
	\stepcounter{counter_img_B}
	\label{us01_mockup}
\end{figure}

\begin{figure}[H] 
	\centering
	\subfloat[Escritorio \hyperlink{US}{US} 1.]{%
		\includegraphics[width=0.73\textwidth]{pantallas/us01_desktop.jpg}%
		\label{fig:a}%
	}%
	\hfill%
	\subfloat[Móvil \hyperlink{US}{US} 1.]{%
		\setlength{\fboxrule}{0.5pt}%
		\fbox{\includegraphics[width=0.24\textwidth]
			{pantallas/us01_mobile.jpg}}%
		\label{fig:b}%
	}%
	\caption*{Figura B.\arabic{counter_img_B}: Versiones de escritorio y móvil de pantalla correspondiente a \hyperlink{US}{US} 1.}
	\addcontentsline{lof}{figure}{B.\arabic{counter_img_B}. Versiones de escritorio y móvil de pantalla correspondiente a US 1.}
	\stepcounter{counter_img_B}
\end{figure}


\subsection*{Administrar usuarios (\hyperlink{US}{US} 2)}

\begin{figure}[H]
	\centering
	{%
		\setlength{\fboxsep}{0pt}%
		\setlength{\fboxrule}{0.5pt}%
		\fbox{\includegraphics[scale = 0.47]{pantallas/us02_mockup_new.png}}%
	}%
	\caption*{Figura B.\arabic{counter_img_B}: Mock-up \hyperlink{US}{US} 2 (nuevo usuario).}
	\addcontentsline{lof}{figure}{B.\arabic{counter_img_B}. Mock-up US 2 (nuevo usuario).}
	\stepcounter{counter_img_B}
	\label{us02_mockup}
\end{figure}

\begin{figure}[H] 
	\centering
	\subfloat[Escritorio \hyperlink{US}{US} 2 (nuevo).]{%
		\includegraphics[width=0.73\textwidth]{pantallas/us02_desktop_new.jpg}%
		\label{fig:a}%
	}%
	\hfill%
	\subfloat[Móvil \hyperlink{US}{US} 2 (nuevo).]{%
		\setlength{\fboxrule}{0.5pt}%
		\fbox{\includegraphics[width=0.24\textwidth]
			{pantallas/us02_mobile_new.jpg}}%
		\label{fig:b}%
	}%
	\caption*{Figura B.\arabic{counter_img_B}: Versiones de escritorio y móvil de pantalla correspondiente a \hyperlink{US}{US} 2 (nuevo usuario).}
	\addcontentsline{lof}{figure}{B.\arabic{counter_img_B}. Versiones de escritorio y móvil de pantalla correspondiente a US 2 (nuevo usuario).}
	\stepcounter{counter_img_B}
\end{figure}

\begin{figure}[H]
	\centering
	{%
		\setlength{\fboxsep}{0pt}%
		\setlength{\fboxrule}{0.5pt}%
		\fbox{\includegraphics[scale = 0.47]{pantallas/us02_mockup_edit.jpg}}%
	}%
	\caption*{Figura B.\arabic{counter_img_B}: Mock-up \hyperlink{US}{US} 2 (editar usuario).}
	\addcontentsline{lof}{figure}{B.\arabic{counter_img_B}. Mock-up US 2 (editar usuario).}
	\stepcounter{counter_img_B}
	\label{us02_mockup_edit}
\end{figure}

\begin{figure}[H] 
	\centering
	\subfloat[Escritorio \hyperlink{US}{US} 2 (editar).]{%
		\includegraphics[width=0.73\textwidth]{pantallas/us02_desktop_edit.jpg}%
		\label{fig:a}%
	}%
	\hfill%
	\subfloat[Móvil \hyperlink{US}{US} 2 (editar).]{%
		\setlength{\fboxrule}{0.5pt}%
		\fbox{\includegraphics[width=0.24\textwidth]
			{pantallas/us02_mobile_edit.jpg}}%
		\label{fig:b}%
	}%
	\caption*{Figura B.\arabic{counter_img_B}: Versiones de escritorio y móvil de pantalla correspondiente a \hyperlink{US}{US} 2  (editar usuario).}
	\addcontentsline{lof}{figure}{B.\arabic{counter_img_B}. Versiones de escritorio y móvil de pantalla correspondiente a US 2 (editar usuario).}
	\stepcounter{counter_img_B}
\end{figure}


\subsection*{Administrar depósitos (\hyperlink{US}{US} 6)}

\begin{figure}[H]
	\centering
	{%
		\setlength{\fboxsep}{0pt}%
		\setlength{\fboxrule}{0.5pt}%
		\fbox{\includegraphics[scale = 0.47]{pantallas/us06_mockup_new.jpg}}%
	}%
	\caption*{Figura B.\arabic{counter_img_B}: Mock-up \hyperlink{US}{US} 6 (nuevo depósito).}
	\addcontentsline{lof}{figure}{B.\arabic{counter_img_B}. Mock-up US 6 (nuevo depósito).}
	\stepcounter{counter_img_B}
	\label{us06_mockup}
\end{figure}

\begin{figure}[H] 
	\centering
	\subfloat[Escritorio \hyperlink{US}{US} 6 (nuevo).]{%
		\includegraphics[width=0.73\textwidth]{pantallas/us06_desktop_new.jpg}%
		\label{fig:a}%
	}%
	\hfill%
	\subfloat[Móvil \hyperlink{US}{US} 6 (nuevo).]{%
		\setlength{\fboxrule}{0.5pt}%
		\fbox{\includegraphics[width=0.24\textwidth]
			{pantallas/us06_mobile_new.jpg}}%
		\label{fig:b}%
	}%
	\caption*{Figura B.\arabic{counter_img_B}: Versiones de escritorio y móvil de pantalla correspondiente a \hyperlink{US}{US} 6  (nuevo depósito).}
	\addcontentsline{lof}{figure}{B.\arabic{counter_img_B}. Versiones de escritorio y móvil de pantalla correspondiente a US 6 (nuevo depósito).}
	\stepcounter{counter_img_B}
\end{figure}

\begin{figure}[H]
	\centering
	{%
		\setlength{\fboxsep}{0pt}%
		\setlength{\fboxrule}{0.5pt}%
		\fbox{\includegraphics[scale = 0.47]{pantallas/us06_mockup_edit.jpg}}%
	}%
	\caption*{Figura B.\arabic{counter_img_B}: Mock-up \hyperlink{US}{US} 6 (editar depósito).}
	\addcontentsline{lof}{figure}{B.\arabic{counter_img_B}. Mock-up US 6 (editar depósito).}
	\stepcounter{counter_img_B}
	\label{us06_mockup_edit}
\end{figure}

\begin{figure}[H] 
	\centering
	\subfloat[Escritorio \hyperlink{US}{US} 6 (editar).]{%
		\includegraphics[width=0.73\textwidth]{pantallas/us06_desktop_edit.jpg}%
		\label{fig:a}%
	}%
	\hfill%
	\subfloat[Móvil \hyperlink{US}{US} 6 (editar).]{%
		\setlength{\fboxrule}{0.5pt}%
		\fbox{\includegraphics[width=0.24\textwidth]
			{pantallas/us06_mobile_edit.jpg}}%
		\label{fig:b}%
	}%
	\caption*{Figura B.\arabic{counter_img_B}: Versiones de escritorio y móvil de pantalla correspondiente a \hyperlink{US}{US} 6  (editar depósito).}
	\addcontentsline{lof}{figure}{B.\arabic{counter_img_B}. Versiones de escritorio y móvil de pantalla correspondiente a US 6 (editar depósito).}
	\stepcounter{counter_img_B}
\end{figure}


\subsection*{Administrar sucursales (\hyperlink{US}{US} 7)}

\begin{figure}[H]
	\centering
	{%
		\setlength{\fboxsep}{0pt}%
		\setlength{\fboxrule}{0.5pt}%
		\fbox{\includegraphics[scale = 0.47]{pantallas/us07_mockup_new.jpg}}%
	}%
	\caption*{Figura B.\arabic{counter_img_B}: Mock-up \hyperlink{US}{US} 7 (nueva sucursal).}
	\addcontentsline{lof}{figure}{B.\arabic{counter_img_B}. Mock-up US 7 (nueva sucursal).}
	\stepcounter{counter_img_B}
	\label{us07_mockup}
\end{figure}

\begin{figure}[H] 
	\centering
	\subfloat[Escritorio \hyperlink{US}{US} 7 (nueva).]{%
		\includegraphics[width=0.73\textwidth]{pantallas/us07_desktop_new.jpg}%
		\label{fig:a}%
	}%
	\hfill%
	\subfloat[Móvil \hyperlink{US}{US} 7 (nueva).]{%
		\setlength{\fboxrule}{0.5pt}%
		\fbox{\includegraphics[width=0.24\textwidth]
			{pantallas/us07_mobile_new.jpg}}%
		\label{fig:b}%
	}%
	\caption*{Figura B.\arabic{counter_img_B}: Versiones de escritorio y móvil de pantalla correspondiente a \hyperlink{US}{US} 7  (nueva sucursal).}
	\addcontentsline{lof}{figure}{B.\arabic{counter_img_B}. Versiones de escritorio y móvil de pantalla correspondiente a US 7  (nueva sucursal).}
	\stepcounter{counter_img_B}
\end{figure}

\begin{figure}[H]
	\centering
	{%
		\setlength{\fboxsep}{0pt}%
		\setlength{\fboxrule}{0.5pt}%
		\fbox{\includegraphics[scale = 0.47]{pantallas/us07_mockup_edit.jpg}}%
	}%
	\caption*{Figura B.\arabic{counter_img_B}: Mock-up \hyperlink{US}{US} 7 (editar sucursal).}
	\addcontentsline{lof}{figure}{B.\arabic{counter_img_B}. Mock-up US 7 (editar sucursal).}
	\stepcounter{counter_img_B}
	\label{us07_mockup_edit}
\end{figure}

\begin{figure}[H] 
	\centering
	\subfloat[Escritorio \hyperlink{US}{US} 7 (editar).]{%
		\includegraphics[width=0.73\textwidth]{pantallas/us07_desktop_edit.jpg}%
		\label{fig:a}%
	}%
	\hfill%
	\subfloat[Móvil \hyperlink{US}{US} 7 (editar).]{%
		\setlength{\fboxrule}{0.5pt}%
		\fbox{\includegraphics[width=0.24\textwidth]
			{pantallas/us07_mobile_edit.jpg}}%
		\label{fig:b}%
	}%
	\caption*{Figura B.\arabic{counter_img_B}: Versiones de escritorio y móvil de pantalla correspondiente a \hyperlink{US}{US} 7  (editar sucursal).}
	\addcontentsline{lof}{figure}{B.\arabic{counter_img_B}. Versiones de escritorio y móvil de pantalla correspondiente a US 7  (editar sucursal).}
	\stepcounter{counter_img_B}
\end{figure}


\subsection*{Administrar clientes (\hyperlink{US}{US} 18)}

\begin{figure}[H]
	\centering
	{%
		\setlength{\fboxsep}{0pt}%
		\setlength{\fboxrule}{0.5pt}%
		\fbox{\includegraphics[scale = 0.47]{pantallas/us18_mockup_new.jpg}}%
	}%
	\caption*{Figura B.\arabic{counter_img_B}: Mock-up \hyperlink{US}{US} 18 (nuevo cliente).}
	\addcontentsline{lof}{figure}{B.\arabic{counter_img_B}. Mock-up US 18 (nuevo cliente).}
	\stepcounter{counter_img_B}
	\label{us18_mockup}
\end{figure}

\begin{figure}[H] 
	\centering
	\subfloat[Escritorio \hyperlink{US}{US} 18 (nuevo).]{%
		\includegraphics[width=0.73\textwidth]{pantallas/us18_desktop_new.jpg}%
		\label{fig:a}%
	}%
	\hfill%
	\subfloat[Móvil \hyperlink{US}{US} 18 (nuevo).]{%
		\setlength{\fboxrule}{0.45pt}%
		\fbox{\includegraphics[width=0.24\textwidth]
			{pantallas/us18_mobile_new.jpg}}%
		\label{fig:b}%
	}%
	\caption*{Figura B.\arabic{counter_img_B}: Versiones de escritorio y móvil de pantalla correspondiente a \hyperlink{US}{US} 18  (nuevo cliente).}
	\addcontentsline{lof}{figure}{B.\arabic{counter_img_B}. Versiones de escritorio y móvil de pantalla correspondiente a US 18  (nuevo cliente).}
	\stepcounter{counter_img_B}
\end{figure}

\begin{figure}[H]
	\centering
	{%
		\setlength{\fboxsep}{0pt}%
		\setlength{\fboxrule}{0.5pt}%
		\fbox{\includegraphics[scale = 0.47]{pantallas/us18_mockup_edit.jpg}}%
	}%
	\caption*{Figura B.\arabic{counter_img_B}: Mock-up \hyperlink{US}{US} 18 (editar cliente).}
	\addcontentsline{lof}{figure}{B.\arabic{counter_img_B}. Mock-up US 18 (editar cliente).}
	\stepcounter{counter_img_B}
	\label{us18_mockup_edit}
\end{figure}

\begin{figure}[H] 
	\centering
	\subfloat[Escritorio \hyperlink{US}{US} 18 (editar).]{%
		\includegraphics[width=0.73\textwidth]{pantallas/us18_desktop_edit.jpg}%
		\label{fig:a}%
	}%
	\hfill%
	\subfloat[Móvil \hyperlink{US}{US} 18 (editar).]{%
		\setlength{\fboxrule}{0.5pt}%
		\fbox{\includegraphics[width=0.24\textwidth]
			{pantallas/us18_mobile_edit.jpg}}%
		\label{fig:b}%
	}%
	\caption*{Figura B.\arabic{counter_img_B}: Versiones de escritorio y móvil de pantalla correspondiente a \hyperlink{US}{US} 18  (editar cliente).}
	\addcontentsline{lof}{figure}{B.\arabic{counter_img_B}. Versiones de escritorio y móvil de pantalla correspondiente a US 18  (editar cliente).}
	\stepcounter{counter_img_B}
\end{figure}
\pagebreak

\section*{Iteración II}

\subsection*{Administrar insumo (\hyperlink{US}{US} 3)}

\begin{figure}[H]
	\centering
	{%
		\setlength{\fboxsep}{0pt}%
		\setlength{\fboxrule}{0.5pt}%
		\fbox{\includegraphics[scale = 0.4]{pantallas/us03_mockup_new.jpg}}%
	}%
	\caption*{Figura B.\arabic{counter_img_B}: Mock-up \hyperlink{US}{US} 3 (nuevo insumo).}
	\addcontentsline{lof}{figure}{B.\arabic{counter_img_B}. Mock-up US 3 (nuevo insumo).}
	\stepcounter{counter_img_B}
	\label{us01_mockup_new}
\end{figure}
\indent

\begin{figure}[H] 
	\centering
	\subfloat[Escritorio \hyperlink{US}{US} 3 (nuevo).]{%
		\includegraphics[width=0.73\textwidth]{pantallas/us03_desktop_new.jpg}%
		\label{fig:a}%
	}%
	\hfill%
	\subfloat[Móvil \hyperlink{US}{US} 3 (nuevo).]{%
		\setlength{\fboxrule}{0.5pt}%
		\fbox{\includegraphics[width=0.24\textwidth]
			{pantallas/us03_mobile_new.jpg}}%
		\label{fig:b}%
	}%
	\caption*{Figura B.\arabic{counter_img_B}: Versiones de escritorio y móvil de pantalla correspondiente a \hyperlink{US}{US} 3 (nuevo insumo).}
	\addcontentsline{lof}{figure}{B.\arabic{counter_img_B}. Versiones de escritorio y móvil de pantalla correspondiente a US 3 (nuevo insumo).}
	\stepcounter{counter_img_B}
\end{figure}

\begin{figure}[H]
	\centering
	{%
		\setlength{\fboxsep}{0pt}%
		\setlength{\fboxrule}{0.5pt}%
		\fbox{\includegraphics[scale = 0.5]{pantallas/us03_mockup_edit.jpg}}%
	}%
	\caption*{Figura B.\arabic{counter_img_B}: Mock-up \hyperlink{US}{US} 3 (editar insumo).}
	\addcontentsline{lof}{figure}{B.\arabic{counter_img_B}. Mock-up US 3 (editar insumo).}
	\stepcounter{counter_img_B}
	\label{us01_mockup}
\end{figure}

\begin{figure}[H] 
	\centering
	\subfloat[Escritorio \hyperlink{US}{US} 3 (editar).]{%
		\includegraphics[width=0.73\textwidth]{pantallas/us03_desktop_edit.jpg}%
		\label{fig:a}%
	}%
	\hfill%
	\subfloat[Móvil \hyperlink{US}{US} 3 (editar).]{%
		\setlength{\fboxrule}{0.5pt}%
		\fbox{\includegraphics[width=0.24\textwidth]
			{pantallas/us03_mobile_edit.jpg}}%
		\label{fig:b}%
	}%
	\caption*{Figura B.\arabic{counter_img_B}: Versiones de escritorio y móvil de pantalla correspondiente a \hyperlink{US}{US} 3 (editar insumo).}
	\addcontentsline{lof}{figure}{B.\arabic{counter_img_B}. Versiones de escritorio y móvil de pantalla correspondiente a US 3 (editar insumo).}
	\stepcounter{counter_img_B}
\end{figure}


\subsection*{Administrar tipo de insumo (\hyperlink{US}{US} 4)}

\begin{figure}[H]
	\centering
	{%
		\setlength{\fboxsep}{0pt}%
		\setlength{\fboxrule}{0.5pt}%
		\fbox{\includegraphics[scale = 0.43]{pantallas/us04_mockup_new.jpg}}%
	}%
	\caption*{Figura B.\arabic{counter_img_B}: Mock-up \hyperlink{US}{US} 4 (nuevo tipo de insumo).}
	\addcontentsline{lof}{figure}{B.\arabic{counter_img_B}. Mock-up US 4 (nuevo tipo de insumo).}
	\stepcounter{counter_img_B}
	\label{us04_mockup}
\end{figure}

\begin{figure}[H] 
	\centering
	\subfloat[Escritorio \hyperlink{US}{US} 4 (nuevo).]{%
		\includegraphics[width=0.73\textwidth]{pantallas/us04_desktop_new.jpg}%
		\label{fig:a}%
	}%
	\hfill%
	\subfloat[Móvil \hyperlink{US}{US} 4 (nuevo).]{%
		\setlength{\fboxrule}{0.5pt}%
		\fbox{\includegraphics[width=0.24\textwidth]
			{pantallas/us04_mobile_new.jpg}}%
		\label{fig:b}%
	}%
	\caption*{Figura B.\arabic{counter_img_B}: Versiones de escritorio y móvil de pantalla correspondiente a \hyperlink{US}{US} 4  (nuevo tipo de insumo).}
	\addcontentsline{lof}{figure}{B.\arabic{counter_img_B}. Versiones de escritorio y móvil de pantalla correspondiente a US 4  (nuevo tipo de insumo).}
	\stepcounter{counter_img_B}
\end{figure}

\begin{figure}[H]
	\centering
	{%
		\setlength{\fboxsep}{0pt}%
		\setlength{\fboxrule}{0.5pt}%
		\fbox{\includegraphics[scale = 0.47]{pantallas/us04_mockup_edit.jpg}}%
	}%
	\caption*{Figura B.\arabic{counter_img_B}: Mock-up \hyperlink{US}{US} 4 (editar tipo de insumo).}
	\addcontentsline{lof}{figure}{B.\arabic{counter_img_B}. Mock-up US 4 (editar tipo de insumo).}
	\stepcounter{counter_img_B}
	\label{us04_mockup_edit}
\end{figure}

\begin{figure}[H] 
	\centering
	\subfloat[Escritorio \hyperlink{US}{US} 4 (editar).]{%
		\includegraphics[width=0.73\textwidth]{pantallas/us04_desktop_edit.jpg}%
		\label{fig:a}%
	}%
	\hfill%
	\subfloat[Móvil \hyperlink{US}{US} 4 (editar).]{%
		\setlength{\fboxrule}{0.5pt}%
		\fbox{\includegraphics[width=0.24\textwidth]
			{pantallas/us04_mobile_edit.jpg}}%
		\label{fig:b}%
	}%
	\caption*{Figura B.\arabic{counter_img_B}: Versiones de escritorio y móvil de pantalla correspondiente a \hyperlink{US}{US} 4  (editar tipo de insumo).}
	\addcontentsline{lof}{figure}{B.\arabic{counter_img_B}. Versiones de escritorio y móvil de pantalla correspondiente a US 4  (editar tipo de insumo).}
	\stepcounter{counter_img_B}
\end{figure}


\subsection*{Modificar cantidad de insumo (\hyperlink{US}{US} 5)}

\begin{figure}[H]
	\centering
	{%
		\setlength{\fboxsep}{0pt}%
		\setlength{\fboxrule}{0.5pt}%
		\fbox{\includegraphics[scale = 0.45]{pantallas/us05_mockup_new.jpg}}%
	}%
	\caption*{Figura B.\arabic{counter_img_B}: Mock-up \hyperlink{US}{US} 5 (nuevo insumo en depósito).}
	\addcontentsline{lof}{figure}{B.\arabic{counter_img_B}. Mock-up US 5 (nuevo insumo en depósito).}
	\stepcounter{counter_img_B}
	\label{us05_mockup}
\end{figure}

\begin{figure}[H] 
	\centering
	\subfloat[Escritorio \hyperlink{US}{US} 5 (nuevo).]{%
		\includegraphics[width=0.73\textwidth]{pantallas/us05_desktop_new.jpg}%
		\label{fig:a}%
	}%
	\hfill%
	\subfloat[Móvil \hyperlink{US}{US} 5 (nuevo).]{%
		\setlength{\fboxrule}{0.5pt}%
		\fbox{\includegraphics[width=0.24\textwidth]
			{pantallas/us05_mobile_new.jpg}}%
		\label{fig:b}%
	}%
	\caption*{Figura B.\arabic{counter_img_B}: Versiones de escritorio y móvil de pantalla correspondiente a \hyperlink{US}{US} 5  (nuevo insumo en depósito).}
	\addcontentsline{lof}{figure}{B.\arabic{counter_img_B}. Versiones de escritorio y móvil de pantalla correspondiente a US 5  (nuevo insumo en depósito).}
	\stepcounter{counter_img_B}
\end{figure}

\begin{figure}[H]
	\centering
	{%
		\setlength{\fboxsep}{0pt}%
		\setlength{\fboxrule}{0.5pt}%
		\fbox{\includegraphics[scale = 0.47]{pantallas/us05_mockup_edit.jpg}}%
	}%
	\caption*{Figura B.\arabic{counter_img_B}: Mock-up \hyperlink{US}{US} 5 (editar cantidad de insumo en depósito).}
	\addcontentsline{lof}{figure}{B.\arabic{counter_img_B}. Mock-up US 5 (editar cantidad de insumo en depósito).}
	\stepcounter{counter_img_B}
	\label{us05_mockup_edit}
\end{figure}

\begin{figure}[H]
	\centering
	{\includegraphics[scale = 0.65]{pantallas/us05_desktop_edit.jpg}%
	}%
	\caption*{Figura B.\arabic{counter_img_B}: Versión de escritorio de pantalla correspondiente a \hyperlink{US}{US} 5  (editar cantidad de insumo en depósito).}
	\addcontentsline{lof}{figure}{B.\arabic{counter_img_B}. Versión de escritorio de pantalla correspondiente a US 5  (editar cantidad de insumo en depósito).}
	\stepcounter{counter_img_B}
	\label{us23_mockup_new}
\end{figure}


\subsection*{Administrar tipo de tarea (\hyperlink{US}{US} 23)}

\begin{figure}[H]
	\centering
	{%
		\setlength{\fboxsep}{0pt}%
		\setlength{\fboxrule}{0.5pt}%
		\fbox{\includegraphics[scale = 0.45]{pantallas/us23_mockup_new.jpg}}%
	}%
	\caption*{Figura B.\arabic{counter_img_B}: Mock-up \hyperlink{US}{US} 23 (nuevo tipo de tarea).}
	\addcontentsline{lof}{figure}{B.\arabic{counter_img_B}. Mock-up US 23 (nuevo tipo de tarea).}
	\stepcounter{counter_img_B}
	\label{us23_mockup_new}
\end{figure}

\begin{figure}[H]
	\centering
	{\includegraphics[scale = 0.60]{pantallas/us23_desktop_new.jpg}%
	}%
	\caption*{Figura B.\arabic{counter_img_B}: Versión de escritorio de pantalla correspondiente a \hyperlink{US}{US} 23  (nuevo tipo de tarea).}
	\addcontentsline{lof}{figure}{B.\arabic{counter_img_B}. Versión de escritorio de pantalla correspondiente a US 23  (nuevo tipo de tarea).}
	\stepcounter{counter_img_B}
	\label{us23_mockup}
\end{figure}

\begin{figure}[H]
	\centering
	{%
		\setlength{\fboxsep}{0pt}%
		\setlength{\fboxrule}{0.5pt}%
		\fbox{\includegraphics[scale = 0.47]{pantallas/us23_mockup_list.jpg}}%
	}%
	\caption*{Figura B.\arabic{counter_img_B}: Mock-up \hyperlink{US}{US} 23 (lista de tipos de tarea).}
	\addcontentsline{lof}{figure}{B.\arabic{counter_img_B}. Mock-up US 23 (lista de tipos de tarea).}
	\stepcounter{counter_img_B}
	\label{us23_mockup}
\end{figure}

\begin{figure}[H] 
	\centering
	\subfloat[Escritorio \hyperlink{US}{US} 23 (lista).]{%
		\includegraphics[width=0.73\textwidth]{pantallas/us23_desktop_list.jpg}%
		\label{fig:a}%
	}%
	\hfill%
	\subfloat[Móvil \hyperlink{US}{US} 23 (lista).]{%
		\setlength{\fboxrule}{0.5pt}%
		\fbox{\includegraphics[width=0.24\textwidth]
			{pantallas/us23_mobile_list.jpg}}%
		\label{fig:b}%
	}%
	\caption*{Figura B.\arabic{counter_img_B}: Versiones de escritorio y móvil de pantalla correspondiente a \hyperlink{US}{US} 23  (lista de tipos de tarea).}
	\addcontentsline{lof}{figure}{B.\arabic{counter_img_B}. Versiones de escritorio y móvil de pantalla correspondiente a US 23  (lista de tipos de tarea).}
	\stepcounter{counter_img_B}
\end{figure}

\begin{figure}[H] 
	\centering
	\subfloat[Móvil \hyperlink{US}{US} 23 (editar 1).]{%
		\setlength{\fboxrule}{0.5pt}%
		\fbox{\includegraphics[width=0.24\textwidth]
			{pantallas/us23_mobile_edit_1.jpg}}%
		\label{fig:b}%
	}%
	\hspace{0.5cm}%
	\subfloat[Móvil \hyperlink{US}{US} 23 (editar 2).]{%
		\setlength{\fboxrule}{0.5pt}%
		\fbox{\includegraphics[width=0.24\textwidth]
			{pantallas/us23_mobile_edit_2.jpg}}%
		\label{fig:b}%
	}%
	\caption*{Figura B.\arabic{counter_img_B}: Versión móvil de pantalla correspondiente a \hyperlink{US}{US} 23  (editar tipo de tarea).}
	\addcontentsline{lof}{figure}{B.\arabic{counter_img_B}. Versión móvil de pantalla correspondiente a US 23  (editar tipo de tarea).}
	\stepcounter{counter_img_B}
\end{figure}

\subsection*{Visualizar novedades (\hyperlink{US}{US} 24)}

\begin{figure}[H]
	\centering
	{%
		\setlength{\fboxsep}{0pt}%
		\setlength{\fboxrule}{0.5pt}%
		\fbox{\includegraphics[scale = 0.47]{pantallas/us24_mockup_list.jpg}}%
	}%
	\caption*{Figura B.\arabic{counter_img_B}: Mock-up \hyperlink{US}{US} 24.}
	\addcontentsline{lof}{figure}{B.\arabic{counter_img_B}. Mock-up US 24.}
	\stepcounter{counter_img_B}
	\label{us24_mockup}
\end{figure}

\begin{figure}[H] 
	\centering
	\subfloat[Escritorio \hyperlink{US}{US} 24.]{%
		\includegraphics[width=0.73\textwidth]{pantallas/us24_desktop_list.jpg}%
		\label{fig:a}%
	}%
	\hfill%
	\subfloat[Móvil \hyperlink{US}{US} 24.]{%
		\setlength{\fboxrule}{0.45pt}%
		\fbox{\includegraphics[width=0.24\textwidth]
			{pantallas/us24_mobile_list.jpg}}%
		\label{fig:b}%
	}%
	\caption*{Figura B.\arabic{counter_img_B}: Versiones de escritorio y móvil de pantalla correspondiente a \hyperlink{US}{US} 24.}
	\addcontentsline{lof}{figure}{B.\arabic{counter_img_B}. Versiones de escritorio y móvil de pantalla correspondiente a US 24.}
	\stepcounter{counter_img_B}
\end{figure}


\section*{Iteración III}

\subsection*{Administrar proyecto (\hyperlink{US}{US} 8)}

\begin{figure}[H] 
	\centering
	\subfloat[Mock-up \hyperlink{US}{US} 8 (nuevo proyecto 1).]{%
		\setlength{\fboxsep}{0pt}%
		\setlength{\fboxrule}{0.5pt}%
		\fbox{\includegraphics[width=0.53\textwidth]{pantallas/us08_mockup_new_1.jpg}}%
		\label{fig:a}%
	}%
	\hfill%
	\subfloat[Mock-up \hyperlink{US}{US} 8 (nuevo proyecto 2).]{%
		\setlength{\fboxsep}{0pt}%
		\setlength{\fboxrule}{0.5pt}%
		\fbox{\includegraphics[width=0.53\textwidth]
			{pantallas/us08_mockup_new_2.jpg}}%
		\label{fig:b}%
	}%
	\caption*{Figura B.\arabic{counter_img_B}: Mock-ups correspondientes a \hyperlink{US}{US} 8 (nuevo proyecto).}
	\addcontentsline{lof}{figure}{B.\arabic{counter_img_B}. Mock-ups correspondientes a US 8 (nuevo proyecto).}
	\stepcounter{counter_img_B}
\end{figure}

\begin{figure}[H] 
	\centering
	\subfloat[Escritorio \hyperlink{US}{US} 8 (nuevo 1).]{%
		\includegraphics[width=0.73\textwidth]{pantallas/us08_desktop_new_1.jpg}%
		\label{fig:a}%
	}%
	\hfill%
	\subfloat[Escritorio \hyperlink{US}{US} 8 (nuevo 2).]{%
		\includegraphics[width=0.73\textwidth]
		{pantallas/us08_desktop_new_2.jpg}%
		\label{fig:b}%
	}%
	\caption*{Figura B.\arabic{counter_img_B}: Versión de escritorio de pantallas correspondientes a \hyperlink{US}{US} 8 (nuevo proyecto).}
	\addcontentsline{lof}{figure}{B.\arabic{counter_img_B}. Versión de escritorio de pantallas correspondientes a US 8 (nuevo proyecto).}
	\stepcounter{counter_img_B}
\end{figure}

\begin{figure}[H] 
	\centering
	\subfloat[Móvil \hyperlink{US}{US} 8 (nuevo 1).]{%
		\fbox{\includegraphics[width=0.24\textwidth]{pantallas/us08_mobile_new_1.jpg}%
			\label{fig:a}%
	}}%
	\hspace{0.5cm}
	\subfloat[Móvil \hyperlink{US}{US} 8 (nuevo 2).]{%
		\fbox{\includegraphics[width=0.24\textwidth]
			{pantallas/us08_mobile_new_2.jpg}%
			\label{fig:b}%
	}}%
	\caption*{Figura B.\arabic{counter_img_B}: Versión móvil de pantallas correspondientes a \hyperlink{US}{US} 8 (nuevo proyecto).}
	\addcontentsline{lof}{figure}{B.\arabic{counter_img_B}. Versión móvil de pantallas correspondientes a US 8 (nuevo proyecto).}
	\stepcounter{counter_img_B}
\end{figure}


\subsection*{Administrar tipos de proyectos (\hyperlink{US}{US} 10)}

\begin{figure}[H]
	\centering
	{%
		\setlength{\fboxsep}{0pt}%
		\setlength{\fboxrule}{0.5pt}%
		\fbox{\includegraphics[scale = 0.45]{pantallas/us10_mockup_new.jpg}}%
	}%
	\caption*{Figura B.\arabic{counter_img_B}: Mock-up \hyperlink{US}{US} 10 (nuevo tipo de proyecto).}
	\addcontentsline{lof}{figure}{B.\arabic{counter_img_B}. Mock-up US 10 (nuevo tipo de proyecto).}
	\stepcounter{counter_img_B}
	\label{us10_mockup_new}
\end{figure}

\begin{figure}[H]
	\centering
	{\includegraphics[scale = 0.60]{pantallas/us10_desktop_new.jpg}%
	}%
	\caption*{Figura B.\arabic{counter_img_B}: Versión de escritorio de pantalla correspondiente a \hyperlink{US}{US} 10 (nuevo tipo de proyecto).}
	\addcontentsline{lof}{figure}{B.\arabic{counter_img_B}. Versión de escritorio de pantalla correspondiente a US 10 (nuevo tipo de proyecto).}
	\stepcounter{counter_img_B}
	\label{us01_mockup}
\end{figure}

\begin{figure}[H]
	\centering
	{%
		\setlength{\fboxsep}{0pt}%
		\setlength{\fboxrule}{0.5pt}%
		\fbox{\includegraphics[scale = 0.5]{pantallas/us10_mockup_list.jpg}}%
	}%
	\caption*{Figura B.\arabic{counter_img_B}: Mock-up \hyperlink{US}{US} 10 (lista de tipos de proyecto).}
	\addcontentsline{lof}{figure}{B.\arabic{counter_img_B}. Mock-up US 10 (lista de tipos de proyecto).}
	\stepcounter{counter_img_B}
	\label{us01_mockup}
\end{figure}

\begin{figure}[H]
	\centering
	{\includegraphics[scale = 0.6]{pantallas/us10_desktop_list.jpg}%
	}%
	\caption*{Figura B.\arabic{counter_img_B}: Versión de escritorio de pantalla correspondiente a \hyperlink{US}{US} 10 (lista de tipos de proyecto).}
	\addcontentsline{lof}{figure}{B.\arabic{counter_img_B}. Versión de escritorio de pantalla correspondiente a US 10 (lista de tipos de proyecto).}
	\stepcounter{counter_img_B}
	\label{us10_desktop_list}
\end{figure}


\subsection*{Asignar tarea a empleado (\hyperlink{US}{US} 12)}

\begin{figure}[H]
	\centering
	{%
		\setlength{\fboxsep}{0pt}%
		\setlength{\fboxrule}{0.5pt}%
		\fbox{\includegraphics[scale = 0.5]{pantallas/us12_mockup_list.jpg}}%
	}%
	\caption*{Figura B.\arabic{counter_img_B}: Mock-up \hyperlink{US}{US} 12.}
	\addcontentsline{lof}{figure}{B.\arabic{counter_img_B}. Mock-up US 12.}
	\stepcounter{counter_img_B}
	\label{us10_mockup_new}
\end{figure}

\begin{figure}[H] 
	\centering
	\subfloat[Escritorio \hyperlink{US}{US} 12.]{%
		\includegraphics[width=0.73\textwidth]{pantallas/us12_desktop_list.jpg}%
		\label{fig:a}%
	}%
	\hfill%
	\subfloat[Móvil \hyperlink{US}{US} 12.]{%
		\setlength{\fboxrule}{0.5pt}%
		\fbox{\includegraphics[width=0.24\textwidth]
			{pantallas/us12_mobile_list.jpg}}%
		\label{fig:b}%
	}%
	\caption*{Figura B.\arabic{counter_img_B}: Versiones de escritorio y móvil de pantalla correspondiente a \hyperlink{US}{US} 12.}
	\addcontentsline{lof}{figure}{B.\arabic{counter_img_B}. Versiones de escritorio y móvil de pantalla correspondiente a US 12.}
	\stepcounter{counter_img_B}
\end{figure}

\subsection*{Buscar proyecto (\hyperlink{US}{US} 16)}
\label{mockups_us_16}

\begin{figure}[H]
	\centering
	{%
		\setlength{\fboxsep}{0pt}%
		\setlength{\fboxrule}{0.5pt}%
		\fbox{\includegraphics[scale = 0.45]{pantallas/us16_mockup.jpg}}%
	}%
	\caption*{Figura B.\arabic{counter_img_B}: Mock-up \hyperlink{US}{US} 16.}
	\addcontentsline{lof}{figure}{B.\arabic{counter_img_B}. Mock-up US 16.}
	\stepcounter{counter_img_B}
	\label{us16_mockup_new}
\end{figure}

\begin{figure}[H] 
	\centering
	\subfloat[Escritorio \hyperlink{US}{US} 16.]{%
		\includegraphics[width=0.73\textwidth]{pantallas/us16_desktop.jpg}%
		\label{fig:a}%
	}%
	\hfill%
	\subfloat[Móvil \hyperlink{US}{US} 16 (1).]{%
		\setlength{\fboxrule}{0.5pt}%
		\fbox{\includegraphics[width=0.24\textwidth]
			{pantallas/us16_mobile_1.jpg}}%
		\label{fig:b}%
	}%
	\hfill%
	\subfloat[Móvil \hyperlink{US}{US} 16 (2).]{%
		\setlength{\fboxrule}{0.5pt}%
		\fbox{\includegraphics[width=0.24\textwidth]
			{pantallas/us16_mobile_2.jpg}}%
		\label{fig:b}%
	}%
	\caption*{Figura B.\arabic{counter_img_B}: Versiones de escritorio y móvil de pantallas correspondientes a \hyperlink{US}{US} 16.}
	\addcontentsline{lof}{figure}{B.\arabic{counter_img_B}. Versiones de escritorio y móvil de pantallas correspondientes a US 16.}
	\stepcounter{counter_img_B}
\end{figure}

\subsection*{Visualizar tareas asignadas (\hyperlink{US}{US} 17)}

\begin{figure}[H]
	\centering
	{%
		\setlength{\fboxsep}{0pt}%
		\setlength{\fboxrule}{0.5pt}%
		\fbox{\includegraphics[scale = 0.45]{pantallas/us17_mockup_list.jpg}}%
	}%
	\caption*{Figura B.\arabic{counter_img_B}: Mock-up \hyperlink{US}{US} 17 (lista de tareas asignadas).}
	\addcontentsline{lof}{figure}{B.\arabic{counter_img_B}. Mock-up US 17 (lista de tareas asignadas).}
	\stepcounter{counter_img_B}
	\label{us17_mockup_list}
\end{figure}

\begin{figure}[H] 
	\centering
	\subfloat[Escritorio \hyperlink{US}{US} 17 (lista).]{%
		\includegraphics[width=0.73\textwidth]{pantallas/us17_desktop_list.jpg}%
		\label{fig:a}%
	}%
	\hfill%
	\subfloat[Móvil \hyperlink{US}{US} 17 (lista).]{%
		\setlength{\fboxrule}{0.5pt}%
		\fbox{\includegraphics[width=0.24\textwidth]
			{pantallas/us17_mobile_list.jpg}}%
		\label{fig:b}%
	}%
	\caption*{Figura B.\arabic{counter_img_B}: Versiones de escritorio y móvil de pantalla correspondiente a \hyperlink{US}{US} 17 (lista de tareas asignadas).}
	\addcontentsline{lof}{figure}{B.\arabic{counter_img_B}. Versiones de escritorio y móvil de pantalla correspondiente a US 17 (lista de tareas asignadas).}
	\stepcounter{counter_img_B}
\end{figure}

\begin{figure}[H]
	\centering
	{%
		\setlength{\fboxsep}{0pt}%
		\setlength{\fboxrule}{0.5pt}%
		\fbox{\includegraphics[scale = 0.5]{pantallas/us17_mockup_detail.png}}%
	}%
	\caption*{Figura B.\arabic{counter_img_B}: Mock-up \hyperlink{US}{US} 17 (detalle de tarea asignada).}
	\addcontentsline{lof}{figure}{B.\arabic{counter_img_B}. Mock-up US 17 (detalle de tarea asignada).}
	\stepcounter{counter_img_B}
	\label{us17_mockup_detail}
\end{figure}

\begin{figure}[H] 
	\centering
	\subfloat[Escritorio \hyperlink{US}{US} 17 (detalle).]{%
		\includegraphics[width=0.73\textwidth]{pantallas/us17_desktop_detail.jpg}%
		\label{fig:a}%
	}%
	\hfill%
	\subfloat[Móvil \hyperlink{US}{US} 17 (detalle).]{%
		\setlength{\fboxrule}{0.5pt}%
		\fbox{\includegraphics[width=0.24\textwidth]
			{pantallas/us17_mobile_detail.jpg}}%
		\label{fig:b}%
	}%
	\caption*{Figura B.\arabic{counter_img_B}: Versiones de escritorio y móvil de pantalla correspondiente a \hyperlink{US}{US} 17 (detalle de tarea asignada).}
	\addcontentsline{lof}{figure}{B.\arabic{counter_img_B}. Versiones de escritorio y móvil de pantalla correspondiente a US 17 (detalle de tarea asignada).}
	\stepcounter{counter_img_B}
\end{figure}
\pagebreak


\section*{Iteración IV}

\subsection*{Agregar insumo a proyecto (\hyperlink{US}{US} 9)}

\begin{figure}[H]
	\centering
	{%
		\setlength{\fboxsep}{0pt}%
		\setlength{\fboxrule}{0.5pt}%
		\fbox{\includegraphics[scale = 0.45]{pantallas/us09. Agregar insumo a proyecto.png}}%
	}%
	\caption*{Figura B.\arabic{counter_img_B}: Mock-up \hyperlink{US}{US} 9.}
	\addcontentsline{lof}{figure}{B.\arabic{counter_img_B}. Mock-up US 9 (agregar insumo a proyecto).}
	\stepcounter{counter_img_B}
	\label{us09_mockup}
\end{figure}

\begin{figure}[H]
	\centering
	{\includegraphics[scale = 0.55]{pantallas/us9_desktop.png}%
	}%
	\caption*{Figura B.\arabic{counter_img_B}: Versión de escritorio de pantalla correspondiente a \hyperlink{US}{US} 9.}
	\addcontentsline{lof}{figure}{B.\arabic{counter_img_B}. Versión de escritorio de pantalla correspondiente a US 9.}
	\stepcounter{counter_img_B}
	\label{us09_desktop}
\end{figure}

\begin{figure}[H] 
	\centering
	\subfloat[Móvil \hyperlink{US}{US} 9 (1).]{%
		\setlength{\fboxrule}{0.5pt}%
		\fbox{\includegraphics[width=0.24\textwidth]
			{pantallas/us9_mobile_1.png}}%
		\label{fig:a}%
	}%
	\hspace{0.5cm}
	\subfloat[Móvil \hyperlink{US}{US} 9 (2).]{%
		\setlength{\fboxrule}{0.5pt}%
		\fbox{\includegraphics[width=0.24\textwidth]
			{pantallas/us9_mobile_2.png}}%
		\label{fig:b}%
	}%
	\hspace{0.5cm}
	\subfloat[Móvil \hyperlink{US}{US} 9 (3).]{%
		\setlength{\fboxrule}{0.5pt}%
		\fbox{\includegraphics[width=0.24\textwidth]
			{pantallas/us9_mobile_3.png}}%
		\label{fig:c}%
	}%
	\caption*{Figura B.\arabic{counter_img_B}: Versión móvil de pantallas correspondientes a \hyperlink{US}{US} 9.}
	\addcontentsline{lof}{figure}{B.\arabic{counter_img_B}. Versión móvil de pantallas correspondientes a US 9.}
	\stepcounter{counter_img_B}
\end{figure}


\subsection*{Editar tarea (\hyperlink{US}{US} 11)}

\begin{figure}[H]
	\centering
	{%
		\setlength{\fboxsep}{0pt}%
		\setlength{\fboxrule}{0.5pt}%
		\fbox{\includegraphics[scale = 0.45]{pantallas/us11. Editar tarea.png}}%
	}%
	\caption*{Figura B.\arabic{counter_img_B}: Mock-up \hyperlink{US}{US} 11.}
	\addcontentsline{lof}{figure}{B.\arabic{counter_img_B}. Mock-up US 11 (editar tarea).}
	\stepcounter{counter_img_B}
	\label{us11_mockup}
\end{figure}

\begin{figure}[H] 
	\centering
	\subfloat[Escritorio \hyperlink{US}{US} 11.]{%
		\includegraphics[width=0.73\textwidth]{pantallas/us11_desktop.png}%
		\label{fig:a}%
	}%
	\hfill%
	\subfloat[Móvil \hyperlink{US}{US} 11.]{%
		\setlength{\fboxrule}{0.5pt}%
		\fbox{\includegraphics[width=0.24\textwidth]
			{pantallas/us11_mobile.png}}%
		\label{fig:b}%
	}%
	\caption*{Figura B.\arabic{counter_img_B}: Versiones de escritorio y móvil de pantalla correspondiente a \hyperlink{US}{US} 11.}
	\addcontentsline{lof}{figure}{B.\arabic{counter_img_B}. Versiones de escritorio y móvil de pantalla correspondiente a US 11.}
	\stepcounter{counter_img_B}
\end{figure}


\subsection*{Agregar comentario a tarea (\hyperlink{US}{US} 13)}

% Congelamos el valor actual del contador antes de incrementarlo
\edef\imgBnumAux{\arabic{counter_img_B}}

\begin{figure}[H] 
	\centering
	\subfloat[Mock-up \hyperlink{US}{US} 13 (vista gerente).]{%
		\setlength{\fboxsep}{0pt}%
		\setlength{\fboxrule}{0.5pt}%
		\fbox{\includegraphics[width=0.47\textwidth]{pantallas/us13. Agregar comentario a tarea [gerente].png}}%
		\phantomsection\label{fig:mockup-us13a}%
	}%
	\hspace{0.1cm}
	\subfloat[Mock-up \hyperlink{US}{US} 13 (vista usuario asignado).]{%
		\setlength{\fboxsep}{0pt}%
		\setlength{\fboxrule}{0.5pt}%
		\fbox{\includegraphics[width=0.47\textwidth]
			{pantallas/us13. Agregar comentario a tarea [usuario asignado].png}}%
		\phantomsection\label{fig:mockup-us13b}%
	}%
	
	% Usamos la variable auxiliar en el caption
	\caption*{Figura B.\imgBnumAux: Mock-ups correspondientes a \hyperlink{US}{US} 13.}
	\addcontentsline{lof}{figure}{B.\imgBnumAux. Mock-ups correspondientes a US 13.}
	
	% Incrementamos después
	\stepcounter{counter_img_B}
\end{figure}

% Comandos para referenciar en el texto
\newcommand{\refMockupGerente}{\hyperref[fig:mockup-us13a]{Figura B.\imgBnumAux a}}
\newcommand{\refMockupUsuario}{\hyperref[fig:mockup-us13b]{Figura B.\imgBnumAux b}}

\begin{figure}[H] 
	\centering
	\subfloat[Escritorio \hyperlink{US}{US} 13.]{%
		\includegraphics[width=0.73\textwidth]{pantallas/us13_desktop.png}%
		\label{fig:a}%
	}%
	\hfill%
	\subfloat[Móvil \hyperlink{US}{US} 13.]{%
		\setlength{\fboxrule}{0.5pt}%
		\fbox{\includegraphics[width=0.24\textwidth]
			{pantallas/us11_mobile.png}}%
		\label{fig:b}%
	}%
	\caption*{Figura B.\arabic{counter_img_B}: Versiones de escritorio y móvil de pantalla correspondiente a \hyperlink{US}{US} 13.}
	\addcontentsline{lof}{figure}{B.\arabic{counter_img_B}. Versiones de escritorio y móvil de pantalla correspondiente a US 13.}
	\stepcounter{counter_img_B}
\end{figure}

\pagebreak


\subsection*{Modificar estado de tarea (\hyperlink{US}{US} 14)}

Los mock-ups correspondientes a la \hyperlink{US}{US} 14 son los mismos que los mostrados para la \hyperlink{US}{US} 13. Véase figuras \refMockupGerente{} y \refMockupUsuario{}.

\begin{figure}[H] 
	\centering
	\subfloat[Escritorio \hyperlink{US}{US} 14 (vista gerente).]{%
		\includegraphics[width=0.73\textwidth]{pantallas/us14_desktop [vista gerente].png}%
		\label{fig:a}%
	}%
	\hfill%
	\subfloat[Escritorio \hyperlink{US}{US} 14 (vista usuario asignado).]{%
		\includegraphics[width=0.73\textwidth]
		{pantallas/us14_desktop [vista usuario asignado].png}%
		\label{fig:b}%
	}%
	\caption*{Figura B.\arabic{counter_img_B}: Versión de escritorio de pantallas correspondientes a \hyperlink{US}{US} 14.}
	\addcontentsline{lof}{figure}{B.\arabic{counter_img_B}. Versión de escritorio de pantallas correspondientes a US 14.}
	\stepcounter{counter_img_B}
\end{figure}

\begin{figure}[H]
	\centering
	\subfloat[Móvil \hyperlink{US}{US} 14 (vista gerente).]{%
		\fbox{\includegraphics[width=0.24\textwidth]{pantallas/us14_mobile [vista gerente].png}%
			\label{fig:a}%
	}}%
	\hspace{0.5cm}
	\subfloat[Móvil \hyperlink{US}{US} 14 (vista usuario asignado).]{%
		\fbox{\includegraphics[width=0.24\textwidth]
			{pantallas/us14_mobile [vista usuario asignado].png}%
			\label{fig:b}%
	}}%
	\caption*{Figura B.\arabic{counter_img_B}: Versión móvil de pantallas correspondientes a \hyperlink{US}{US} 14.}
	\addcontentsline{lof}{figure}{B.\arabic{counter_img_B}. Versión móvil de pantallas correspondientes a US 14.}
	\stepcounter{counter_img_B}
\end{figure}


\subsection*{Finalizar proyecto (\hyperlink{US}{US} 15) y Visualizar proyecto (\hyperlink{US}{US} 25)}

\begin{figure}[H]
	\centering
	{%
		\setlength{\fboxsep}{0pt}%
		\setlength{\fboxrule}{0.5pt}%
		\fbox{\includegraphics[scale = 0.45]{pantallas/us15-us25. Visualizar proyecto.png}}%
	}%
	\caption*{Figura B.\arabic{counter_img_B}: Mock-up \hyperlink{US}{US} 15 y 25.}
	\addcontentsline{lof}{figure}{B.\arabic{counter_img_B}. Mock-up US 15 y 25.}
	\stepcounter{counter_img_B}
	\label{us15_mockup}
\end{figure}

\begin{figure}[H] 
	\centering
	\subfloat[Escritorio \hyperlink{US}{US} 15 y 25 (1).]{%
		\includegraphics[width=0.73\textwidth]{pantallas/us15-25_desktop.png}%
		\label{fig:a}%
	}%
	\hfill%
	\subfloat[Escritorio \hyperlink{US}{US} 15 y 25 (2).]{%
		\includegraphics[width=0.73\textwidth]
		{pantallas/us15-25_desktop [iniciado].png}%
		\label{fig:b}%
	}%
	\caption*{Figura B.\arabic{counter_img_B}: Versión de escritorio de pantallas correspondientes a \hyperlink{US}{US} 15 y 25.}
	\addcontentsline{lof}{figure}{B.\arabic{counter_img_B}. Versión de escritorio de pantallas correspondientes a US 15 y 25.}
	\stepcounter{counter_img_B}
\end{figure}

\begin{figure}[H]
	\centering
	\subfloat[Móvil \hyperlink{US}{US} 15 y 25.]{%
		\fbox{\includegraphics[width=0.24\textwidth]{pantallas/us15-25_mobile_1.png}%
			\label{fig:a}%
	}}%
	\hspace{0.5cm}
	\subfloat[Móvil \hyperlink{US}{US} 15 y 25.]{%
		\fbox{\includegraphics[width=0.24\textwidth]
			{pantallas/us15-25_mobile_2.png}%
			\label{fig:b}%
	}}%
	\caption*{Figura B.\arabic{counter_img_B}: Versión móvil de pantallas correspondientes a \hyperlink{US}{US} 15 y 25.}
	\addcontentsline{lof}{figure}{B.\arabic{counter_img_B}. Versión móvil de pantallas correspondientes a US 15 y 25.}
	\stepcounter{counter_img_B}
\end{figure}


\section*{Iteración V}
Ya que las \hyperlink{US}{US} pertenecientes a la iteración V tienen relación con la eliminación de instancias y la generación de un reporte en el caso particular de la \hyperlink{US}{US} 31, no fue necesario realizar pantallas adicionales. Por lo tanto, el desarrollo fue hecho utilizando pantallas ya presentadas para las iteraciones precedentes.



\chapter*{\hypertarget{apendice_c}{}Apéndice C\\Objetivos del proyecto}
\markboth{Apéndice C - Objetivos del proyecto}{Apéndice C - Objetivos del proyecto}
\sectionmark{Apéndice C - Objetivos del proyecto}
\addcontentsline{toc}{chapter}{C. Objetivos del proyecto}

\appendix
\renewcommand{\thesection}{C.\arabic{section}}

\section{Objetivos generales}
\label{objetivos_generales}
Implementar un sistema de información destinado al control del inventario de insumos así como de proyectos de producción desarrollados en Imprenta Lux S.A.

\section{Objetivos específicos}
\label{objetivos_especificos}
\begin{itemize}
	\item Describir las tareas y procesos concernientes al actual funcionamiento de la empresa, principalmente en lo que refiere a las medidas y acciones llevadas a cabo en el control de inventarios y de la producción.
	\item Generar una propuesta de roles y responsabilidades que se ajuste adecuadamente a las tareas a realizar por los usuarios, y que tienda a minimizar la resistencia de estos a utilizar la nueva tecnología a implementar.
	\item Identificar los puntos de interdependencia entre los procesos de control de inventarios y de la producción con el fin de integrarlos de manera idónea dentro de la solución planteada.
	\item Establecer el alcance y los requerimientos de la solución planteada de acuerdo con los conocimientos adquiridos e información recopilada teniendo como base fundamental las expectativas del cliente.
	\item Desarrollar un producto de software basado en los requerimientos y características acordadas con el cliente que abarque, dentro de lo posible, las distintas etapas de control de los procesos productivos relacionándolos asimismo con el control de insumos, con el fin de aumentar la eficiencia en las actividades que a estos se relacionan.
\end{itemize}



\chapter*{\hypertarget{apendice_d}{}Apéndice D\\Plan de gestión de riesgos}
\markboth{Apéndice D - Plan de gestión de riesgos}{Apéndice D - Plan de gestión de riesgos}
\sectionmark{Apéndice D - Plan de gestión de riesgos}
\addcontentsline{toc}{chapter}{D. Plan de gestión de riesgos}

%%%%%%%%%%%%%%%%%%%%%%%%%%%%%%%%%%%%%%%%%
\newcounter{counter_img_D}  \setcounter{counter_img_D}{1}
\newcounter{counter_tbl_D}  \setcounter{counter_tbl_D}{1}
%%%%%%%%%%%%%%%%%%%%%%%%%%%%%%%%%%%%%%%%%

\indent Analizando las variables costo, tiempo, calidad, alcance, se pueden inferir las siguientes relaciones entre el tiempo y las restantes:
\begin{itemize}
	\item Incrementar la calidad incrementa el tiempo requerido debido a una mayor cantidad de testeos necesarios en las funcionalidades desarrolladas para asegurar la calidad deseada.
	\item El costo en este caso no es una variable a tener en cuenta, ya que, desde el punto de vista del equipo de desarrollo, se tienen fines meramente académicos a la vez que se provee un beneficio a la empresa. No existe la posibilidad, por ejemplo, de introducir más desarrolladores o mejorar el equipamiento utilizado, lo que sí tendría incidencia en los tiempos del proyecto en condiciones habituales.
	\item Incrementar el alcance impacta sobre el tiempo directamente al ser necesaria una mayor cantidad de trabajo para realizar lo especificado.
\end{itemize}
De esto se obtiene que la variable central, al relacionarse con las restantes de manera directa, es el tiempo. Esto se esquematiza en la figura \hyperref[relacion variables del proyecto]{D.\arabic{counter_img_D}}.

\begin{figure}[h!]
	\centering
	{%
		\setlength{\fboxsep}{0pt}%
		\setlength{\fboxrule}{0.5pt}%
		\fbox{\includegraphics[scale = 0.5]{Relaciones de variables del proyecto.png}}%
	}%
	\caption*{Figura D.\arabic{counter_img_D}: Relación entre las variables costo, calidad y alcance con la variable tiempo.}
	\addcontentsline{lof}{figure}{D.\arabic{counter_img_D}. Relación entre las variables costo, calidad y alcance con la variable tiempo.}
	\stepcounter{counter_img_D}
	\label{relacion variables del proyecto}
\end{figure}

De acuerdo a esto podemos simplificar el análisis de impacto ante la ocurrencia de un riesgo teniendo solo en cuenta su incidencia en el tiempo disponible para trabajar sobre el proyecto. De este modo establecemos las categorías \textit{<<Muy bajo>>, <<Bajo>>, <<Moderado>>, <<Alto>>, <<Muy alto>>} en relación a la expectativa de impacto que tiene determinado riesgo en caso de ocurrir. Damos las siguientes valoraciones a las categorías definidas:\\
\textit{Muy bajo: i=0,1; Bajo: i=0,3; Moderado: i=0,5; Alto: i=0,7; Muy Alto: i=0,9.}\\
\indent Del mismo modo definimos la probabilidad de ocurrencia de un riesgo de la siguiente manera:\\
\textit{Muy bajo: p=0,1; Bajo: p=0,3; Moderado: p=0,5; Alto: p=0,7; Muy Alto: p=0,9.}\\
\indent De acuerdo al producto $P \times I$ se clasifican los riesgos de la forma:\\
\indent\textit{$P \times I < 0,15 \rightarrow$ \colorbox{riesgo_bajo}{riesgo bajo}}\\
\indent\textit{$0,15 \leq P \times I<0,25 \rightarrow$ \colorbox{riesgo_medio}{riesgo medio}}\\
\indent\textit{$0,25 \leq P \times I \rightarrow$ \colorbox{riesgo_alto}{riesgo alto}}\\
Las tres categorías se definieron de manera relativa a la distribución de valores que se fueron obteniendo, de modo que no se incurriera en presentar los casos extremos de dar baja importancia a cada riesgo, o, por el contrario, tomar como de gran importancia la mayoría sino es que (posiblemente) todos los riesgos señalados.\\
\indent Con el fin de una identificación rápida de la naturaleza de los riesgos y a partir de los valores de probabilidades e impactos anteriormente definidos se construyó la tabla de probabilidad e impacto que se muestra en la tabla \hyperref[tabla_matriz_prob_impacto]{D.\arabic{counter_tbl_D}} (solo para impactos negativos, es decir, amenazas, teniendo en cuenta que son los que pueden incidir en el proyecto incrementando su tiempo de ejecución).

\edef\tempTblMatProbImp{\the\value{counter_tbl_D}}%vble aux para evitar el cambio de counter_tbl_C en referencia por \stepcounter
\begin{table*}[h!]
	\centering
	\begin{tabular}{
			|p{2.6cm}|p{2cm}|p{2cm}|p{2cm}|p{2cm}|p{2cm}|  }
		
		\hline
		\multicolumn{6}{|c|}{\textbf{Amenazas}} \\
		\hline
		\textbf{Probabilidad}&
		\multicolumn{5}{|c|}{\textbf{Impacto}} \\
		\hline
		\textbf{0,9}& \cellcolor{riesgo_bajo}0,09& \cellcolor{riesgo_alto}0,27& \cellcolor{riesgo_alto}0,45& \cellcolor{riesgo_alto}0,63& \cellcolor{riesgo_alto}0,81 \\
		\hline
		\textbf{0,7}& \cellcolor{riesgo_bajo}0,07& \cellcolor{riesgo_medio}0,21& \cellcolor{riesgo_alto}0,35& \cellcolor{riesgo_alto}0,49& \cellcolor{riesgo_alto}0,63 \\
		\hline
		\textbf{0,5}& \cellcolor{riesgo_bajo}0,05& \cellcolor{riesgo_medio}0,15& \cellcolor{riesgo_alto}0,25& \cellcolor{riesgo_alto}0,35& \cellcolor{riesgo_alto}0,45 \\
		\hline
		\textbf{0,3}& \cellcolor{riesgo_bajo}0,03& \cellcolor{riesgo_bajo}0,09& \cellcolor{riesgo_medio}0,15& \cellcolor{riesgo_medio}0,21& \cellcolor{riesgo_alto}0,27 \\
		\hline
		\textbf{0,1}& \cellcolor{riesgo_bajo}0,01& \cellcolor{riesgo_bajo}0,03& \cellcolor{riesgo_bajo}0,05& \cellcolor{riesgo_bajo}0,07& \cellcolor{riesgo_bajo}0,09 \\
		\hline
		\textbf{}& \textbf{0,1}& \textbf{0,3}& \textbf{0,5}& \textbf{0,7}& \textbf{0,9} \\
		\hline
	\end{tabular}
	\caption*{Tabla D.\arabic{counter_tbl_D}: Matriz de Probabilidad e Impacto.}
	\addcontentsline{lot}{table}{D.\arabic{counter_tbl_D}. Matriz de Probabilidad e Impacto.}\stepcounter{counter_tbl_D}
	\label{tabla_matriz_prob_impacto}
\end{table*}

\indent Se identificaron las amenazas que se muestran en la tabla \hyperref[tabla_riesgos]{D.\arabic{counter_tbl_D}}. Las estimaciones de probabilidades de ocurrencia e impactos ante la ocurrencia se determinaron teniendo en cuenta estimaciones de los miembros del equipo, de acuerdo a las cualidades de los stakeholders del proyecto además de las expectativas de incidencia mayormente en el factor tiempo, como se explicó con anterioridad, y tomando como extremos los valores 0 (sin incidencia, por lo cual no debería tomarse en cuenta) y 1 (el tiempo del proyecto se prolongará indefectiblemente ante la ocurrencia del riesgo).\\

\edef\tempTblRiesgos{\the\value{counter_tbl_D}}%vble aux para evitar el cambio de counter_tbl_C en referencia por \stepcounter
\begin{table*}[h!]
	\centering
	\begin{tabular}{ |p{0.4cm}|p{9cm}|p{2.6cm}|p{1.7cm}|p{1cm}|  }
		\hline
		\verb|#|& \textbf{Riesgo}& \textbf{Probabilidad}& \textbf{Impacto}& \textbf{\textit{$P \times I$}} \\
		\hline
		\textbf{1}& \Copy{riesgo_1}{Malas estimaciones de tiempos por falta de experiencia} (probable subestimación).& 0,7& 0,7& \cellcolor{riesgo_alto} 0,49 \\
		\hline
		\textbf{2}& \Copy{riesgo_2}{Tareas no lo suficientemente simples para ser estimables en cuanto al tiempo}.& 0,3& 0,7& \cellcolor{riesgo_medio} 0,21 \\
		\hline
		\textbf{3}& \Copy{riesgo_3}{Falta de habilidad en el uso de las herramientas de desarrollo}.& 0,9& 0,1& \cellcolor{riesgo_bajo} 0,09 \\
		\hline
		\textbf{4}& \Copy{riesgo_4}{Imposibilidad de trabajar la cantidad de horas pactadas}.& 0,3& 0,7& \cellcolor{riesgo_medio} 0,21 \\
		\hline
		\textbf{5}& \Copy{riesgo_5}{Disminución en el número del equipo de desarrollo}.& 0,1& 0,9& \cellcolor{riesgo_bajo} 0,09 \\
		\hline
		\textbf{6}& \Copy{riesgo_6}{Desvinculación del proyecto de un miembro de la empresa}.& 0,3& 0,5& \cellcolor{riesgo_medio} 0,15 \\
		\hline
		\textbf{7}& \Copy{riesgo_7}{Falta de disponibilidad por parte del cliente}.& 0,3& 0,9& \cellcolor{riesgo_alto} 0,27 \\
		\hline
		\textbf{8}& \Copy{riesgo_8}{Requerimientos cambiantes por parte del cliente}.& 0,7& 0,3& \cellcolor{riesgo_medio} 0,21 \\
		\hline
		\textbf{9}& \Copy{riesgo_9}{Trabajo estipulado inconcluso en una iteración}.& 0,5& 0,5& \cellcolor{riesgo_alto} 0,25 \\
		\hline
	\end{tabular}
	\caption*{Tabla D.\arabic{counter_tbl_D}: Estimaciones de probabilidades e impactos de riesgos.}
	\addcontentsline{lot}{table}{D.\arabic{counter_tbl_D}. Estimaciones de probabilidades e impactos de riesgos.}
	\stepcounter{counter_tbl_D}
	\label{tabla_riesgos}
\end{table*}

\textbf{Explicación detallada de los riesgos (tabla \hyperref[tabla_riesgos]{D.\tempTblRiesgos}):}
\begin{enumerate}
	\item \underline{\smash{\Paste{riesgo_1}}:}\\
	Debido a la poca experiencia de los miembros del equipo de trabajo en la estimación de tiempos, es posible que se incurra en el error de establecer de manera inconsciente el escenario del mejor caso para la estimación de las tareas en que se dividen las iteraciones, dando por resultado una subestimación del tiempo requerido para la realización de las actividades de que estas se componen.
	\item \underline{\smash{\Paste{riesgo_2}}:}\\
	La división de requerimientos en tareas no lo suficientemente simples para simplificar el proceso de estimación del tiempo necesario para las iteraciones puede, a su vez, ocasionar errores de estimación en los tiempos del proyecto.
	\item \underline{\smash{\Paste{riesgo_3}}:}\\
	La falta de habilidad en el uso de las herramientas de desarrollo de software particulares del proyecto también pueden motivar retrasos, pudiendo ser más evidentes estos mayormente en las primeras etapas del proyecto.
	\item \underline{\smash{\Paste{riesgo_4}}:}\\
	Imposibilidad de algún miembro del equipo de desarrollo de cumplir la cantidad de horas de trabajo pactadas por jornada hábil.
	\item \underline{\smash{\Paste{riesgo_5}}:}\\
	Alguno de los miembros del equipo de desarrollo se desliga de manera definitiva el proyecto.
	\item \underline{\smash{\Paste{riesgo_6}}:}\\
	Desvinculación del proyecto de un miembro de la empresa (cliente), por lo que se espera tener que reemplazarlo con el tiempo y trabajo de transmisión de conocimientos necesarios que eso implica para poder vincular a alguien más en su lugar, o simplemente perdiendo la posibilidad de contar con su ayuda o la de alguien en su lugar ante dudas respecto a los requerimientos del sistema.
	\item \underline{\smash{\Paste{riesgo_7}}:}\\
	La disponibilidad del cliente puede no ser la adecuada para lograr una buena comprensión de los requerimientos del producto, pudiendo ocasionar demoras directas (procesos bloqueados por no obtener respuestas) o indirectas (al ocasionar mayor cantidad de modificaciones en estadios más avanzados del proyecto).
	\item \underline{\smash{\Paste{riesgo_8}}:}\\
	Los requerimientos del cliente se modifican en el transcurso del proyecto.
	\item \underline{\smash{\Paste{riesgo_9}}:}\\
	Al concluir con el tiempo previsto para una iteración, no se ha alcanzado a concluir con el desarrollo correspondiente a las historias previstas para tal caso.
\end{enumerate}

\textbf{Explicación de las estimaciones realizadas para los riesgos señalados (tabla \hyperref[tabla_riesgos]{D.\tempTblRiesgos}):}
\begin{enumerate}
	\item \underline{\smash{\Paste{riesgo_1}}\hypertarget{explicacion_estimacion_riesgo_uno}{}:}\\
	- La probabilidad de hacer malas estimaciones por parte del equipo es alta ya que la experiencia es escasa. Aún así no es inexistente y en el caso previo se tuvieron resultados aceptables.\\
	- El impacto es alto debido a la incidencia directa en los tiempos del proyecto. Los tiempos discordantes respecto a los estimados para las distintas \hyperlink{US}{US} pueden incluso acumularse dentro tanto como entre distintas iteraciones de no actuarse de manera adecuada.
	\item \underline{\smash{\Paste{riesgo_2}}:}\\
	- De seguirse con el criterio \hyperlink{INVEST}{INVEST}\setcounter{footnote}{0}\footnote{Bajo este criterio, buenas \hyperlink{US}{US} tienen las características: Independiente, Negociable, Valuable, Estimable, Pequeña. El concepto fue tomado de: Pokharel, Prabhat - Vaidya, Pramesh. \textit{<<A Study of User Story in Practice>>}, Octubre 2020. Págs. 1-2.} este riesgo debería tener una probabilidad baja de ocurrencia.\\
	- El argumento en el caso del impacto es similar que en el riesgo 1 (véase \hyperlink{explicacion_estimacion_riesgo_uno}{explicación de estimaciones de riesgo 1}), ya que el problema se traduce en una mala estimación de los tiempos de las iteraciones.
	\item \underline{\smash{\Paste{riesgo_3}}:}\\
	- Es muy probable que se tenga desconocimiento por parte de los miembros del equipo en el manejo de las tecnologías de desarrollo debido a que no son las que habitualmente se utilizan por parte de estos.\\
	- A pesar de lo mencionado, el desconocimiento no es absoluto, sino que se tiene una falta de costumbre que, se prevé, podrá afectar mayormente en las primeras etapas del desarrollo, por lo que en tales instancias será de mayor importancia atender a este aspecto. En la medida que esto pueda manejarse de manera correcta (como en general suele suceder de acuerdo con nuestra experiencia), el impacto debería ser muy bajo.
	\item \underline{\smash{\Paste{riesgo_4}}:}\\
	- La probabilidad de que al menos uno de los miembros del equipo de trabajo no pueda dedicar el tiempo pactado a sus labores es baja debido a que actualmente se cuenta con tiempo escasamente comprometido a otras actividades. Esto debería mantenerse a lo largo del tiempo de duración del proyecto a menos que se suscite una condición diferente, lo cual se estima muy poco probable.\\
	- El impacto de ocurrencia sería alto dado que parte del trabajo del miembro del equipo cuya cantidad de horas de trabajo se vería afectado debería suplirse con el trabajo del otro miembro. Se espera que, de ocurrir esto, el miembro sobrecargado no tenga mayores inconvenientes relativos a la adquisición de conocimientos (de ser esta necesaria) debido a la metodología de trabajo utilizada que incluye una continua comunicación entre los miembros del equipo, así es que todo el impacto esperado se debería encontrar en la adición de horas de trabajo para el miembro no afectado (al menos directamente) por la condición.
	\item \underline{\smash{\Paste{riesgo_5}}:}\\
	- La probabilidad de salida de un miembro del equipo es incluso inferior a la probabilidad de imposibilidad de poder cumplir con la cantidad de horas de trabajo pactadas, y ya que sería un caso extremo, se lo considera como algo bastante improbable.\\
	- De ocurrir el impacto sería muy grande debido a que todo el peso del proyecto recaería sobre el miembro restante del equipo, que no solamente debería redoblar esfuerzos en cuanto a cantidad de horas de trabajo, sino que también perdería los beneficios conseguidos mediante la sinergia del equipo.
	\item \underline{\smash{\Paste{riesgo_6}}:}\\
	- La probabilidad de desvinculación de un miembro de la empresa es baja ya que se sabe que en general el equipo de trabajo no es renovado de manera demasiado frecuente.\\
	- Teniendo en cuenta que el impacto de ocurrencia de desvinculación para los sectores de jerarquía superior sería muy alto, así como muy bajo para los miembros de los puestos más bajos en jerarquía, pero a su vez la probabilidad de cambios en los sectores aumenta de manera inversa con la importancia jerárquica de estos, podemos hacer un promedio de las situaciones, lo que ubicaría al riesgo en un impacto moderado.
	\item \underline{\smash{\Paste{riesgo_7}}:}\\
	- El cliente se ha comprometido a responder a las dudas del equipo y mantener una comunicación constante de ser necesario para el cumplimiento de los objetivos del proyecto. De cualquier modo, y aunque en general la empresa no presenta aumentos estacionales de trabajo, sus empleados podrían encontrarse más atareados en ocasiones.\\
	- El valor estimado proviene del supuesto de que el contacto continuo con el cliente es de alta importancia, de producirse falencias de este tipo se afectaría, como se ha mencionado, de manera directa (bloqueos en tareas) así como indirecta (ocasionando mayor cantidad de modificaciones en etapas más avanzados del proyecto) a los tiempos del proyecto.
	\item \underline{\smash{\Paste{riesgo_8}}:}\\
	- Es altamente esperable que el cliente tenga requerimientos razonablemente cambiantes, y fue esa una de las razones (posiblemente la más importante) por la cual se ha decidido emplear una metodología con características ágiles para desarrollar la solución planteada.\\
	- Debido a la continua comunicación que se espera tener con el cliente y a la característica iterativa utilizada en la implementación de las etapas del proyecto, se espera que la incidencia en el tiempo del proyecto sea baja al trabajar naturalmente previendo posibles cambios en las definiciones.
	\item \underline{\smash{\Paste{riesgo_9}}:}\\
	- La probabilidad de ocurrencia en las primeras iteraciones y sobre todo en la primera (cuando la conclusión del trabajo depende en mayor medida de un buen ajuste entre lo realizado y las estimaciones basadas en una baja experiencia del equipo) debería ser mayor, siendo mínima en las últimas al ajustarse las velocidades de trabajo de acuerdo con las precedentes. En promedio debería ser moderada.\\
	- El impacto debería ser de bajo a moderado para cada iteración particular, pudiendo llegar a ser alto de no tomarse las medidas necesarias para corregir las estimaciones realizadas.
\end{enumerate}

\textbf{Planificación de respuesta a los riesgos (tabla \hyperref[tabla_riesgos]{D.\tempTblRiesgos}):}
\begin{enumerate}
	\item \underline{\smash{\Paste{riesgo_1}}\hypertarget{planificacion_respuesta_riesgo_uno}{}:}\\
	Se utilizará una redefinición de la velocidad de desarrollo (cantidad de horas de trabajo empleado por \hyperlink{SP}{SP}) en la iteración en función de la anterior:\\
	La iteración termina aún si no ha sido finalizada la totalidad de las \hyperlink{US}{US} previstas para esta. Se calcula la velocidad real de la iteración (cantidad de horas de trabajo empleado por \hyperlink{SP}{SP}, como resultado de la cantidad de horas trabajadas en la iteración sobre la cantidad de \hyperlink{SP}{SP} concluidas de manera efectiva), la cual será utilizada para planificar la siguiente iteración. Luego la cantidad de \hyperlink{SP}{SP} a realizar en la iteración siguiente no puede exceder a la cantidad de \hyperlink{SP}{SP} realizados en la iteración precedente.\\
	\indent Por otro lado, dado que la estrategia solo es aplicable en iteraciones distintas de la primera, en esta se deberá intentar subestimar la velocidad de modo de contemplar el tiempo de investigación, adquisición de conocimientos fundamentales así como periodo inicial de acostumbramiento a las tecnologías utilizadas.
	\item \underline{\smash{\Paste{riesgo_2}}:}\\
	Como se ha mencionado, mediante el empleo del criterio \hyperlink{INVEST}{INVEST} este riesgo debería tener una probabilidad baja de ocurrencia. De cualquier modo, de detectarse alguna \hyperlink{US}{US} que no cumpla de manera correcta con alguno de las características perseguidas por el criterio, la misma puede ser redefinida (en general mediante la división de la misma, de acuerdo a la tendencia propia de las características perseguidas de independencia I y pequeñez S) así como redefinidas deben ser sus características como por ejemplo la cantidad de \hyperlink{SP}{SP} necesarios para su conclusión o la redistribución de trabajo en el equipo para su realización.
	\item \underline{\smash{\Paste{riesgo_3}}:}\\
	La planificación de respuesta a esta amenaza es tenida en cuenta en la planificación de respuesta a la amenaza 1, ya que es evidente que la falta de habilidad en el uso de las herramientas se dará principalmente en la primera o primeras iteraciones, disminuyendo a medida que el proyecto progrese. Se propone intentar establecer un valor menor al inicialmente propuesto en la estimación de la velocidad de trabajo del equipo (intentar utilizar una subestimación de la misma).
	\item \underline{\smash{\Paste{riesgo_4}}:}\\
	Ante la imposibilidad por parte de alguno de los miembros del equipo de desarrollo de trabajar la cantidad de horas pactadas, se deberán redistribuir las responsabilidades en las iteraciones de manera ponderada de acuerdo a la cantidad de horas en que cada uno puede trabajar. En el caso en que el restante miembro del equipo no pueda suplir al otro, deberá asumirse una velocidad inferior en las iteraciones, lo que implicará un retraso inevitable en los tiempos generales del proyecto (esto bajo la suposición de la no existencia de una subestimación en el valor de la velocidad de trabajo, en cuyo caso se podría, dependiendo del caso, incluso cumplir con los tiempos planeados, pero no es un supuesto que suela ser cierto).
	\item \underline{\smash{\Paste{riesgo_5}}:}\\
	La única manera de afrontar esta situación es asumiendo el aumento de trabajo en el miembro restante del equipo, lo cual a su vez posiblemente demandaría una redefinición de las métricas utilizadas en las estimaciones de tiempos realizadas. Como se dijo, el riesgo de que esto ocurra es bajo y la manera de prevenir saltos importantes de productividad ante la ocurrencia es la comunicación continua entre los miembros del equipo. Esto último evitaría al restante miembro un gasto extra de tiempo para la adquisición de conocimientos relacionados a tareas que inicialmente le eran ajenas, al menos desde la perspectiva del trabajo a desarrollar.
	\item \underline{\smash{\Paste{riesgo_6}}:}\\
	Ante el reemplazo o desvinculación permanente de un miembro de la empresa (sin alguien nuevo asignado en su posición así como para el proyecto) de jerarquía baja, se espera asimismo que el impacto sobre los requerimientos sea bajo, ya que su baja jerarquía implica a su vez baja influencia en las decisiones empresariales.
	En el caso en que se reemplace a alguien de una jerarquía alta (ya no hablamos de falta de reemplazo ya que esto es impensado si se desea no solo el correcto funcionamiento del proyecto, sino también de la empresa) se deberá asumir el tiempo extra invertido en la puesta en contexto del nuevo empleado, lo que podría ocasionar una rediagramación del trabajo. Se propone en ese caso un aumento en la comunicación durante la incorporación del nuevo empleado al proyecto. Sin embargo, como ya se dijo, esto es altamente improbable para los puestos pertenecientes a las jerarquías superiores (esto teniendo en cuenta la naturaleza de la empresa, ya que puede ser más frecuente en otro tipo de empresas).
	\item \underline{\smash{\Paste{riesgo_7}}:}\\
	Debido a que este es uno de los supuestos más fuertes así como uno de los principales por los cuales se optó por emplear un modelo de desarrollo basado en los principios ágiles, de ocurrir, la falta de comunicación podría ser muy perjudicial en cuanto a la calidad del producto terminado. Al depender esto del compromiso que tiene la empresa con el proyecto, solo se puede poner énfasis en su importancia e intentar establecer varios medios de comunicación (los más cómodos en cada caso) con los miembros de la empresa que formen parte del proyecto de manera directa.
	\item \underline{\smash{\Paste{riesgo_8}}:}\\
	El enfoque ágil que se le da al desarrollo del proyecto debería, de ser llevado de manera correcta, bastar para bajar considerablemente el impacto de los requerimientos cambiantes. La comunicación constante con el cliente da por resultado una mejor comprensión de sus necesidades así como de los requisitos que este adiciona al sistema. La naturaleza iterativa del modelo de desarrollo del software adoptado posibilita la validación de requerimientos implementados sobre la solución planteada antes de tener un impacto mayor en etapas tardías (y relativamente distantes) del proyecto.
	\item \underline{\smash{\Paste{riesgo_9}}:}\\
	Se utilizará una redefinición de la velocidad de trabajo en la iteración siguiente en función de la anterior (véase explicación en la \hyperlink{planificacion_respuesta_riesgo_uno}{planificación de respuesta a riesgo 1}). Ya que no se concluye con la totalidad del trabajo de la iteración, se debe estimar el tiempo necesario para finalizarlo en la iteración siguiente, y en base a esto y a la redefinición de la velocidad, reorganizarla convenientemente. Esto último implica a su vez una posible reestructuración de la iteración (es posible eliminar o reemplazar \hyperlink{US}{US} inicialmente planificadas por otras en concordancia con lo estimado).
	En adición, a modo de prevención se utilizará el método de \textit{<<halfway point>>}. A mitad de la iteración el equipo se reúne para verificar que la mitad de las \hyperlink{US}{US} planificadas para la iteración se hayan completado. En el caso de que esto no ocurra, el equipo deberá redistribuir las tasks y responsabilidades para asegurar que todas las \hyperlink{US}{US} puedan ser concluidas antes de la finalización de la iteración.
\end{enumerate}




\chapter*{\hypertarget{apendice_e}{}Apéndice E\\Requerimientos definidos inicialmente}
\markboth{Apéndice E - Requerimientos definidos inicialmente}{Apéndice E - Requerimientos definidos inicialmente}
\sectionmark{Apéndice E - Requerimientos definidos inicialmente}
\addcontentsline{toc}{chapter}{E. Requerimientos definidos inicialmente}


\label{requerimientos_definidos_inicialmente}
A continuación se listan las \hyperlink{US}{US} definidas en la planificación inicial del proyecto. Estas fueron modificándose en el transcurso del mismo, de acuerdo a la planificación de las distintas iteraciones y en función de los requerimientos (cambiantes) del cliente. Se indica la cantidad estimada de \hyperlink{SP}{SP} para su realización en cada caso:
\begin{itemize}
	\item \underline{Inicio de sesión:} los usuarios deben identificarse para acceder al sistema y poder realizar cualquier acción sobre el mismo.\\
	\textit{Story Points estimados:} \textbf{3}
	\item \underline{Administrar usuarios:} solo el gerente general debe poder agregar nuevos usuarios al sistema. Los usuarios del sistema serán únicamente los empleados de la empresa, por lo que no se espera que estos varíen de manera frecuente. Además, el gerente debe poder editar la información de los usuarios o eliminarlos  si así lo desea.\\
	\textit{Story Points estimados:} \textbf{2}
	\item \underline{Administrar insumo:} el encargado de mantenimiento debe poder registrar, eliminar o modificar un insumo de características particulares. En caso de alcanzarse el valor de punto de pedido del insumo se debe notificar al encargado de mantenimiento.\\
	\textit{Story Points estimados:} \textbf{4}
	\item \underline{Administrar ti\smash{p}o de insumo:} el encargado de mantenimiento debe poder buscar insumos que sean equivalentes o alternativos, en caso de faltante imprevisto de un insumo particular (ya sea por demoras en la reposición o la imposibilidad de conseguir stock en el mercado). En caso de alcanzarse el valor de punto de pedido del tipo de insumo se debe notificar al encargado de mantenimiento.\\
	\textit{Story Points estimados:} \textbf{4}
	\item \underline{Modificar cantidad de insumo:} se deberá poder llevar registro de los insumos disponibles en cada depósito. El encargado de reposición deberá registrar el ingreso y egreso de insumos en los distintos depósitos de la empresa de manera periódica a medida que estos sean solicitados para cumplir con los distintos trabajos. Por su parte, el encargado de mantenimiento debe poder ajustar los valores reales de los distintos insumos en caso de notar discrepancias durante algún control de inventario, esto incluye la posibilidad de adición de un nuevo insumo al depósito.\\
	\textit{Story Points estimados:} \textbf{1}
	\item \underline{Administrar de\smash{p}ósitos:} la empresa no está limitada a posibles futuras ampliaciones físicas o nuevas disposiciones de gestión, por lo que el encargado de mantenimiento debe poder agregar nuevos depósitos e incluso quitar los existentes.\\
	Se debe tener en consideración que la empresa cuenta con varios depósitos, los cuales pueden incluso estar ubicados en una misma sucursal. Por ello es que el encargado de mantenimiento también debe poder designar los distintos depósitos existentes para poder diferenciarlos.\\
	\textit{Story Points estimados:} \textbf{3}
	\item \underline{Administrar sucursales:} el gerente general debe poder añadir y eliminar nuevos establecimientos o sucursales de la empresa, ya que es posible que en el futuro la disposición geográfica de la empresa cambie.\\
	\textit{Story Points estimados:} \textbf{3}
	\item \underline{\smash{Administrar proyecto}:} el jefe de producción debe poder registrar nuevos proyectos, así como también editar o eliminar proyectos previamente registrados. Estos deben incluir datos administrativos básicos como el nombre o designación del proyecto, el cliente y la fecha límite de entrega del producto.\\
	\textit{Story Points estimados:} \textbf{3}
	\item \underline{\smash{Agregar insumo a proyecto}:} con fines de planeación y gestión de proyectos, el gerente general deberá poder solicitar la reserva de determinada cantidad de un material/insumo para un proyecto determinado. Por su parte, el encargado de mantenimiento deberá poder seleccionar los depósitos de los cuales reservará cada insumo para hacer frente a la solicitud, pudiendo incluso obtener la cantidad requerida de un insumo de diferentes depósitos.\\
	\textit{Story Points estimados:} \textbf{5}
	\item \underline{\smash{Administrar tipos de proyectos}:} el encargado de producción debe poder agregar nuevos tipos de proyecto e indicar una secuencia de tareas distinta de la secuencia de los tipos de proyectos genéricos (almanaque, libro, revista o volante/afiche).\\
	\textit{Story Points estimados:} \textbf{5}
	\item \underline{Editar tarea:} el jefe de producción debe poder editar las características de las tareas del proceso de producción, o sea los trabajos que los empleados de producción deberán realizar para cumplir con los pedidos de los clientes. Las tareas pueden ser de preprensa, impresión, corte, plegado, encuadernado, prensado o embalado. Además se debe poder dar seguimiento al avance de cada proyecto, de manera de saber qué tareas se realizaron y cuáles están pendientes para finalizarlo.\\
	\textit{Story Points estimados:} \textbf{2}
	\item \underline{\smash{Asignar tarea a empleado}:} el jefe de producción debe poder designar a los empleados que se encargarán de realizar las distintas tareas del proceso de producción.\\
	\textit{Story Points estimados:} \textbf{1}
	\item \underline{\smash{Agregar comentario a tarea}:} los empleados de producción deben poder dejar un comentario en las tareas a las que hayan sido asignados. Estos comentarios sirven para dar mayores detalles acerca de la tarea en cuestión además del simple estado de la misma.\\
	\textit{Story Points estimados:} \textbf{2}
	\item \underline{Modificar estado de tarea:} el jefe de producción y los empleados de producción deben poder modificar el estado de una tarea perteneciente a un proyecto del cual forman parte. Al finalizar una tarea se debe habilitar el inicio de aquellas tareas que dependen de su finalización para comenzar.\\
	\textit{Story Points estimados:} \textbf{2}
	\item \underline{\smash{Finalizar proyecto}:} el jefe de producción debe poder marcar un proyecto como finalizado e indicar la cantidad sobrante de cada insumo reservado en un inicio. Luego, el encargado de mantenimiento debe poder elegir a qué depósito destinar dichos insumos. Además, se deberá registrar datos estadísticos del proyecto tales como  el tiempo real empleado para su realización, así como también el presupuesto estimado y la cantidad real de insumos utilizados.\\
	\textit{Story Points estimados:} \textbf{4}
	\item \underline{\smash{Buscar proyecto}:} el encargado de producción debe poder buscar los proyectos tanto por título, estado o cliente del proyecto.\\
	\textit{Story Points estimados:} \textbf{2}
	\item \underline{\smash{Visualizar tareas asignadas}:} los empleados de producción deben poder visualizar la lista de tareas que se le han asignado y que aún no se han realizado. Además, deben poder visualizar un mayor detalle de una tarea específica y del proyecto del cual forma parte.\\
	\textit{Story Points estimados:} \textbf{3}
	\item \underline{Administrar clientes:} el gerente general debe poder llevar registro de los clientes de la empresa y sus datos de contacto.\\
	\textit{Story Points estimados:} \textbf{2}
\end{itemize}




\chapter*{\hypertarget{apendice_f}{}Apéndice F\\Ambiente de desarrollo de software, tecnologías y plataformas}
\markboth{Apéndice F - Ambiente de desarrollo de software, tecnologías y plataformas}{Apéndice F - Ambiente de desarrollo de software, tecnologías y plataformas}
\sectionmark{Apéndice F - Ambiente de desarrollo de software, tecnologías y plataformas}
\addcontentsline{toc}{chapter}{F. Ambiente de desarrollo de software, tecnologías y plataformas}

\indent Las tecnologías y plataformas empleadas en el proyecto fueron seleccionadas teniendo en cuenta que el sistema debe utilizarse tanto en dispositivos de escritorio como móviles. El sistema es compatible con los sistemas operativos Windows 7 (y superiores) para la versión de escritorio y Android 7 (y superiores) para la versión móvil. Esta elección se realizó teniendo en cuenta que todas las PCs actualmente disponibles en la empresa cuentan con el sistema operativo Windows 7/10. Por su parte, se optó por desarrollar la versión móvil para el sistema Android ya que todos los empleados de la empresa cuentan con dispositivos móviles con dicho sistema operativo. En particular, se consideró que el sistema debe ser compatible con la versión 7 de Android y posteriores teniendo en cuenta que en la actualidad alrededor del 97\% de los dispositivos Android activos en el mundo tienen instalada la versión 7 de Android o posteriores\setcounter{footnote}{0}\footnote{Esto fue obtenido de:\\
	<<\textit{Android API Levels}>> de API Levels, \url{https://apilevels.com/}. Consultado el 03/05/2024;\\
	en base a: <<\textit{Mobile} \& \textit{Tablet Android Version Market Share Worldwide}>> de Statcounter Global Stats,\\ \url{https://gs.statcounter.com/android-version-market-share/mobile-tablet/worldwide}}.\\
\indent El siguiente factor que se ha tenido en cuenta fue la experiencia inicial de los miembros del equipo de desarrollo con las distintas tecnologías, de manera de reducir lo máximo posible el tiempo de aprendizaje de las tecnologías a utilizar. Además, se consideró de importancia la disponibilidad de documentación en línea y la probabilidad de que las tecnologías no pierdan soporte oficial en un futuro cercano, de manera de evitar, en la medida de lo posible, que el sistema quede obsoleto en un periodo relativamente corto por problemas de soporte/compatibilidad.\\
\indent Bajo estas premisas, se ha decidido desarrollar la versión móvil en el entorno de desarrollo Android Studio\footnote{Web oficial de Android Studio, \url{https://developer.android.com/studio}. Consultado el 03/05/2024.}, mientras que la versión de escritorio fue desarrollada en el entorno de desarrollo NetBeans\footnote{Web oficial de Apache NetBeans, \url{https://netbeans.apache.org/}. Consultado el 03/05/2024.}. Para ambos casos se utilizó el lenguaje de programación Java\footnote{Web oficial de Oracle Java, \url{https://www.oracle.com/es/java/}. Consultado el 03/05/2024.}, con la salvedad de que para el sistema de escritorio se empleó el framework JavaFX\footnote{Web oficial de JavaFX, \url{https://openjfx.io/}. Consultado el 03/05/2024.}. De esta manera ambos sistemas utilizan la misma lógica de backend y una estructura de frontend distinta.\\
\indent Para la persistencia de datos se decidió utilizar el sistema de gestión de bases de datos relacional PostgreSQL\footnote{Web oficial de PostgreSQL, \url{https://www.postgresql.org/}. Consultado el 03/05/2024.}.\\
\indent Por último, se utilizó GitHub\footnote{Web oficial de GitHub, \url{https://github.com/}. Consultado el 03/05/2024.} como sistema de gestión de versiones.\\
Las versiones utilizadas para cada caso son las mencionadas a continuación:
\begin{itemize}
	\item Android Studio: versión 2023.1.1 Patch 2 (Hedgehog).
	\item Apache NetBeans: versión 22.
	\item Java (JDK): versión 22.0.1.
	\item JavaFX: versión 22.0.1.
	\item PostgreSQL: 13.21.
\end{itemize}




\clearpage
\phantomsection
{
	\centering
	\vspace*{\fill}
	\Huge \hypertarget{apendice_g}{}\textbf{Apéndice G}\\[1em]
	\Huge \textbf{Diagramas de clases y de E-R}\\[1em]
	\vspace*{\fill}
	\par
}
\addcontentsline{toc}{chapter}{G. Diagramas de clases y de E-R}
\markboth{Apéndice G - Diagramas de clases y de E-R}{Apéndice G - Diagramas de clases y de E-R}

\sectionmark{Apéndice G - Diagramas de clases y de E-R}


%%%%%%%%%%%%%%%%%%%%%%%%%%%%%%%%%%%%%%%%%
\appendix
\renewcommand{\thefigure}{G.\arabic{figure}}
\setcounter{figure}{0}
%%%%%%%%%%%%%%%%%%%%%%%%%%%%%%%%%%%%%%%%%

%\begin{landscape}
%	\begin{figure}[H]
%		\centering
%		\includegraphics[width=\textheight, angle=90, scale = %0.47]{Diagrama de Clases.png}
%		\caption{Diagrama de clases del sistema.}
%		\label{diagrama_de_clases}
%	\end{figure}
%\end{landscape}

\begin{figure}[H]
	\centering
	\includegraphics[scale=0.17]{Diagrama de Clases.png}
	\caption{Diagrama de clases del sistema.}
	\label{diagrama_de_clases}
\end{figure}


A continuación se da una descripción breve para cada clase:\\\\
\underline{\smash{Usuario}:}\\
Representa un usuario del sistema (empleado de la empresa).\\
\\
\underline{\smash{TipoUsuario}:}\\
Se usa para definir las capacidades que tiene un usuario en el sistema. Asimismo, indica el rol del usuario en la empresa.\\
\\
\underline{\smash{Cliente}:}\\
Representa un cliente de la institución.\\
\\
\underline{\smash{Sucursal}:}\\
Representa un edificio físico bajo control de la empresa donde se almacenan insumos.\\
\\
\underline{\smash{Deposito}:}\\
Representa un espacio de almacenamiento dentro de una sucursal específica.\\
\\
\underline{\smash{TipoInsumo}:}\\
Agrupación de distintos productos bajo un criterio común y general.\\
\\
\underline{\smash{Insumo}:}\\
Representa una distribución particular de un producto, pudiendo incluir su marca, forma de distribución o cantidad, entre otras.\\
\\
\underline{\smash{Provision}:}\\
Representa la existencia de un insumo específico almacenado en un depósito dado.\\
\\
\underline{\smash{Reserva}:}\\
Indica la cantidad de un insumo, proveniente de un depósito particular y destinado a la realización de un proyecto.\\
\\
\underline{\smash{Novedad}:}\\
Mensaje del sistema que sirve para mantener al tanto a un usuario de sucesos que pueden resultar de interés de acuerdo a sus funciones en la empresa.\\
\\
\underline{\smash{TipoNovedad}:}\\
Se usa para indicar la razón que originó a cada novedad particular.\\
\\
\underline{\smash{TipoTarea}:}\\
Sirve para definir de manera abstracta las características generales de un conjunto de tareas agrupándolas según el tipo de trabajo que se debe llevar a cabo para realizarlas.\\
\\
\underline{\smash{TipoTareaPropiedad}:}\\
Propiedad de un tipo de tarea.\\
\\
\underline{\smash{TipoProyecto}:}\\
Sirve para definir de manera abstracta las características generales de un conjunto de proyectos agrupándolos en función del tipo de producto final a elaborar.\\
\\
\underline{\smash{Proyecto}:}\\
Representa el pedido de un cliente para la consecución de un producto final particular.\\
\\
\underline{\smash{EstadoProyecto}:}\\
Se usa para indicar el estado actual de un proyecto.\\
\\
\underline{\smash{Tarea}:}\\
Representa una tarea particular que se debe realizar en un proyecto.\\
\\
\underline{\smash{EstadoTarea}:}\\
Se usa para indicar el estado actual de una tarea.\\
\\
\underline{\smash{TareaPropiedad}:}\\
Indica una propiedad y su valor específico para una tarea.\\
\\
\underline{\smash{ProRelacionTareas}:}\\
Indica el orden de precedencia de las tareas en un proyecto.\\
\\
\underline{\smash{ComentarioTarea}:}\\
Son los comentarios hechos por los empleados acerca de las distintas tareas que se les han asignado. Sirven para dar cuenta de algún suceso o expresar consideraciones que crean relevantes relacionadas con una tarea dada.

\begin{landscape}
	\begin{figure}[H]
		\centering
		\includegraphics[width=\textheight, angle=90]{Diagrama E-R.png}
		\caption{Diagrama de entidad-relación del sistema.}
		\label{diagrama_ER}
	\end{figure}
\end{landscape}

De la estructura dada por el DER se obtuvieron las tablas listadas a continuación:
\\\\
\noindent\textbf{usuario}(\underline{nombre}, contrasenia, tipo, activo, per\_nombre, per\_apellido)
\\\\
\textbf{cliente}(\underline{id}, nombre, telefono, direccion, email)
\\\\
\textbf{sucursal}(\underline{id}, nombre, direccion)
\\\\
\textbf{deposito}(\underline{id}, \underline{\dashuline{id\_sucursal}}, nombre, descripcion)
\\\\
\textbf{tipo\-insumo}(\underline{id}, nombre, unidad, punto\_pedido)
\\\\
\textbf{insumo}(\underline{id}, \underline{\dashuline{id\_tipo\_insumo}}, nombre, descripcion, medida, punto\_pedido)
\\\\
\textbf{provision}(\underline{\dashuline{id\_sucursal}}, \underline{\dashuline{id\_deposito}}, \underline{\dashuline{id\_tipo\_insumo}}, \underline{\dashuline{id\_insumo}}, cantidad, \underline{\dashuline{id\_cliente}})
\\\\
\textbf{reserva}(\underline{\dashuline{id\_proyecto}}, \underline{\dashuline{id\_sucursal}}, \underline{\dashuline{id\_deposito}}, \underline{\dashuline{id\_tipo\_insumo}}, \underline{\dashuline{id\_insumo}}, cantidad, \underline{\dashuline{id\_cliente}})
\\\\
\textbf{novedad}(\underline{id}, \dashuline{nombre\_usuario}, tipo, contenido, json\_entidades, fecha\_creacion, fecha\_vista)
\\\\
\textbf{tipo\_tarea}(\underline{id}, nombre)
\\\\
\textbf{tipo\_tarea\_propiedad}(\underline{\dashuline{id\_tipo\_tarea}}, \underline{id}, nombre)
\\\\
\textbf{tipo\_tarea\_propiedad\_opcion}(\underline{\dashuline{id\_tipo\_tarea}}, \underline{\dashuline{id\_tipo\_tarea\_propiedad}}, \underline{id}, opcion)
\\\\
\textbf{tipo\_proyecto}(\underline{id}, nombre)
\\\\
\textbf{tipo\_tarea\_en\_tipo\_proyecto}(\underline{\dashuline{id\_tipo\_proyecto}}, \underline{id}, \dashuline{id\_tipo\_tarea}, nombre)
\\\\
\textbf{tippro\_relacion\_tipos\_tareas}(\underline{\dashuline{id\_tipo\_proyecto}}, \underline{\dashuline{id\_padre}}, \underline{\dashuline{id\_hija}})
\\\\
\textbf{proyecto}(\underline{id}, titulo, \dashuline{id\_tipo\_proyecto}, \dashuline{id\_cliente}, fecha\_inicio, fecha\_fin, fecha\_entrega, estado, nombre\_tipo\_proyecto)
\\\\
\textbf{tarea}(\underline{id}, \underline{\dashuline{id\_proyecto}}, \dashuline{id\_tipo\_tarea}, estado, \dashuline{usuario\_asignado}, nombre\_tipo\_tarea,\\ nombre\_tipo\_tarea\_en\_tipo\_proyecto)
\\\\
\textbf{propiedad\_de\_tarea}(\underline{id}, \underline{\dashuline{id\_tarea}}, \underline{\dashuline{id\_proyecto}}, \dashuline{id\_tipo\_tarea}, \dashuline{id\_tipo\_tarea\_propiedad}, nombre, valor)
\\\\
\textbf{pro\_relacion\_tareas}(\underline{\dashuline{id\_proyecto}}, \underline{\dashuline{id\_padre}}, \underline{\dashuline{id\_hija}})
\\\\
\textbf{comentario\_tarea}(\underline{id}, \underline{\dashuline{id\_tarea}}, \underline{\dashuline{id\_proyecto}}, contenido, \dashuline{usuario}, fecha\_creacion)
\pagebreak

A continuación se da una descripción breve para cada tabla:\\\\
\underline{\smash{usuario}:}\\
Representa un usuario del sistema (empleado de la empresa).\\
\\
\underline{\smash{cliente}:}\\
Representa un cliente de la institución.\\
\\
\underline{\smash{sucursal}:}\\
Representa un edificio físico bajo control de la empresa donde se almacenan insumos.\\
\\
\underline{\smash{deposito}:}\\
Representa un espacio de almacenamiento dentro de una sucursal específica.\\
\\
\underline{\smash{tipo\_insumo}:}\\
Agrupación de distintos productos bajo un criterio común y general.\\
\\
\underline{\smash{insumo}:}\\
Representa una distribución particular de un producto, pudiendo incluir su marca, forma de distribución o cantidad, entre otras.\\
\\
\underline{\smash{provision}:}\\
Representa la existencia de un insumo específico almacenado en un depósito dado.\\
\\
\underline{\smash{reserva}:}\\
Indica la cantidad de un insumo, proveniente de un depósito particular y destinado a la realización de un proyecto.\\
\\
\underline{\smash{novedad}:}\\
Mensaje del sistema que sirve para mantener al tanto a un usuario de sucesos que pueden resultar de interés de acuerdo a sus funciones en la empresa.\\
\\
\underline{\smash{tipo\_tarea}:}\\
Sirve para definir de manera abstracta las características generales de un conjunto de tareas agrupándolas según el tipo de trabajo que se debe llevar a cabo para realizarlas.\\
\\
\underline{\smash{tipo\_tarea\_propiedad}:}\\
Propiedad de un tipo de tarea.\\
\\
\underline{\smash{tipo\_tarea\_propiedad\_opcion}:}\\
Sirve para sugerir al usuario valores frecuentes que puede tomar una propiedad de las tareas pertenecientes a un tipo de tarea dado.\\
\\
\underline{\smash{tipo\_proyecto}:}\\
Sirve para definir de manera abstracta las características generales de un conjunto de proyectos agrupándolos en función del tipo de producto final a elaborar.\\
\\
\underline{\smash{tipo\_tarea\_en\_tipo\_proyecto}:}\\
Indica los tipos de tareas que se deben realizar en un tipo de proyecto.\\
\\
\underline{\smash{tippro\_relacion\_tipos\_tareas}:}\\
Indica el orden de precedencia de los tipos de tareas en un tipo de proyecto.\\
\\
\underline{\smash{proyecto}:}\\
Representa el pedido de un cliente para la consecución de un producto final particular.\\
\\
\underline{\smash{tarea}:}\\
Representa una tarea particular que se debe realizar en un proyecto.\\
\\
\underline{\smash{propiedad\_de\_tarea}:}\\
Indica una propiedad y su valor específico para una tarea.\\
\\
\underline{\smash{pro\_relacion\_tareas}:}\\
Indica el orden de precedencia de las tareas en un proyecto. Esta secuencia se almacena para evitar 	que se pierda la estructura de la diagramación al modificarse la equivalente del tipo de proyecto correspondiente (no se modifica una vez creada con el fin de conservar las estructuras de las diagramaciones existentes para proyectos registrados en la base de datos).\\
\\
\underline{\smash{comentario\_tarea}:}\\
Son los comentarios hechos por los empleados acerca de las distintas tareas que se les han asignado. Sirven para dar cuenta de algún suceso o expresar consideraciones que crean relevantes relacionadas con una tarea dada.




\chapter*{\hypertarget{apendice_h}{}Apéndice H\\Errores encontrados por iteración}
\markboth{Apéndice H - Errores encontrados por iteración}{Apéndice H - Errores encontrados por iteración}
\sectionmark{Apéndice H - Errores encontrados por iteración}
\addcontentsline{toc}{chapter}{H. Errores encontrados por iteración}

\noindent\underline{Iteración I:}
\begin{itemize}
	\item En la pantalla de agregar nuevo depósito, error en la verificación de la falta de elementos en el selector de sucursales (US 6 - I).
	\item En la pantalla de editar cliente se puede seleccionar e incluso eliminar el <<cliente>> Imprenta Lux, necesario para el correcto funcionamiento del sistema (US 18 - I).
	\item No se verifica que los inputs de tipo texto no empiecen, terminen o estén completamente conformados por espacios en blanco (error general observado desde la iteración I).
\end{itemize}
\underline{Iteración II:}
\begin{itemize}
	\item Primary key de provision no tenía id\_cliente, por lo que habían colisiones si es que se intentaba agregar un mismo insumo de un cliente dado a un depósito (US 5 - II).
	\item Errores en ejecuciones de llamadas a bases de datos y manejo de secuencias de las interfaces por mal manejo de hilos en la parte móvil (US 4 - II).
	\item En pantalla móvil de administrar insumo en depósito se debe mostrar la cantidad disponible con el formato X,XX y no X.XX (US 3 - II).
	\item En pantalla de administrar insumo en depósito versión de escritorio, si se modifica el depósito en el selector correspondiente, la tabla no se actualiza (problema no visto en prueba inicial al haberse utilizado un solo depósito por sucursal) (US 5 - II).
	\item En pantalla de administrar insumo en depósito, al intentar agregar un nuevo insumo con valor de input vacío, la cantidad asignada debería ser 0, no null (US 5 - II).
	\item En pantalla de administrar insumo en depósito versión de escritorio, no debería estar habilitado o visible el botón de agregar nuevo insumo a depósito sin tenerse un depósito previamente seleccionado (US 5 - II).
	\item En la pantalla de editar insumo, al modificar el PP del insumo a un valor menor al anterior no se verifica si se ha alcanzado el nuevo PP (US 3 - II).
	\item En la pantalla de editar tipo de insumo, al modificar el PP del tipo de insumo a un valor menor al anterior no se verifica si se ha alcanzado el nuevo PP (US 4 - II).
	\item En la pantalla de modificar cantidad de insumo en depósito, al verificar el PP del insumo y del tipo de insumo se incluyen las cantidades de los insumos pertenecientes a clientes, cuando en realidad solo se deben considerar los insumos de Imprenta Lux (US 5 - II).
	\item El botón <<Cancelar>> de la pantalla de crear tipo de tarea versión de escritorio, debería redirigir a la pantalla de bienvenida, pero no ejecuta ninguna acción (US 23 - II).
\end{itemize}
\underline{Iteración III:}
\begin{itemize}
	\item En la pantalla de buscar proyectos versión de escritorio, no se actualiza la lista sin previamente presionar el botón que elimina los registros mostrados por la tabla de resultados (de proyectos) (US 16 - III).
	\item En pantalla de editar proyecto la fecha de inicio no necesariamente debe ser la actual (US 8 - III).
	\item En pantalla de editar proyecto la fecha de fin solo debería ser modificable si esta ya existía (el proyecto ya estaba en estado finalizado) (US 8 - III).
	\item En la pantalla de visualizar proyecto si el proyecto deja de estar en estado finalizado la fecha de fin debe volver a ser null (US 8 - III).
	\item El perfil de Encargado de Producción no tiene la pantalla de agregar nuevo proyecto (US 8 - III).
	\item El botón <<Cancelar>> de la pantalla de crear tipo de proyecto versión de escritorio, debería redirigir a la pantalla de bienvenida, pero no ejecuta ninguna acción (US 10 - III).
	\item En el refinamiento de la US 12 (asignar tarea a empleado) no se especificó que se notifica al empleado asignado a la realización de una tarea cuando esta le es asignada (US 12 - III).
	\item El botón <<Cancelar>> de la pantalla de visualizar detalles de tarea versión de escritorio, debería redirigir a la pantalla de bienvenida, pero no ejecuta ninguna acción (US 17 - III).
\end{itemize}
\underline{Iteración IV:}
\begin{itemize}
	\item Primary key de reserva no tenía id\_cliente, por lo que había colisiones si es que se intentaba agregar un mismo insumo de un cliente dado a un proyecto (US 9 - IV).
	\item En clase DbcProvision, en la función updateProvisiones, no se disminuía (o eliminaba en caso de la reasignación total) del proyecto la cantidad reasignada arbitrariamente a los depósitos (US 15 - IV).
	\item En pantalla Reasignar insumos de proyecto versión de escritorio, el nombre del depósito era en realidad el nombre de la sucursal (US 15 - IV).
	\item Error en el retorno a la anterior pantalla luego de finalizar un proyecto (US 15 - IV).
	\item En pantalla de visualizar proyecto versión móvil es posible eludir (cerrar) el mensaje de alerta (que consulta acerca de precondiciones para modificar el estado de proyecto) presionando afuera de este, y en tal caso el selector de estado de proyecto no retorna al estado original, sino que se muestra como si el estado se hubiese ya modificado en la base de datos (US 25 - IV).
	\item No debe ser posible asignar recursos a un proyecto si este se encuentra en estado finalizado (US 9 - IV).
	
	\item En la pantalla de agregar insumos a proyecto no se muestra la cantidad de insumos ya reservados de cada depósito (US 9 - IV).
	\item Los empleados asignados a una tarea no pueden dejar comentarios en la tarea ni ver los que ya fueron agregados a ella debido a que la opción correspondiente se colocó únicamente en la pantalla de editar tarea, a la cual los empleados de producción no tienen acceso (US 13 - IV).
	\item Los empleados asignados a una tarea no pueden cambiar el estado de la tarea debido a que la opción correspondiente se colocó únicamente en la pantalla de editar tarea, a la cual los empleados de producción no tienen acceso (US 14 - IV).
	\item Al cambiar el estado de una tarea finalizada que no tiene tareas dependientes ocurre un error en la instrucción SQL ya que se intenta actualizar el estado de tareas cuyos IDs están dentro de un conjunto vacío (US 14 - IV).
	\item Al cambiar el estado de una tarea no se generan las novedades correspondientes debido a que el mensaje de la novedad contiene comillas simples que interfieren con las comillas de la instrucción de inserción SQL (US 14 - IV).
	\item Las novedades generadas al cambiar el estado de tareas se debería indicar que la tarea ha pasado de un estado inicial A a un estado final B. En la práctica las novedades indican que se ha pasado desde un estado final B al mismo estado final B (US 14 - IV).
	\item Al finalizar un proyecto también se genera una novedad para el mismo usuario que finalizó el proyecto, en tanto solo se debería generar para el resto de los usuarios con los perfiles de interés (US 15 - IV).
\end{itemize}
\underline{Iteración V:}
\begin{itemize}
	\item En pantallas de la versión móvil, los botones de eliminar se ocultan en el editor pero no en el template debido al uso de tools:visibility=``gone'' en lugar de android:visibility=``gone'' (USs 19-22 y USs 26-30 - V).
	\item En la pantalla de editar sucursal, al eliminar una sucursal (y sus depósitos) no se informa al usuario (en caso de ser necesario) que no se puede proceder con la eliminación si no queda al menos un depósito perteneciente a otra sucursal al cual elegir como depósito sustituto (US 21 - V).
	\item En la pantalla de editar cliente versión móvil, si no es posible eliminar el cliente por tener proyectos con reservas de insumos de Imprenta Lux, se redirige a la pantalla Buscar proyecto mostrando los proyectos del cliente. La redirección genera un error al intentar acceder a los campos para realizar la búsqueda antes de cargarlos (US 22 - V).
\end{itemize}






\renewcommand{\bibname}{Bibliografía}
\begin{thebibliography}{99}
\addcontentsline{toc}{chapter}{Bibliografía}
%\pagenumbering{gobble} %borra la nomeración de la página

%for the first the first part of the document
\bibitem{Sommerville} Sommerville, Ian. \textit{<<Software Engineering>>}, 9na Edición. Pearson Education, 2010.
\bibitem{Pressman} Pressman, Roger S. \textit{<<Software Engineering: A Practitioner's Approach>>}, 7ma Edición. McGraw-Hill, 2010.
\bibitem{RichardsFord} Richards, Mark - Ford, Neal. \textit{<<Fundamentals of Software Architecture: An Engineering Approach>>}. O'Reilly Media, 2020.
\bibitem{Elmasri} Elmasri, Ramez - Navathe, Shamkant. \textit{<<Fundamentals of Database Systems>>}, 5ta Edición. Pearson Education, 2007.
\bibitem{Silberschatz} Silberschatz, Abraham - Korth, Henry F. - Sudarshan, S. \textit{<<Database System Concepts>>}, 4ta Edición. McGraw-Hill, 2002.
\bibitem{Chapman} Chapman, Stephen N. \textit{<<The Fundamentals of Production Planning and Control>>}. Pearson Education, 2006.
\bibitem{PMBOK} Project Management Institute. \textit{<<A Guide to the Project Management Body of Knowledge (PMBOK Guide)>>}, 5ta Edición. PMI, 2013.
\bibitem{Moreno} Moreno Monsalve, Nelson - Sánchez Ayala, Luz - Velosa García, José. \textit{<<Introducción a la Gerencia de Proyectos, Conceptos y Aplicaciones>>}. Ediciones EAN, 2018.
\bibitem{Clifford} Gray, Clifford F. - Larson, Erik W. \textit{<<Project Management, the Managerial Process>>}, 4ta Edición. McGraw-Hill, 2007.
\bibitem{Martin} Martin, Robert C. \textit{<<Agile Software Development: Principles, Patterns, and Practices>>}.  Pearson Education, 2003.
\bibitem{Kumar} Kumar, Gaurav - Bhatia, Pradeep K. \textit{<<Impact of Agile Methodology on Software Development Process>>}. International Journal of Computer Technology and Electronics Engineering (IJCTEE)
Volume 2, Issue 4, Agosto 2012.
\bibitem{Pokharel} Prabhat Pokharel, Pramesh Vaidya. \textit{<<A Study of User Story in Practice>>}, octubre 2020.
\bibitem{Kustiawan} Yanche Ari Kustiawan, Tek Yong Lim. \textit{<<User Stories in Requirements Elicitation: A Systematic Literature Review>>}, Agosto 2023.
\end{thebibliography}


\end{document}
